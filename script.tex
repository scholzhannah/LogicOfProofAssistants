\documentclass{article}


\setcounter{secnumdepth}{3}     % numbering depth (chapter, section, subsection, ...).
\setcounter{tocdepth}{3}        % numbering depth in table of contents

\usepackage[left=4cm, right=4cm, top=4cm, bottom=4cm]{geometry}
\usepackage{amsmath,amssymb,amsthm,amsfonts,amsbsy,latexsym, mathtools} 
\usepackage[backend=biber, style=alphabetic]{biblatex}  % Extension for bibtex
\bibliography{thesis}                   % The bibliography file that is used
\usepackage[nodayofweek]{datetime}      % To get \today
\usepackage{enumitem}                   % Change \item symbol in enumerate
\setlist[enumerate,1]{label=(\roman*)}  % Get roman numerals as items
\usepackage[T1]{fontenc}                % Better fonts
\usepackage{ifthen}                     % if then
\usepackage{lmodern}                    % Bessere Schrift
\usepackage{listings}                   % Code Listings.
\usepackage{stackrel}                   % write thing above relations
\usepackage[dvipsnames]{xcolor}         % color text
\usepackage{tikz}                       % commutative diagrams
\usepackage{tikz-cd}                    % commutative diagrams
\usepackage{bussproofs}                 % To write prooftrees
\usepackage{stmaryrd}                   % To get double brackets
\usepackage[many]{tcolorbox}    	    % for COLORED BOXES (tikz and xcolor included)
\usepackage{mathspec} 			        % for FONTS
\usepackage{setspace}                   % for LINE SPACING
\usepackage{multicol}                   % for MULTICOLUMNS
\usepackage{environ}                    % better custom enviroments
\usepackage{adjustbox}                  % to get custom alignment of diagrams
\usepackage{xparse}                     % to define the claim environment

% to get upright greek letters
\usepackage{upgreek}

% To break urls correctly in bibliography
\setcounter{biburllcpenalty}{7000}
\setcounter{biburlucpenalty}{8000}

% to get rid of an error
\usepackage{bookmark}

%to get code in footnotes write \cprotect\footnote 
\usepackage{cprotect}

% to get the external link symbol
\usepackage{fontawesome}

% Last package!!!!
\usepackage{hyperref}                   
\hypersetup{                            % options
    colorlinks=true,                    % colours
    linkcolor=black,                     % colour for links
    citecolor=black,                     % colour for refrences
    urlcolor=black                       % colour for links
}

% Boxes 

\setlength\parindent{0pt}   % killing indentation for all the text
\setstretch{1.3}            % setting line spacing to 1.3
\setlength\columnsep{0.25in} % setting length of column separator
\pagestyle{empty}           % setting pagestyle to be empty

\tcbset{
    sharp corners,
    colback = white,
    before skip = 0.2cm,    % add extra space before the box
    after skip = 0.5cm      % add extra space after the box
}                           % setting global options for tcolorbox

\definecolor{maindef}{HTML}{5989cf}    % setting main color to be used
\definecolor{subdef}{HTML}{e5f1ff}     % setting sub color to be used

\newtcolorbox{boxdef}{
    colback = subdef, 
    colframe = maindef, 
    boxrule = 0pt, 
    leftrule = 6pt, % left rule weight
    breakable
}


\definecolor{mainthe}{HTML}{73937E}    % setting main color to be used
\definecolor{subthe}{HTML}{eef7f1}     % setting sub color to be used

%\definecolor{mainthe}{HTML}{D387AB}    % setting main color to be used
%\definecolor{subthe}{HTML}{fff1fa}     % setting sub color to be used

\newtcolorbox{boxthe}{
    colback = subthe, 
    colframe = mainthe, 
    boxrule = 0pt, 
    leftrule = 6pt, % left rule weight
    breakable,
}


% Environments
\theoremstyle{definition}              
\newtheorem{defi}{Definition}[section] 
\newtheorem*{defi*}{Definition}
\newtheorem{example}[defi]{Example}
\newtheorem{notation}[defi]{Notation}
\newtheorem{rem}[defi]{Remark}
\newtheorem{defcor}[defi]{Definition/Corollary}
\newtheorem{defprop}[defi]{Definition/Proposition}
\newtheorem{defthm}[defi]{Definition/Theorem}

\theoremstyle{plain}
\newtheorem*{conj}{Conjecture}
\newtheorem{cor}[defi]{Corollary}
\newtheorem{lem}[defi]{Lemma}
\newtheorem{prop}[defi]{Proposition}
\newtheorem*{prop*}{Proposition}
\newtheorem{thm}[defi]{Theorem}
\newtheorem*{thm*}{Theorem}
\newtheorem{exercise}[defi]{Exercise}

% things with boxes build in
\NewEnviron{boxthm}[1][]{
    \begin{boxthe}
        \begin{thm}[#1]
            \BODY
        \end{thm}
    \end{boxthe}
}

\NewEnviron{boxdefi}{
    \begin{boxdef}
        \begin{defi}
            \BODY
        \end{defi}
    \end{boxdef}
}

\NewEnviron{boxprop}[1][]{
    \begin{boxthe}
        \begin{prop}[#1]
            \BODY
        \end{prop}
    \end{boxthe}
}

\NewEnviron{boxlem}[1][]{
    \begin{boxthe}
        \begin{lem}[#1]
            \BODY
        \end{lem}
    \end{boxthe}
}

\NewEnviron{boxcor}[1][]{
    \begin{boxthe}
        \begin{cor}[#1]
            \BODY
        \end{cor}
    \end{boxthe}
}

% Claim environment (nestable once)
\begin{comment}
    The syntax is: 
    \begin{claim}
        Claim statement
        \begin{proof}
            Proof of claim.
        \end{proof}
    \end{claim}
\end{comment}

\newlist{Claim}{description}{2}
\setlist[Claim]{labelindent=2em,leftmargin=*}
\newif\ifInsideClaim
\newcounter{claim}[defi]
\newcounter{cclaim}[claim]
\renewcommand\theclaim{\arabic{claim}}
\renewcommand\thecclaim{\arabic{claim}.\arabic{cclaim}}
\let\originalqedsymbol\qedsymbol
\newenvironment{claim}{%
  % disable qed symbol if there is no star
  \let\qedsymbol\relax%
  \ifInsideClaim% we have a nested environment
    \refstepcounter{cclaim}%
    \let\theclaimcounter\thecclaim%
  \else%
    \refstepcounter{claim}%
    \let\theclaimcounter\theclaim%
    \InsideClaimtrue%
  \fi%
  \Claim\item[\textbf{Claim \theclaimcounter:}]%
}{\endClaim\InsideClaimfalse\let\qedsymbol\originalqedsymbol}

%macros
\newcommand{\bA}{\mathbb{A}}
\newcommand{\bB}{\mathbb{B}}
\newcommand{\bC}{\mathbb{C}}
\newcommand{\bD}{\mathbb{D}}
\newcommand{\bE}{\mathbb{E}}
\newcommand{\bF}{\mathbb{F}}
\newcommand{\bG}{\mathbb{G}}
\newcommand{\bH}{\mathbb{H}}
\newcommand{\bI}{\mathbb{I}}
\newcommand{\bJ}{\mathbb{J}}
\newcommand{\bK}{\mathbb{K}}
\newcommand{\bL}{\mathbb{L}}
\newcommand{\bM}{\mathbb{M}}
\newcommand{\bN}{\mathbb{N}}
\newcommand{\bO}{\mathbb{O}}
\newcommand{\bP}{\mathbb{P}}
\newcommand{\bQ}{\mathbb{Q}}
\newcommand{\bR}{\mathbb{R}}
\newcommand{\bS}{\mathbb{S}}
\newcommand{\bT}{\mathbb{T}}
\newcommand{\bU}{\mathbb{U}}
\newcommand{\bV}{\mathbb{V}}
\newcommand{\bW}{\mathbb{W}}
\newcommand{\bX}{\mathbb{X}}
\newcommand{\bY}{\mathbb{Y}}
\newcommand{\bZ}{\mathbb{Z}}

\newcommand{\fC}{\mathcal{C}}
\newcommand{\fL}{\mathcal{L}}
\newcommand{\fF}{\mathcal{F}}
\newcommand{\fR}{\mathcal{R}}
\newcommand{\fV}{\mathcal{V}}
\newcommand{\fM}{\mathcal{M}}
\newcommand{\fP}{\mathcal{P}}


\DeclareMathOperator{\aeq}{\equiv_{\alpha}} %alpha-equivalence
\newcommand{\alert}[1]{{\color{blue}{\ifmmode\mathbf{#1}\else\textbf{#1}\fi}}}
\DeclareMathOperator{\cupdot}{\dot{\cup}} % Disjoint union
\DeclareMathOperator{\smodels}{\models_{\sigma}} %models with variable assignment

\newcommand{\pow}[1]{\fP(#1)}
\newcommand{\fv}[1]{\text{fv}(#1)} %free variables
\newcommand{\abs}[1]{\lvert #1 \rvert} % abs value (underlying set of a model)
\newcommand{\interpret}[1]{\llbracket #1 \rrbracket} % interpretation in a model
\newcommand{\vect}[1]{\bar{#1}}
\newcommand{\menquote}[1]{\ensuremath{\text{``} #1 \text{''}}} % quotes in math mode

\newcommand{\pr}[2]{\text{pr}_{#1}(#2)} %projection
\newcommand{\compl}[1]{#1^{c}} %complement
\newcommand{\id}{\text{id}} %identity
\newcommand{\preim}[2]{#1^{-1}(#2)} %preimage
\newcommand{\interior}[1]{\text{int}(#1)} %interior
\newcommand{\closure}[1]{\overline{#1}} %closure
\newcommand{\boundary}{\partial} %boundary
\newcommand{\restrict}[2]{\ensuremath{\left.#1\right|_{#2}}} %restriction
\newcommand{\seqclosure}[1]{\text{scl}(#1)} %sequential closure
\newcommand{\oball}[2]{\text{B}_{#1}(#2)} %open ball
\newcommand{\closedCell}[2]{\closure{e}_{#2}^{#1}} %closed n-cell
\newcommand{\openCell}[2]{e_{#2}^{#1}} %open n-cell
\newcommand{\cellFrontier}[2]{\partial e_{#2}^{#1}} %edge n-cell
\newcommand{\closedCellf}[2]{\closure{f}_{#2}^{#1}} %closed n-cell
\newcommand{\openCellf}[2]{f_{#2}^{#1}} %open n-cell
\newcommand{\cellFrontierf}[2]{\partial f_{#2}^{#1}} %edge n-cell
\newcommand{\norm}[1]{\left\lVert#1\right\rVert} %norm
\newcommand{\maximum}[2]{\text{max}(#1, #2)} %maximum



\title{Logic of Proof Assistants}
\author{Prof. Floris van Doorn \thanks{\LaTeX-realization by Hannah Scholz} \\ University of Bonn}

\begin{document}

\maketitle
\tableofcontents        % table of contents

%introduction
\clearpage
\section{Introduction}

The topics of this class are: 
\begin{enumerate}
    \item First-order Logic/Set Theory
    \item Lambda Calculus
    \item Simple Type Theory (Higher-Order Logic)
    \item Dependent Type Theory/Homotopy Type Theory
\end{enumerate}

\begin{example}
    Here are examples of proof assistants for these different types of logics: 
    \begin{enumerate}
        \item First-order Logic/Set Theory: Mizar, Metamath
        \item Simple Type Theory: Isabelle/HoL, HoL Light
        \item Dependent Type Theory: Lean, Rocq (formerly Coq), Agda
        \item Homotopy Type Theory: cubicaltt, rzk
    \end{enumerate}
\end{example}

\begin{rem}
    You might want to have the following criteria for a logic: 
    \begin{enumerate}
        \item Appropriate (You can encode mathematical arguments.)
        \item Simple (It is relatively easy to understand.)
        \item Expressive (Mathematical arguments are convenient to express.)
    \end{enumerate}
\end{rem}

\begin{thm}
Let $\pi$ be the prime counting function, i.e. $\pi \colon \bR \to \bN$, $x \mapsto \lvert \{ p \le x \mid p \text{\textup{ prime}} \} \rvert$.
    Then $\lim_{x \to \infty} \frac{\pi (x)}{x / \log(x)} = 1$.
\end{thm}

\begin{rem}
When formalizing/stating this theorem in a formal logic there are a few things that you need to think about: 
\begin{enumerate}
    \item What do you do about division by zero?
    \item What does division even mean? (Do you define division for $\bR$ explicitly? 
        Do you define it generally for a field? 
        Or even for a group? 
        How do you ensure that the ``correct'' field structure on $\bR$ gets used?)
    \item How do you define a limit? (Do you define a limit for $\bR$ explicitly? 
        Or for every topological space?
        How do you ensure the ``correct'' topology on $\bR$ gets used? 
        How do you deal with potentially non-unique limits (for example in non-Hausdorff spaces)?)
\end{enumerate}
\end{rem}

\begin{rem}
You can make the following design choices for ``a logic'': 
\begin{enumerate}
    \item Is the logic typed or untyped?
    \item Is the logic constructive or classical?
    \item Does the logic support computation?
\end{enumerate}
\end{rem}

\begin{rem}
In logic there is the \alert{object language} and we reason about it in a \alert{meta-language} (``ordinary mathematical reasoning'').
\end{rem}

\subsection{Inductive Definitions}

\begin{example}
    The natural numbers are inductively defined by $0 \in \bN$ and $S \colon \bN \to \bN$, $n \mapsto n + 1$. 
\end{example}

\begin{boxdef}
\begin{defi}
    Let $U$ be a set and $\fC \subseteq \bigcup_{n \in \bN} (U^n \to U)$ a set of \alert{constructors},
    where $c \colon U^n \to U$ is called an \alert{$n$-ary function} and $(U^n \to U)$ is the collection of $n$-ary functions.
    \begin{enumerate}
        \item $A \subseteq U$ is \alert{closed under $\fC$} if for any $n$-ary $c \in \fC$ and for all $x_1, \dots , x_n \in A$ we have that $c(x_1, \dots, x_n) \in A$.
        \item $A \subseteq U$ is  \alert{generated by $\fC$} or \alert{inductively defined by $\fC$} if $A$ is the smallest set that is closed under $\fC$, i.e. $A = \bigcap \{B \subseteq U \mid B \text{ is closed under } \fC\}$.
        \item $A \subseteq U$ is \alert{freely generated by $\fC$} if 
            \begin{enumerate}
                \item each constructor is injective on $A^n$ and
                \item the images of different constructors are disjoint.
            \end{enumerate}
    \end{enumerate}
\end{defi}
\end{boxdef}

\begin{rem}
$\varnothing$ is closed under $\fC$ iff $\fC$ has no nullary constructors.
\end{rem}

\begin{exercise}
    $\bigcap \{B \subseteq U \mid B \text{\textup{ is closed under }} \fC\}$ is closed under $\fC$.
\end{exercise}

\begin{example}
    \hfill
    \begin{enumerate}
        \item The free group.
        \item The $\sigma$-algebra generated by a collection of subsets. (This is not freely generated.)
        \item The topology generated by a collection of subsets. (This is not freely generated.)
    \end{enumerate}
\end{example}

\begin{boxthe}
\begin{thm}[Structural Induction]
    If $A \subseteq U$ is generated by $\fC$ and $P \colon A \to \{ \top, \bot\}$ is a predicate on $A$, to prove $\forall a \in A, P(a)$ it suffices to show: for any $n$-ary $c \in \fC$ and any $x_1, \dots, x_n \in A$ if $P(x_1), \dots, P(x_n)$ then $P(c(x_1, \dots, x_n))$.
\end{thm}
\end{boxthe}
\begin{proof}
    Exercise.
\end{proof}

\begin{rem}
    The base case of the induction is given by nullary constructors.
\end{rem}

\begin{boxthe}
\begin{thm}[Structural Recursion]
    If $A \subseteq U$ is freely generated by $\fC$, $B$ is a set and for any $n$-ary $c \in \fC$ we have a $g_c \colon B^n \to B$ then there is a unique function $f \colon A \to B$ such that $f(c(a_1, \dots, a_n)) = g_c(f(a_1), \dots, f(a_n))$ for every $c \in \fC$ and $a_1, \dots a_n \in A$. 
\end{thm}
\end{boxthe}
\begin{proof}
Exercise.
\end{proof}

\begin{example}
    For $A = \bN$ this reduces to $f(0) \coloneq g_0$ and $f(S(n)) \coloneq g_s(f(n))$.
\end{example}

\clearpage
\section{First-Order Logic}

\begin{boxdef}
\begin{defi}
A (first-order) \alert{language $\fL$} is a triple $(\fF, \fR, a)$ where $\fF$ is a set of function symbols, $\fR$ is a set of relation symbols, $\fF$ and $\fR$ are disjoint and $a \colon \fF \cup \fR \to \bN$ is the arity function.
\end{defi}
\end{boxdef}

\begin{example}
A language for groups $\fL_{\text{Group}}$ has $\fF \coloneq \{\cdot, ^{-1}, 1 \}$, $\fR \coloneq \varnothing$, $a(\cdot) = 2$, $a(^{-1}) = 1$ and $a(1) = 0$.
\end{example}

\begin{boxdef}
\begin{defi}
We fix an infinite set of \alert{variables} $\alert{\fV} \coloneq \{ x_0, x_1, \dots \}$.
\end{defi}
\end{boxdef}

\begin{rem}
We use $x$ for variables, $f$ and $g$ for functions and $R$ and $S$ for relations.
\end{rem}

\begin{boxdef}
\begin{defi}
We can define the \alert{terms $T_{\fL}$} in the language $\fL$ using the \alert{Backus–Naur form (BNF)}: 
\begin{equation*}
    s, t \Coloneqq x \mid f(t_1, \dots, t_n)
\end{equation*}
where $f$ is an $n$-ary function symbol.
\end{defi}
\end{boxdef}

\begin{boxdef}
\begin{defi}
Formally, we define the \alert{terms $T_{\fL}$} in the language $\fL$ in the following way. We define the set of \alert{symbols} $S \coloneq \fF \cupdot \fV \cupdot \{\menquote{(}, \menquote{)}, \menquote{,} \}$ and the set of finite sequences of symbols $S^*$. 
Let $\fC$ be defined as: 
\begin{enumerate}
    \item for each variable $x \in \fV$ there is a nullary constructor $c_x \coloneq x$
    \item for each $n$-ary function symbol $f$ there is an $n$-ary constructor $c_f \colon (S^*)^n \to S^*$, $c_f(t_1, \dots, t_n) \coloneq f\menquote{(}t_1\menquote{,} \dots \menquote{,} t_n \menquote{)}$
\end{enumerate}
Then $T_{\fL} \subseteq S^*$ is the set generated by $\fC$.
\end{defi}
\end{boxdef}

\begin{example}
    \hfill
    \begin{enumerate}
        \item $\menquote{(}\menquote{)}\menquote{,}f$ is in $S^*$ but not in $T_{\fL}$. 
        \item If $f$ is binary then $f\menquote{(}x_0 \menquote{,} x_1 \menquote{)}$ is in $T_{\fL}$.
    \end{enumerate}
\end{example}

\begin{rem}
Technically, the brackets and commas are not necessary. 
They are however necessary when you use infix notation. 
(For example the meaning of $a \cdot b + c$ is unclear.)
\end{rem}

\begin{boxdef}
\begin{defi}
First-order \alert{formulas $\Phi_{\fL}$} are specified by
\begin{equation*}
    \varphi, \psi \Coloneqq \bot \mid s = t \mid R(t_1, \dots ,t_n) \mid (\varphi \wedge \psi) \mid (\varphi \lor \psi) \mid (\varphi \to \psi) \mid (\forall x. \varphi) \mid (\exists x. \varphi)
\end{equation*}
where $\fR$ is an $n$-ary relation symbol and $t_1, \dots, t_n \in T_{\fL}$.
\end{defi}
\end{boxdef}

\begin{rem}
In classical logic one could omit the rules $(\varphi \wedge \psi)$ and $(\varphi \lor \psi)$ (as they can be defined using the other rules). 
They are however necessary for constructive logic.
\end{rem}

\begin{rem}
We can define other connectives: 
\begin{enumerate}
    \item $\neg \varphi \coloneq (\varphi \to \bot)$
    \item $\varphi \leftrightarrow \psi \coloneq ((\varphi \to \psi) \wedge (\psi \to \varphi))$
\end{enumerate}
\end{rem}

\begin{rem}
When writing formulas we omit some parentheses: 
\begin{enumerate}
    \item $\varphi \to \psi \to \theta$ means $\varphi \to (\psi \to \theta)$
    \item $\forall x. \varphi \to \psi$ means $\forall x. (\varphi \to \psi)$
\end{enumerate}
\end{rem}

\begin{rem}
We want $\forall x. x = x$ and $\forall y. y = y$ to mean the same thing. 
Options to achieve this are: 
\begin{enumerate}
    \item Define $(\forall x. x = x) \aeq (\forall y. y = y)$ to be \alert{$\alpha$-equivalent}. And then define the set of formulas to be $\Phi_{\fL} / \aeq$.
    \item We could not use variable names for bound variables and use \alert{de Bruijn indices} instead.
\end{enumerate}
\end{rem}

\begin{rem}
    $\forall x. x = y$ has \alert{bound variables} $\{x\}$ and \alert{free variables} $\{y\}$. For a formula $\varphi$ or a term $t$ we also write \alert{$\fv{\varphi}$} and \alert{$\fv{t}$} for the set of free variables in $\varphi$ and $t$.
\end{rem}

\begin{boxdef}
\begin{defi}
    A \alert{sentence} is a formula without free variables.
\end{defi}
\end{boxdef}

\begin{boxdef}
\begin{defi}
    \alert{Substitution $s[t/x]$} of $x$ by $t$ in a term $s$ is defined recursively by 
    \begin{enumerate}
        \item {$ y[t/x] \coloneq 
                \begin{cases}
                    t & \text{if } y = x \\
                    y & \text{otherwise}
                \end{cases}$}
        \item $f(s_1, \dots, s_n)[t/x] \coloneq f(s_1[t/x], \dots, s_n[t/x])$
    \end{enumerate}
\end{defi}
\end{boxdef}

\begin{example}
    Defining substitution in formulas is a little bit harder as we need to avoid \alert{variable capture}:
    $(\exists x. x \leq z)[(x + 1)/z]$ should not be $\exists x. x \leq x + 1$ but $\exists y. y \leq x + 1$.
\end{example}

\begin{boxdef}
\begin{defi}
    For a formula $\varphi$ \alert{substitution $\varphi[t/x]$} is defined as: 
    \begin{enumerate}
        \item $(s = s')[t/x] \coloneq (s[t/x] = s'[t/x])$
        \item $R(t_1, \dots, t_n)[t/x] \coloneq R(t_1[t/x], \dots, t_n[t(x)])$
        \item $(\varphi \lor \psi)[t/x] \coloneq (\varphi[t/x] \lor \psi[t/x])$
        \item $(\varphi \wedge \psi)[t/x] \coloneq (\varphi[t/x] \wedge \psi[t/x])$
        \item $(\varphi \to \psi)[t/x] \coloneq (\varphi[t/x] \to \psi[t/x])$
        \item {$ (\forall y. \varphi)[t/x] \coloneq 
            \begin{cases}
                \forall y. \varphi & \text{if } y = x \\
                \forall z. \varphi[z/y][t/x] & \text{otherwise}
            \end{cases}$ where $z$ does not occur in $t$.}
        \item {$ (\exists y. \varphi)[t/x] \coloneq 
        \begin{cases}
            \exists y. \varphi & \text{if } y = x \\
            \exists z. \varphi[z/y][t/x] & \text{otherwise}
        \end{cases}$ where $z$ does not occur in $t$.}
    \end{enumerate}
\end{defi}
\end{boxdef}

\begin{boxdef}
\begin{defi}
    \alert{$\alpha$-equivalence} is the \alert{congruence closure} of 
    \begin{enumerate}
        \item $(\forall x. \varphi) \aeq (\forall y. \varphi[y/x])$
        \item $(\exists x. \varphi) \aeq (\exists y. \varphi[y/x])$
    \end{enumerate}
    i.e. it is the smallest equivalence relation containing these two rules and respecting the connectives: 
    \begin{enumerate}
        \item $(\varphi_1 \wedge \varphi_2) \aeq (\psi_1 \wedge \psi_2)$ for $\varphi_1 \aeq \psi_1$ and $\varphi_2 \aeq \psi_2$
        \item $(\varphi_1 \lor \varphi_2) \aeq (\psi_1 \lor \psi_2)$ for $\varphi_1 \aeq \psi_1$ and $\varphi_2 \aeq \psi_2$
        \item $(\varphi_1 \to \varphi_2) \aeq (\psi_1 \to \psi_2)$ for $\varphi_1 \aeq \psi_1$ and $\varphi_2 \aeq \psi_2$
        \item $(\forall x. \varphi) \aeq (\forall x. \psi)$ if $\varphi \aeq \psi$
        \item $(\exists x. \varphi) \aeq (\exists x. \psi)$ if $\varphi \aeq \psi$
    \end{enumerate}
\end{defi}
\end{boxdef}

\begin{rem}
    We will treat $\alpha$-equivalence as an equivalence relation. 
    You could also define the formulas as $\Phi_{\fL} / \aeq$ and thus treat $\alpha$-equivalence as equality.
\end{rem}

\subsection{Provability}

\begin{boxdef}
\begin{defi}
    Let $\Gamma$ be a set of formulas and $\varphi$ a formula.
    Then $\alert{\Gamma \vdash \varphi}$ (read: ``$\Gamma$ proves $\varphi$'') is defined inductively by 
    \begin{enumerate}
        \item {
            \AxiomC{}
            \UnaryInfC{$\Gamma, \varphi \vdash \varphi$}
            \DisplayProof
            (assumption rule)}
        \item {
            \AxiomC{$\Gamma \vdash \varphi$}
            \AxiomC{$\Gamma \vdash \psi$}
            \BinaryInfC{$\Gamma \vdash \varphi \wedge \psi$}
            \DisplayProof
            ($\wedge$-introduction)}
        \item {
            \AxiomC{$\Gamma \vdash \varphi_1 \wedge \varphi_2$}
            \UnaryInfC{$\Gamma \vdash \varphi_i$}
            \DisplayProof
            for $i = 1, 2$ 
            ($\wedge$-elimination)}
        \item {
            \AxiomC{$\Gamma \vdash \varphi_i$}
            \UnaryInfC{$\Gamma \vdash \varphi_1 \lor \varphi_2$}
            \DisplayProof
            for $i = 1, 2$
            ($\lor$-introduction)}
        \item{
            \AxiomC{$\Gamma \vdash \varphi \lor \psi$}
            \AxiomC{$\Gamma, \varphi \vdash \theta$}
            \AxiomC{$\Gamma, \psi \vdash \theta$}
            \TrinaryInfC{$\Gamma \vdash \theta$}
            \DisplayProof
            ($\lor$-elimination)}
        \item{
            \AxiomC{$\Gamma, \varphi \vdash \psi$}
            \UnaryInfC{$\Gamma \vdash \varphi \to \psi$}
            \DisplayProof
            ($\to$-introduction)}
        \item {
            \AxiomC{$\Gamma \vdash \varphi \to \psi$}
            \AxiomC{$\Gamma \vdash \varphi$}
            \BinaryInfC{$\Gamma \vdash \psi$}
            \DisplayProof
            ($\to$-elimination)}
        \item {
            \AxiomC{$\Gamma, \neg \varphi \vdash \bot$}
            \UnaryInfC{$\Gamma \vdash \varphi$}
            \DisplayProof
            (proof by contradiction)}
        \item {
            \AxiomC{$\Gamma \vdash \varphi$}
            \UnaryInfC{$\Gamma \vdash \forall x. \varphi$}
            \DisplayProof
            for $x \notin \fv{\Gamma}$
            ($\forall$-introduction)}
        \item {
            \AxiomC{$\Gamma \vdash \forall x. \varphi$}
            \UnaryInfC{$\Gamma \vdash \varphi[t/x]$}
            \DisplayProof
            ($\forall$-elimination)}
        \item{
            \AxiomC{$\Gamma \vdash \varphi[t/x]$}
            \UnaryInfC{$\Gamma \vdash \exists x. \varphi$}
            \DisplayProof
            ($\exists$-introduction)}
        \item {
            \AxiomC{$\Gamma \vdash \exists x. \varphi$}
            \AxiomC{$\Gamma, \varphi \vdash \psi$}
            \BinaryInfC{$\Gamma \vdash \psi$}
            \DisplayProof
            for $x \notin \fv{\Gamma, \psi}$
            ($\exists$-elimination)}
        \item {
            \AxiomC{}
            \UnaryInfC{$\Gamma \vdash t = t$}
            \DisplayProof
            ($=$-introduction)}
        \item {
            \AxiomC{$\Gamma \vdash s = t$}
            \AxiomC{$\Gamma \vdash \varphi[t/x]$}
            \BinaryInfC{$\Gamma \vdash \varphi[s/x]$}
            \DisplayProof
            ($=$-elimination)}
        \item {
            \AxiomC{$\Gamma \vdash \varphi$}
            \UnaryInfC{$\Gamma \vdash \psi$}
            \DisplayProof
            for $\varphi \aeq \psi$
            ($\alpha$-equivalence)}
    \end{enumerate}
\end{defi}
\end{boxdef}

\begin{rem}
    Read ``\AxiomC{A} \UnaryInfC{B} \DisplayProof'' as: ``Under the assumptions A we can prove B''.
    With ``$\Gamma, \varphi$'' we really mean $\Gamma \cup \{\varphi\}$.
\end{rem}

\begin{example}
    If $\varphi$ and $\psi$ are formulas then we can show $\vdash (\varphi \wedge \psi) \to (\psi \wedge \varphi)$ using the following \alert{proof tree} : 
    \begin{prooftree}
        \AxiomC{}
        \RightLabel{Assump.}
        \UnaryInfC{$\varphi \wedge \psi \vdash \varphi \wedge \psi$}
        \RightLabel{$\wedge$-elim.}
        \UnaryInfC{$\varphi \wedge \psi \vdash \psi$}
        \AxiomC{}
        \RightLabel{Assump.}
        \UnaryInfC{$\varphi \wedge \psi \vdash \varphi \wedge \psi$}
        \RightLabel{$\wedge$-elim.}
        \UnaryInfC{$\varphi \wedge \psi \vdash \varphi$}
        \RightLabel{$\wedge$-intro.}
        \BinaryInfC{$\varphi \wedge \psi \vdash \psi \wedge \varphi$}
        \RightLabel{$\to$-intro.}
        \UnaryInfC{$\vdash (\varphi \wedge \psi) \to (\psi \wedge \varphi)$}
    \end{prooftree}
\end{example}

\subsection{Semantics}

\begin{boxdef}
\begin{defi}
    An \alert{$\fL$-structure} $\fM$ consists of 
    \begin{enumerate}
        \item a non-empty set $\abs{\fM}$
        \item for any $n$-ary function symbol $f$ a function $f_{\fM} \colon \abs{\fM}^n \to \abs{\fM}$
        \item for any $n$-ary relation symbol $R$ a set $R_{\fM} \subseteq \abs{\fM}^n$
    \end{enumerate}
\end{defi}
\end{boxdef}

\begin{boxdef}
\begin{defi}
    If $t$ is an $\fL$-term and $\sigma \colon \fV \to \abs{\fM}$ we define \alert{$\interpret{t}_{\fM, \sigma}$} as: 
    \begin{enumerate}
        \item $\interpret{x}_{\fM, \sigma} \coloneq \sigma(x)$
        \item $\interpret{f(t_1, \dots, t_n)}_{\fM, \sigma} \coloneq f_{\fM}(\interpret{t_1}_{\fM, \sigma}, \dots, \interpret{t_n}_{\fM, \sigma})$
    \end{enumerate}
    For formulas we define recursively that \alert{$\fM \smodels \varphi$} holds if
    \begin{enumerate}
        \item $\fM \smodels R(t_1, \dots, t_n)$ iff $R_{\fM}(\interpret{t_1}_{\fM, \sigma}, \dots, \interpret{t_n}_{\fM, \sigma})$
        \item $\fM \smodels \bot$ never holds
        \item $\fM \smodels s = t$ iff $\interpret{s}_{\fM, \sigma} = \interpret{t}_{\fM, \sigma}$
        \item $\fM \smodels \varphi \wedge \psi$ iff $\fM \smodels \varphi$ and $\fM \smodels \psi$
        \item $\fM \smodels \varphi \lor \psi$ iff $\fM \smodels \varphi$ or $\fM \smodels \psi$
        \item $\fM \smodels \varphi \to \psi$ iff $\fM \smodels \varphi$ implies $\fM \smodels \psi$
        \item $\fM \smodels \forall x. \varphi$ iff for all $a \in \abs{\fM}$ we know that $\fM \operatorname{\models_{\sigma, x \mapsto a}} \varphi$ where 
        
        $(\sigma, x \mapsto a)(y) \coloneq
        \begin{cases}
            a & y = x \\
            \sigma(y) & \text{otherwise}
        \end{cases}$
        \item $\fM \smodels \exists x. \varphi$ iff there is $a \in \abs{\fM}$ such that $\fM \operatorname{\models_{\sigma, x \mapsto a}} \varphi$
    \end{enumerate}
\end{defi}
\end{boxdef}

\begin{rem}
    We write \alert{$\varphi(\vec{x})$} to mean that $\fv{\varphi} \subseteq \vec{x}$ and \alert{$\varphi(\vec{t})$} for $\varphi[\vec{t}/\vec{x}]$.
\end{rem}

\begin{rem}
    $\interpret{t}_{\fM, \sigma}$ and $\fM \smodels \varphi$ only depend on the values $\sigma(x)$ where $x \in \fv{t}$ and $x \in \fv{\varphi}$ respectively.
    If $\varphi$ is a sentence then $\fM \smodels \varphi$ does not depend on $\sigma$ and is denoted \alert{$\fM \models \varphi$} (read: ``$\fM$ realizes $\varphi$'').
\end{rem}

\begin{boxdef}
\begin{defi}
    If $\Gamma$ is a set of formulas and $\varphi$ is a formula, then \alert{$\Gamma \models \varphi$} means that for any $\fL$-structure $\fM$ and assignment $\sigma \colon \fV \to \abs{\fM}$ such that $\fM \smodels \psi$ for all $\psi \in \Gamma$ we have $\fM \smodels \varphi$.
\end{defi}
\end{boxdef}

\begin{boxthe}
\begin{thm}[Soundness theorem]
    If $\Gamma \vdash \varphi$ then $\Gamma \models \varphi$.
\end{thm}
\end{boxthe}

\begin{boxthe}
\begin{thm}[Completeness theorem]
    If $\Gamma \models \varphi$ then $\Gamma \vdash \varphi$.
\end{thm}
\end{boxthe}

\begin{boxthe}
\begin{thm}[Compactness theorem]
    If $\Gamma \models \varphi$ then for some finite $\Gamma' \subseteq \Gamma$ we have $\Gamma' \models \varphi$.
\end{thm}
\end{boxthe}

\subsection{Definite descriptions}

\begin{boxdef}
\begin{defi}
    $\alert{\exists!}x. \varphi(x, \vec{z}) \coloneq \exists x. (\varphi(x, \vec{z}) \wedge \forall y. \varphi(y, \vec{z}) \to y = x)$.
\end{defi}
\end{boxdef}

\begin{boxdef}
\begin{defi}
    Suppose $\Gamma$ is a set of $\fL$-sentences, $\Gamma'$ a set of $\fL'$-sentences and $\fL \subseteq \fL'$.
    Then \alert{$\Gamma'$ is conservative over $\Gamma$} if $\Gamma \subseteq \Gamma'$ and for all $\fL$-formulas $\psi$ such that $\Gamma' \vdash \psi$ we have $\Gamma \vdash \psi$. 
\end{defi}
\end{boxdef}

\begin{boxthm}
    Suppose that $\Gamma \vdash \forall \vec{x}. \exists! y. \varphi(\vec{x}, y)$ and that $f$ is a fresh function symbol (i.e.\ not among the function symbols of $\fL$) then $\Gamma \cup \{\forall \vec{x}. \varphi(\vec{x}, f(\vec{x}))\}$ is conservative over $\Gamma$.
\end{boxthm}

\begin{boxdef}
\begin{defi}[Axioms of ZFC]
    \alert{$\fL_{ZFC}$} has no function symbol and one binary relation ``$\in$''. 
    The axioms of ZFC are
    \begin{enumerate}
        \item Extensionality : $\forall x \forall y. (\forall z. z \in x \leftrightarrow z \in y) \to x = y$
        \item Pairing: $\forall x \forall y \exists z \forall w. w \in z \leftrightarrow w = x \lor w = y$ (``$z = \{x,y\}$'')
        
        This allows us to define $\{x\} \coloneq \{x, x\}$.
        \item Union: $\forall x \exists y \forall z. z \in y \leftrightarrow \exists w. (w \in x \wedge z \in w)$ (``$y = \bigcup x$'')
        
        This allows us to define $x \cup y \coloneq \bigcup \{x, y\}$.
        \item {Power set: $\forall x \exists y \forall z. z \in y \leftrightarrow z \subseteq x$ (``$y = \pow{x}$'')
            
        where $z \subseteq x$ means $\forall w. w \in z \to w \in x$}
        \item Separation (axiom schema): for any formula $\varphi(\vec{x}, y)$ we have $\forall \vec{x} \forall y \exists z \forall w. w \in z \leftrightarrow (w \in y \wedge \varphi(\vec{x}, w))$ (``$z = \{ w \in y \mid \varphi(\vec{x}, w)\}$'')
        \item Infinity: $\exists x. \varnothing \in x \wedge \forall y. y \in x \to y \cup \{y\} \in x$ where $\varnothing \coloneq \{w \in y \mid \bot\}$
        \item Foundation: $\forall x. (\exists y. y \in x) \to \exists y. y \in x \wedge \forall z. z \in x \to z \notin y$ (``Every set $x$ contains an element $y$ disjoint from $x$'')
        \item Replacement (axiom schema): For every formula $\varphi(z, w, \vec{y})$ we have $\forall x \forall \vec{y} (\forall z. z \in x \to \exists! w \varphi(z, w, \vec{y})) \to \exists u \forall w. w \in u \leftrightarrow \exists z. z \in x \wedge \varphi(z, w, \vec{y})$ (``If $\varphi$ is a function with domain $x$ then the image of $\varphi$ is a set.'')
        \item Choice: $\forall x. \varnothing \notin x \to \exists f. f \in (x \to \bigcup x) \wedge \forall y. y \in x  \to f(y) \in y$
        
        where we define $(x , y) \coloneq \{\{x\}, \{x, y\}\}$, 
        
        $A \times B \coloneq \{ z \in \pow{\pow{A \cup B}} \mid \exists x \in A \exists y \in B. z = (x, y)\}$, 
        
        $(A \to B) \coloneq \{f \in \pow{A \times B} \mid \forall x \in A \exists! y. (x, y) \in f \}$ and
        
        $f(x) \coloneq 
        \begin{cases}
            y & \text{if } (x, y) \in f \\
            \varnothing & \text{if no such $y$ exists} \\
        \end{cases}$
    \end{enumerate}
\end{defi}
\end{boxdef}

\begin{rem}
    The existence of at least one set is provable and therefore the empty set also exists. 
    Nonetheless, the existence of the empty set is often added as an axiom.
\end{rem}

\begin{rem}
    In principle, you can do almost all math in the combination of FOL and ZFC.
    In practice, you want to do meta-logical operations (e.g. quantifying over formulas). 
    For example: 
    \begin{enumerate}
        \item You want to be able to define new definitions with definite descriptions.
        \item You want to be able to talk about theorem schemes.
    \end{enumerate}
    Mizar and Metamath are two proof systems implementing FOL + ZFC in a metalogic.
\end{rem}

\clearpage
\section{\texorpdfstring{$\lambda$}{lambda}-calculus}

\begin{boxdefi}
    Let $\alert{\fV} = \{ x_0, x_1, \dots \} $ a countably infinite set of variables and $\alert{C} = \{ c_0, c_1, \dots \}$ be any set of constants. 
    The terms of the \alert{$\lambda$-calculus} are given by the following BNF: 
    \begin{equation*}
        s,t \Coloneqq x \mid c \mid (s t) \mid (\lambda x. t)
    \end{equation*}
    where $x$ is a variable and $c$ a constant.
\end{boxdefi}

\begin{rem}
    \hfill
    \begin{enumerate}
        \item Think of $(\lambda x. t)$ as the function $x \mapsto t(x)$ and $(s t)$ as function application. 
        \item Anything can apply to any term, e.g. $(x x)$ is a term.
        \item Abbreviate $((rs)t)u$ to $rstu$ and $\lambda x. \lambda y. \lambda z. t$ to $\lambda x y z. t$. For example, $\lambda xy. xy$ means $(\lambda x. (\lambda y. (x y)))$.
        \item $\lambda x.t$ binds the variable $x$. Like in first-order logic we can define $\alpha$-equivalence ($\aeq$) and substitution ($t[s/x]$). Here we identify $\alpha$-equivalent terms, i.e. $\lambda x.x = \lambda y.y$.
    \end{enumerate}
\end{rem}

\begin{example}
    $(x(\lambda x.x))[s/x] = s(\lambda x.x)$
\end{example}

\begin{rem}
    Consider $(\lambda x. t) s$. 
    We want this to correspond to $t[s/x]$. 
\end{rem}

\begin{boxdefi}
    \hfill
    \begin{enumerate}
        \item \alert{$\beta$-contraction ($\becont$)} is defined as $(\lambda x. t) s \becont t [s/x]$.
        \item {\alert{One-step-$\beta$-reduction ($\beored$)} is defined as the compatible closure if $\becont$, i.e.
            \begin{enumerate}
                \item If $s \becont t$ then $s \beored t$.
                \item If $s \beored t$ then $su \beored tu$, $us \beored ut$ and $\lambda x. s \beored \lambda x. t$.
            \end{enumerate}}
        \item \alert{$\beta$-reduction ($\bered$)} is the reflexive transitive closure of $\beored$, i.e. it is the smallest relation that is reflexive, transitive and contains $\beored$.
        \item \alert{$\beta$-equivalence ($\beq$)} is the smallest equivalence relation containing $\bered$.
    \end{enumerate}
\end{boxdefi}

\begin{example}\label{ex:betared}
    \hfill
    \begin{enumerate}
        \item $(\lambda x. xxy)(yz) \becont yz(yz)y$
        \item \adjustbox{valign=t}{
            \begin{tikzcd}
                (\lambda x. xx)y((\lambda z. yz)(ww))  \ar[r, "{\beta , 1}"] \ar[d, "{\beta , 1}"] & (\lambda x.xx)y(y(ww)) \ar[d, "{\beta , 1}"] \\
                yy ((\lambda z. yz)(ww)) \ar[r, "{\beta , 1}"] & yy (y(ww)) \\
            \end{tikzcd}}
            \vspace{-2em}
        \item $(\lambda x.xx) (\lambda x. xx) \beored (\lambda x.xx) (\lambda x. xx)$
    \end{enumerate}
\end{example}

\begin{boxdefi}
    We define the following combinators: 
    \begin{enumerate}
        \item $\alert{I} = \lambda x.x$
        \item $\alert{K} = \lambda xy.x$
        \item $\alert{K_*} = \lambda xy.y$
        \item $\alert{S} = \lambda xyz.xz(yz)$
    \end{enumerate}
\end{boxdefi}

\begin{rem}
    Every term can be defined using $K$ and $S$ up to $\beta$-equivalence.
\end{rem}

\begin{example}
    $SKK \bered \lambda z. Kz(Kz) \bered \lambda z.z = I$.
\end{example}

\begin{boxprop} \label{prop:fixpoi}
    There exists a \alert{fixed-point combinator} $Y$ such that $Y t \bered t(Yt)$.
\end{boxprop}
\begin{proof}
    Let $A \coloneq \lambda fx. x(ffx)$ and let $Y \coloneq AA$ be the \alert{Turing operator}.
    Then $Yt = AAt \bered t (AAt) = t(Yt)$.
\end{proof}

\begin{boxdefi}
    We can define the \alert{pairing} $\alert{P} \coloneq \lambda stx.xst$ and denote $\alert{(s, t)} \coloneq Pst \bered \lambda x. xst$.
\end{boxdefi}

\begin{rem}
    Naming this a pairing makes sense because we have $(s, t) K \bered Kst \bered s$ and $(s, t) K_* \bered K*(s, t) \bered t$.
\end{rem}

\begin{boxdefi}
    We can define \alert{Church numerals}. 
    If $n$ is a natural number we encode it as $\alert{[n]} \coloneq \lambda fx. f^n x$ where $f^0 x \coloneq x$ and $f^{n+1} x = f(f^nx) = f^n(fx)$.
\end{boxdefi}

\begin{boxdefi}
    Let $A$ be a set and $\ored$ a binary relation on $A$ with reflective transitive closure $\to$.
    \begin{enumerate}
        \item $t \in A$ is \alert{in normal form} if there is no $s$ such that $t \ored s$.
        \item $t \in A$ \alert{has normal form $s$} if $s$ is in normal form and $t \to s$.
        \item $t \in A$ is \alert{strongly normalizing} if there exists no infinite sequence $t \ored t_1 \ored t_2 \ored t_3 \ored \cdots$.
        \item $\ored$ is \alert{(weakly) normalizing} if every $t \in A$ has a normal form.
        \item $\ored$ is \alert{strongly normalizing} if every $t \in A$ is strongly normalizing.
        \item {$\ored$ is \alert{confluent} (has the \alert{Church-Rosser property}) if whenever $u \leftarrow t \to v$ there is an $s \in A$ with $u \to s \leftarrow v$.
            \begin{equation*}
            \begin{tikzcd}[ampersand replacement=\&]
                t \ar[r] \ar[d] \& v \ar[d, dashed] \\
                u \ar[r, dashed] \& s 
            \end{tikzcd}
        \end{equation*}}
    \end{enumerate}
\end{boxdefi}

\begin{rem}
    $\beta$-reduction is neither weakly nor strongly normalizing.
\end{rem}
\begin{proof}
    See the counterexamples presented in Example \ref{ex:betared} (iii) and Proposition \ref{prop:fixpoi}.
\end{proof}

\begin{boxthm}[Church-Rosser] \label{thm:ChurchRosser}
    $\beored$ is confluent.
\end{boxthm}

\begin{rem}
    It does not suffice to prove 
    \begin{equation*}
        \begin{tikzcd}
            t \ar[r, "{\beta, 1}"] \ar[d, "{\beta, 1}"] & v \ar[d, dashed, "\beta"] \\
            u \ar[r, dashed, "\beta"] & s 
        \end{tikzcd}
    \end{equation*}
    i.e. for a general binary relation Theorem \ref{thm:ChurchRosser} does not follow from this.
\end{rem}

\begin{boxdefi} \label{def:parared}
    \alert{Parallel reduction ($\Rightarrow$)} is defined inductively as
    \begin{enumerate}
        \item $x \Rightarrow x$ and $c \Rightarrow c$ where $x$ is a variable and $c$ is a constant.
        \item If $t \Rightarrow t'$ then $\lambda x.t \Rightarrow \lambda x. t'$.
    \end{enumerate}
    If $t \Rightarrow t'$ and $u \Rightarrow u'$ then 
    \begin{enumerate}[resume]
        \item $tu \Rightarrow t'u'$.
        \item $(\lambda x.t)u \Rightarrow t'[u'/x]$.
    \end{enumerate}
\end{boxdefi}

\begin{boxlem}
    Parallel induction is reflexive, i.e. $t \Rightarrow t$.
\end{boxlem}
\begin{proof}
    We show this by induction on $t$. 
    Consider $t = \lambda x.s$. 
    Then by induction hypothesis $s \Rightarrow s$, so by Definition \ref{def:parared} (ii) $\lambda x.s \Rightarrow \lambda x.s$.
    The rest of the cases are similar.
\end{proof}

\begin{boxlem} \label{lem:beoredtoarrow}
    If $t \beored s$ then $t \Rightarrow s$.
\end{boxlem}
\begin{proof}
    We show this by induction on $t \beored s$.
    If $t \becont s$, say $(\lambda x.u)v \becont u [v/x]$. 
    Then $(\lambda x. u)v \Rightarrow u [v/x]$ by Definition \ref{def:parared} (iv).
    The other cases are also easy to show.
\end{proof}

\begin{boxlem} \label{lem:arrowbered}
    If $t \Rightarrow t'$ then $t \bered t'$.
\end{boxlem}
\begin{proof}
    We show this by induction on $t \Rightarrow t'$.
    Suppose the last rule was Definition \ref{def:parared} (iv), i.e. concluding $(\lambda x.t)u \Rightarrow t'[u'/x]$ from $t \Rightarrow t'$ and $u \Rightarrow u'$.
    By induction hypothesis we have $t \bered t'$ and $u \bered u'$.
    Then $(\lambda x.t)u \bered (\lambda x.t') u \bered (\lambda x. t') u' \beored t'[u'/x]$.
    The other cases are similar.
\end{proof}

\begin{boxlem} \label{lem:arrowsubst}
    If $t \Rightarrow t'$ and $w \Rightarrow w'$ then $t[w/y] \Rightarrow t'[w'/y]$.
\end{boxlem}
\begin{proof}
    We show this by induction on $t \Rightarrow t'$.
    Suppose the last step was Definition \ref{def:parared} (iv), i.e. concluding $(\lambda x.t)u \Rightarrow t'[u'/x]$ from $t \Rightarrow t'$ and $u \Rightarrow u'$.
    By induction hypothesis we know that $t[w/y] \Rightarrow t'[w/y]$ and $u[w /y] \Rightarrow u'[w/y]$.
    We have to show $(\lambda x.t [w/y])u[w/y] \Rightarrow t'[u'/x][w'/y]$.
    $x$ is bound so we may assume that $x$ is not free in $w'$. 
    Then one can prove that $t'[u'/x][w'/y] = t'[w'/y][(u'[w'/y])/x]$.
    Now the claim follows from Definition \ref{def:parared} (iv).
    The rest of the cases are similar.
\end{proof}

\begin{boxdefi} \label{def:star}
    If $t$ is a term then \alert{$t^*$} is recursively defined as 
    \begin{enumerate}
        \item $x^* = x$ and $c^* = c$ for variables $x$ and constants $c$.
        \item $(\lambda x.t)^* = \lambda x. t^*$.
        \item $(ts)^* = t^*s^*$ if $t$ is not a $\lambda$.
        \item $((\lambda x.t) s)^* = t^*[s^*/x]$.
    \end{enumerate}
\end{boxdefi}

\begin{boxlem} \label{lem:arrowstar}
    If $s \Rightarrow t$ then $t \Rightarrow s^*$.
\end{boxlem}
\begin{proof}
    We show this by induction of the length of the derivation for $s \Rightarrow t$.

    If the last step was Definition \ref{def:parared} (iii), i.e. concluding $tu \Rightarrow t'u'$ from $t \Rightarrow t'$ and $u \Rightarrow u'$, then by induction hypothesis $t' \Rightarrow t^*$ and $u' \Rightarrow u^*$.
    We need to show that $t'u' \Rightarrow (t u)^*$.
    If $t$ is not a $\lambda$ then $(tu)^* = t^*u^*$, so we are done by Definition \ref{def:parared} (iii).
    If $t = \lambda x. s$ then $(tu)^* = s^* [u*/x]$.
    We know $\lambda x.s \Rightarrow t'$ which can only be derived using Definition \ref{def:parared} (ii). 
    So $t' = \lambda x. s'$ with $s \Rightarrow s'$.
    By induction hypothesis $s' \Rightarrow s^*$. 
    Then $(\lambda x. s') u' \Rightarrow s^* [u*/x]$ follows from Definition \ref{def:star} (iv).

    If the last step was Definition \ref{def:parared} (iv), i.e. concluding $(\lambda x.t) u \Rightarrow t'[u'/x]$ from $t \Rightarrow t'$ and $u \Rightarrow u'$, then by induction hypothesis we know that $t' \Rightarrow t^*$ and $u' \Rightarrow u^*$.
    Then $t'[u'/x] \Rightarrow ((\lambda x.t)u)^* = t^*[u^* /x]$ by Lemma \ref{lem:arrowsubst}.

    The rest of the cases are easy.
\end{proof}

\begin{boxlem}\label{lem:almostconfluence}
    If $t \beored u$ and $t \bered v$ then there is an $s$ with $u \bered s \prescript{}{\beta}{\leftarrow} v$.
    \begin{equation*}
        \begin{tikzcd}[ampersand replacement=\&]
            t \ar[r, "\beta"] \ar[d, "{\beta, 1}"] \& v \ar[d, dashed, "\beta"] \\
            u \ar[r, dashed, "\beta"] \& s 
        \end{tikzcd}
    \end{equation*}
\end{boxlem}
\begin{proof}
    Decomposing $t \bered v$ into individual steps gives us
    \begin{equation*}
        \begin{tikzcd}[/tikz/cells={/tikz/nodes={shape=asymmetrical
            rectangle,text width=0.5cm,text height=2ex,text depth=0.3ex,align=center}}]
            & t \ar[dl, "{\beta, 1}"] \ar[dr, "{\beta, 1}"] & \\
            u && v_1 \ar[dr, "{\beta, 1}"] \\
            &&& v_2 \ar[dr, "{\beta, 1}"] \\
            &&&& \ddots \ar[dr, "{\beta, 1}"] \\
            &&&&& \mathllap{v_n} = \mathrlap{v}
        \end{tikzcd}
    \end{equation*}
    which with Lemma \ref{lem:beoredtoarrow} becomes
    \begin{equation*}
        \begin{tikzcd}[/tikz/cells={/tikz/nodes={shape=asymmetrical
            rectangle,text width=0.5cm,text height=2ex,text depth=0.3ex,align=center}}]
            & t \ar[dl, Rightarrow] \ar[dr, Rightarrow] & \\
            u && v_1 \ar[dr, Rightarrow] \\
            &&& v_2 \ar[dr, Rightarrow] \\
            &&&& \ddots \ar[dr, Rightarrow] \\
            &&&&& \mathllap{v_n} = \mathrlap{v}
        \end{tikzcd}
    \end{equation*}
    which with Lemma \ref{lem:arrowstar} yields
    \begin{equation*}
        \begin{tikzcd}[/tikz/cells={/tikz/nodes={shape=asymmetrical
            rectangle,text width=0.5cm,text height=2ex,text depth=0.3ex,align=center}}]
            & t \ar[dl, Rightarrow] \ar[dr, Rightarrow] & \\
            u \ar[dr, Rightarrow] && v_1 \ar[dr, Rightarrow] \ar[dl, Rightarrow] \\
            & t^* \ar[dr, Rightarrow] && v_2 \ar[dr, Rightarrow] \ar[dl, Rightarrow] \\
            && v_1^* \ar[dr, Rightarrow] && \ddots \ar[dr, Rightarrow] \\
            &&& \ddots \ar[dr, Rightarrow] && \mathllap{v_n} = \mathrlap{v} \ar[dl, Rightarrow]\\
            &&&&\mathllap{v}_{n\mathrlap{-1}}^*
        \end{tikzcd}
    \end{equation*}
    which implies our desired statement since by Lemma \ref{lem:arrowbered} parallel induction implies $\beta$-induction and $\beta$-induction is transitive. 
\end{proof}

\begin{exercise}
    Derive Theorem \ref{thm:ChurchRosser} (Church-Rosser) from Lemma \ref{lem:almostconfluence}.
\end{exercise}

\clearpage
\section{Simple Type Theory}

Simple type theory was first presented by Church in 1940. 

\begin{boxdefi}
    We add to $\lamchu$ the type constants 
    \begin{enumerate}
        \item \alert{$I$} (``\alert{individuals}'')
        \item \alert{$\proptype$}
    \end{enumerate}
    and the term constants 
    \begin{enumerate}
        \item $\alert{\forallterm{\tau}} : (\tau \to \proptype) \to \proptype$
        \item $\alert{{\Rightarrow}} : \proptype \to \proptype \to \proptype$
        \item $\alert{=_\tau} : \tau \to \tau \to \proptype$
        \item $\alert{\varepsilon_\tau} : (\tau \to \proptype) \to \tau$
    \end{enumerate}
    We view $A : \proptype$ as a formula. 
    We write: 
    \begin{enumerate}
        \item $\forall x : \tau. A$ for $\forallterm{\tau}(\lambda x : \tau.A)$
        \item $A \Rightarrow B : \proptype$ for ${\Rightarrow} (A)(B)$ where $A, B : \proptype$
        \item $s =_\tau t : \proptype$ for ${=_\tau}(s)(t)$ where $s, t : \tau$
    \end{enumerate}
    Now we can define: 
    \begin{enumerate}
        \item $\alert{\bot} \coloneq \forall P : \proptype. P$
        \item $\alert{\neg A} \coloneq A \Rightarrow \bot$
        \item $\alert{A \vee B} \coloneq \neg A \Rightarrow B$
        \item $\alert{A \wedge B} \coloneq \neg (\neg A \vee \neg B)$
        \item $\alert{A \Leftrightarrow B} \coloneq (A \Rightarrow B) \wedge (B \Rightarrow A)$
        \item $\alert{\exists x : \tau. A} \coloneq \neg (\forall x : \tau. \neg A)$
    \end{enumerate}
\end{boxdefi}

\begin{rem}
    See sheet 6 exercise 6 for a different (equivalent) way to define $A \wedge B$, $A \vee B$ and $\exists x : \tau. A$.
\end{rem}

\begin{boxdefi}
    In addition to the typing judgements $\Gamma \vdash t : \tau$ we can now define \alert{provability judgements} $\alert{\Delta \mathrel{\vdash_\Gamma} A}$ (or $\Delta \vdash A$ for short) where $\Delta$ is a set of terms $B$ such that $\Gamma \vdash B : \proptype$ and $\Gamma \vdash A : \proptype$.
    This gives a proof system with the following rules where the typing constraints are marked in gray: 
    \begin{enumerate}
        \item {(Ass)
            \AxiomC{}
            \UnaryInfC{$\Delta, A \vdash A$}
            \DisplayProof} 
        \item {($\Rightarrow$ I)
            \AxiomC{$\Delta, A \vdash B$}
            \UnaryInfC{$\Delta \vdash A \Rightarrow B$}
            \DisplayProof}
        \item {($\Rightarrow$ E)
            \AxiomC{$\Delta \vdash A \Rightarrow B$}
            \AxiomC{$\Delta \vdash A$}
            \BinaryInfC{$\Delta \vdash B$}
            \DisplayProof}
        \item {($\forall$ I) 
            \AxiomC{$\Delta \vdash A$}
            \AxiomC{$x \notin \fv{\Delta}$}
            \AxiomC{$\color{gray}\Gamma \vdash x : \tau$}
            \TrinaryInfC{$\Delta \vdash \forall x : \tau. A$}
            \DisplayProof}
        \item {($\forall$ E)
            \AxiomC{$\Delta \vdash \forall x : \tau. A$}
            \AxiomC{$\color{gray}\Gamma \vdash t : \tau$}
            \BinaryInfC{$\Delta \vdash A [t/x]$}
            \DisplayProof}
        \item {($=$ Refl)
            \AxiomC{$\color{gray} \Gamma \vdash t : \tau$}
            \UnaryInfC{$\Delta \vdash t \mathrel{=_\tau} t$}
            \DisplayProof}
        \item {($=$ Symm)
            \AxiomC{$\Delta \vdash t \mathrel{=_\tau} s$}
            \UnaryInfC{$\Delta \vdash s \mathrel{=_\tau} t$}
            \DisplayProof}
        \item {($=$ Trans)
            \AxiomC{$\Delta \vdash t \mathrel{=_\tau} s$}
            \AxiomC{$\Delta \vdash s \mathrel{=_\tau} r$}
            \BinaryInfC{$\Delta \vdash t \mathrel{=_\tau} r$}
            \DisplayProof}
        \item {(${\to}$ Congr)
            \AxiomC{$\Delta \vdash t \mathrel{=_{\sigma \to \tau}} t'$}
            \AxiomC{$\Delta \vdash s \mathrel{=_{\sigma}} s'$}
            \BinaryInfC{$\Delta \vdash t s \mathrel{=_\tau} t' s'$}
            \DisplayProof}
        \item {($\beta$ Equiv)
            \AxiomC{$\color{gray} \Gamma, x : \sigma \vdash t : \tau$}
            \AxiomC{$\color{gray} \Gamma \vdash s : \sigma$}
            \BinaryInfC{$\Delta \vdash (\lambda x : \sigma. t) s \mathrel{=_\tau} t [s/x]$}
            \DisplayProof}
        \item To be continued\dots
    \end{enumerate}
\end{boxdefi}

\clearpage
\section{Dependant Type Theory}


The rules of simple type theory (when ignoring the terms) are precisely the rules of implicational logic. 
So, instead of having a type specifically for propositions, we could view types as propositions and terms as proofs. 
This is called \alert{Curry-Howard correspondence} or the \alert{propositions-as-types interpretation}.
We could add other type formers for connectives, e.g.\ $\sigma \times \tau$ corresponding to conjunction and $\sigma + \tau$ corresponding to disjunction.

How do we represent quantifiers?
We will add the the \alert{dependant product type}, or \alert{pi type}, $\prod x : \sigma. \tau$ or $\prod_{x : \sigma} \tau$, corresponding to $\forall x : \sigma. \tau$ and where $\tau$ can depend on $x$.
A term $t : \prod x : \sigma. \tau$ is a dependant function. 
For $s : \sigma$, we get $ts : \tau[s/x]$.

\begin{example}
    We could then define $\tau \coloneq \bR^n$ where $n : \bN$. 
    Then $f$, defined to send $n : \bN$ to $\underbrace{(0, \dots, 0)}_{\text{length }n}$ would have type $\prod_{n : \bN} \bR^n$.
\end{example}

\subsection{Pure Type Systems (PTSs)}

\begin{boxdefi}
    A \alert{pure type system (PTS)} is determined by $(\fS, \fA, \fR)$, where 
    \begin{enumerate}
        \item $\fS$ is a set of \alert{sorts} (or \alert{universes})
        \item $\fA \subseteq \fS \times \fS$ is a set of \alert{axioms}
        \item $\fR \subseteq \fS \times \fS \times \fS$ is a set of \alert{relations}
    \end{enumerate}
    We additionally have an infinite set of variables. 
    A PTS has \alert{preterms} 
    \begin{equation*}
        A, B, M, N \Coloneqq x  \mid s \mid (MN) \mid (\lambda x : A. M) \mid \prod x : A, B.
    \end{equation*}
    If $B$ does not depend on $x$, we write $\prod x : A. B$ as $\alert{A \to B}$.
    \alert{Contexts} are lists if the form $\Gamma \coloneq x_1 : A_1, x_2 : A_2, \dots, x_n : A_n$.
    Here, the order of the context matters. 
    We set $\alert{\domain{\Gamma}} \coloneq \{x_1, \dots, x_n\}$. 
    We identify terms up to $\alpha$-equivalence, so e.g.\ $\lambda x : \tau. x \aeq \lambda y : \tau. y$ and $\prod x : \tau. B(x) \aeq \prod y : \tau. B(y)$.
    To be continued\dots 
\end{boxdefi}

\end{document}