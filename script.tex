\documentclass{article}


\setcounter{secnumdepth}{3}     % numbering depth (chapter, section, subsection, ...).
\setcounter{tocdepth}{3}        % numbering depth in table of contents

\usepackage[left=4cm, right=4cm, top=4cm, bottom=4cm]{geometry}
\usepackage{amsmath,amssymb,amsthm,amsfonts,amsbsy,latexsym, mathtools} 
\usepackage[backend=biber, style=alphabetic]{biblatex}  % Extension for bibtex
\bibliography{thesis}                   % The bibliography file that is used
\usepackage[nodayofweek]{datetime}      % To get \today
\usepackage{enumitem}                   % Change \item symbol in enumerate
\setlist[enumerate,1]{label=(\roman*)}  % Get roman numerals as items
\usepackage[T1]{fontenc}                % Better fonts
\usepackage{ifthen}                     % if then
\usepackage{lmodern}                    % Bessere Schrift
\usepackage{listings}                   % Code Listings.
\usepackage{stackrel}                   % write thing above relations
\usepackage[dvipsnames]{xcolor}         % color text
\usepackage{tikz}                       % commutative diagrams
\usepackage{tikz-cd}                    % commutative diagrams
\usepackage{bussproofs}                 % To write prooftrees
\usepackage{stmaryrd}                   % To get double brackets
\usepackage[many]{tcolorbox}    	    % for COLORED BOXES (tikz and xcolor included)
\usepackage{mathspec} 			        % for FONTS
\usepackage{setspace}                   % for LINE SPACING
\usepackage{multicol}                   % for MULTICOLUMNS
\usepackage{environ}                    % better custom enviroments
\usepackage{adjustbox}                  % to get custom alignment of diagrams
\usepackage{xparse}                     % to define the claim environment

% to get upright greek letters
\usepackage{upgreek}

% To break urls correctly in bibliography
\setcounter{biburllcpenalty}{7000}
\setcounter{biburlucpenalty}{8000}

% to get rid of an error
\usepackage{bookmark}

%to get code in footnotes write \cprotect\footnote 
\usepackage{cprotect}

% to get the external link symbol
\usepackage{fontawesome}

% Last package!!!!
\usepackage{hyperref}                   
\hypersetup{                            % options
    colorlinks=true,                    % colours
    linkcolor=black,                     % colour for links
    citecolor=black,                     % colour for refrences
    urlcolor=black                       % colour for links
}

% Boxes 

\setlength\parindent{0pt}   % killing indentation for all the text
\setstretch{1.3}            % setting line spacing to 1.3
\setlength\columnsep{0.25in} % setting length of column separator
\pagestyle{empty}           % setting pagestyle to be empty

\tcbset{
    sharp corners,
    colback = white,
    before skip = 0.2cm,    % add extra space before the box
    after skip = 0.5cm      % add extra space after the box
}                           % setting global options for tcolorbox

\definecolor{maindef}{HTML}{5989cf}    % setting main color to be used
\definecolor{subdef}{HTML}{e5f1ff}     % setting sub color to be used

\newtcolorbox{boxdef}{
    colback = subdef, 
    colframe = maindef, 
    boxrule = 0pt, 
    leftrule = 6pt, % left rule weight
    breakable,
}


\definecolor{mainthe}{HTML}{73937E}    % setting main color to be used
\definecolor{subthe}{HTML}{eef7f1}     % setting sub color to be used

%\definecolor{mainthe}{HTML}{D387AB}    % setting main color to be used
%\definecolor{subthe}{HTML}{fff1fa}     % setting sub color to be used

\newtcolorbox{boxthe}{
    colback = subthe, 
    colframe = mainthe, 
    boxrule = 0pt, 
    leftrule = 6pt, % left rule weight
    breakable,
}


% Environments
\theoremstyle{definition}              
\newtheorem{defi}{Definition}[section] 
\newtheorem{example}[defi]{Example}
\newtheorem{notation}[defi]{Notation}
\newtheorem{rem}[defi]{Remark}

\theoremstyle{plain}
\newtheorem{conj}[defi]{Conjecture}
\newtheorem{cor}[defi]{Corollary}
\newtheorem{lem}[defi]{Lemma}
\newtheorem{prop}[defi]{Proposition}
\newtheorem{thm}[defi]{Theorem}
\newtheorem{exercise}[defi]{Exercise}

% things with boxes build in
\NewEnviron{boxthm}[1][]{
    \begin{boxthe}
        \begin{thm}[#1]
            \BODY
        \end{thm}
    \end{boxthe}
}

\NewEnviron{boxdefi}{
    \begin{boxdef}
        \begin{defi}
            \BODY
        \end{defi}
    \end{boxdef}
}

\NewEnviron{boxprop}[1][]{
    \begin{boxthe}
        \begin{prop}[#1]
            \BODY
        \end{prop}
    \end{boxthe}
}

\NewEnviron{boxlem}[1][]{
    \begin{boxthe}
        \begin{lem}[#1]
            \BODY
        \end{lem}
    \end{boxthe}
}

\NewEnviron{boxcor}[1][]{
    \begin{boxthe}
        \begin{cor}[#1]
            \BODY
        \end{cor}
    \end{boxthe}
}

% Claim environment (nestable once)
\begin{comment}
    The syntax is: 
    \begin{claim}
        Claim statement
        \begin{proof}
            Proof of claim.
        \end{proof}
    \end{claim}
\end{comment}

\newlist{Claim}{description}{2}
\setlist[Claim]{labelindent=2em,leftmargin=*}
\newif\ifInsideClaim
\newcounter{claim}[defi]
\newcounter{cclaim}[claim]
\renewcommand\theclaim{\arabic{claim}}
\renewcommand\thecclaim{\arabic{claim}.\arabic{cclaim}}
\let\originalqedsymbol\qedsymbol
\newenvironment{claim}{%
  % disable qed symbol if there is no star
  \let\qedsymbol\relax%
  \ifInsideClaim% we have a nested environment
    \refstepcounter{cclaim}%
    \let\theclaimcounter\thecclaim%
  \else%
    \refstepcounter{claim}%
    \let\theclaimcounter\theclaim%
    \InsideClaimtrue%
  \fi%
  \Claim\item[\textbf{Claim \theclaimcounter:}]%
}{\endClaim\InsideClaimfalse\let\qedsymbol\originalqedsymbol}

%macros
\newcommand{\bA}{\mathbb{A}}
\newcommand{\bB}{\mathbb{B}}
\newcommand{\bC}{\mathbb{C}}
\newcommand{\bD}{\mathbb{D}}
\newcommand{\bE}{\mathbb{E}}
\newcommand{\bF}{\mathbb{F}}
\newcommand{\bG}{\mathbb{G}}
\newcommand{\bH}{\mathbb{H}}
\newcommand{\bI}{\mathbb{I}}
\newcommand{\bJ}{\mathbb{J}}
\newcommand{\bK}{\mathbb{K}}
\newcommand{\bL}{\mathbb{L}}
\newcommand{\bM}{\mathbb{M}}
\newcommand{\bN}{\mathbb{N}}
\newcommand{\bO}{\mathbb{O}}
\newcommand{\bP}{\mathbb{P}}
\newcommand{\bQ}{\mathbb{Q}}
\newcommand{\bR}{\mathbb{R}}
\newcommand{\bS}{\mathbb{S}}
\newcommand{\bT}{\mathbb{T}}
\newcommand{\bU}{\mathbb{U}}
\newcommand{\bV}{\mathbb{V}}
\newcommand{\bW}{\mathbb{W}}
\newcommand{\bX}{\mathbb{X}}
\newcommand{\bY}{\mathbb{Y}}
\newcommand{\bZ}{\mathbb{Z}}

\newcommand{\fC}{\mathcal{C}}
\newcommand{\fL}{\mathcal{L}}
\newcommand{\fF}{\mathcal{F}}
\newcommand{\fR}{\mathcal{R}}
\newcommand{\fV}{\mathcal{V}}
\newcommand{\fM}{\mathcal{M}}
\newcommand{\fP}{\mathcal{P}}


\DeclareMathOperator{\aeq}{\equiv_{\alpha}} %alpha-equivalence
\newcommand{\alert}[1]{{\color{blue}{\ifmmode\mathbf{#1}\else\textbf{#1}\fi}}}
\DeclareMathOperator{\cupdot}{\dot{\cup}} % Disjoint union
\DeclareMathOperator{\smodels}{\models_{\sigma}} %models with variable assignment

\newcommand{\pow}[1]{\fP(#1)}
\newcommand{\fv}[1]{\text{fv}(#1)} %free variables
\newcommand{\abs}[1]{\lvert #1 \rvert} % abs value (underlying set of a model)
\newcommand{\interpret}[1]{\llbracket #1 \rrbracket} % interpretation in a model
\newcommand{\vect}[1]{\bar{#1}}
\newcommand{\menquote}[1]{\ensuremath{\text{``} #1 \text{''}}} % quotes in math mode

\newcommand{\pr}[2]{\text{pr}_{#1}(#2)} %projection
\newcommand{\compl}[1]{#1^{c}} %complement
\newcommand{\id}{\text{id}} %identity
\newcommand{\preim}[2]{#1^{-1}(#2)} %preimage
\newcommand{\interior}[1]{\text{int}(#1)} %interior
\newcommand{\closure}[1]{\overline{#1}} %closure
\newcommand{\boundary}{\partial} %boundary
\newcommand{\restrict}[2]{\ensuremath{\left.#1\right|_{#2}}} %restriction
\newcommand{\seqclosure}[1]{\text{scl}(#1)} %sequential closure
\newcommand{\oball}[2]{\text{B}_{#1}(#2)} %open ball
\newcommand{\closedCell}[2]{\closure{e}_{#2}^{#1}} %closed n-cell
\newcommand{\openCell}[2]{e_{#2}^{#1}} %open n-cell
\newcommand{\cellFrontier}[2]{\partial e_{#2}^{#1}} %edge n-cell
\newcommand{\closedCellf}[2]{\closure{f}_{#2}^{#1}} %closed n-cell
\newcommand{\openCellf}[2]{f_{#2}^{#1}} %open n-cell
\newcommand{\cellFrontierf}[2]{\partial f_{#2}^{#1}} %edge n-cell
\newcommand{\norm}[1]{\left\lVert#1\right\rVert} %norm
\newcommand{\maximum}[2]{\text{max}(#1, #2)} %maximum



\title{Logic of Proof Assistants}
\author{Prof. Floris van Doorn \thanks{\LaTeX-realization by Hannah Scholz} \\ University of Bonn}

\begin{document}

\maketitle
\tableofcontents        % table of contents

%introduction
\clearpage
\section{Introduction}

The topics of this class are: 
\begin{enumerate}
    \item First-order Logic/Set Theory
    \item Lambda Calculus
    \item Simple Type Theory (Higher-Order Logic)
    \item Dependent Type Theory/Homotopy Type Theory
\end{enumerate}

\begin{example}
    Here are examples of proof assistants for these different types of logics: 
    \begin{enumerate}
        \item First-order Logic/Set Theory: Mizar, Metamath
        \item Simple Type Theory: Isabelle/HoL, HoL Light
        \item Dependent Type Theory: Lean, Rocq (formerly Coq), Agda
        \item Homotopy Type Theory: cubicaltt, rzk
    \end{enumerate}
\end{example}

\begin{rem}
    You might want to have the following criteria for a logic: 
    \begin{enumerate}
        \item Appropriate (You can encode mathematical arguments.)
        \item Simple (It is relatively easy to understand.)
        \item Expressive (Mathematical arguments are convenient to express.)
    \end{enumerate}
\end{rem}

\begin{thm}
Let $\pi$ be the prime counting function, i.e. $\pi \colon \bR \to \bN$, $x \mapsto \lvert \{ p \le x \mid p \text{\textup{ prime}} \} \rvert$.
    Then $\lim_{x \to \infty} \frac{\pi (x)}{x / \log(x)} = 1$.
\end{thm}

\begin{rem}
When formalizing/stating this theorem in a formal logic there are a few things that you need to think about: 
\begin{enumerate}
    \item What do you do about division by zero?
    \item What does division even mean? (Do you define division for $\bR$ explicitly? 
        Do you define it generally for a field? 
        Or even for a group? 
        How do you ensure that the ``correct'' field structure on $\bR$ gets used?)
    \item How do you define a limit? (Do you define a limit for $\bR$ explicitly? 
        Or for every topological space?
        How do you ensure the ``correct'' topology on $\bR$ gets used? 
        How do you deal with potentially non-unique limits (for example in non-Hausdorff spaces)?)
\end{enumerate}
\end{rem}

\begin{rem}
You can make the following design choices for ``a logic'': 
\begin{enumerate}
    \item Is the logic typed or untyped?
    \item Is the logic constructive or classical?
    \item Does the logic support computation?
\end{enumerate}
\end{rem}

\begin{rem}
In logic there is the \alert{object language} and we reason about it in a \alert{meta-language} (``ordinary mathematical reasoning'').
\end{rem}

\subsection{Inductive Definitions}

\begin{example}
    The natural numbers are inductively defined by $0 \in \bN$ and $S \colon \bN \to \bN$, $n \mapsto n + 1$. 
\end{example}

\begin{boxdef}
\begin{defi}
    Let $U$ be a set and $\fC \subseteq \bigcup_{n \in \bN} (U^n \to U)$ a set of \alert{constructors},
    where $c \colon U^n \to U$ is called an \alert{$n$-ary function} and $(U^n \to U)$ is the collection of $n$-ary functions.
    \begin{enumerate}
        \item $A \subseteq U$ is \alert{closed under $\fC$} if for any $n$-ary $c \in \fC$ and for all $x_1, \dots , x_n \in A$ we have that $c(x_1, \dots, x_n) \in A$.
        \item $A \subseteq U$ is  \alert{generated by $\fC$} or \alert{inductively defined by $\fC$} if $A$ is the smallest set that is closed under $\fC$, i.e. $A = \bigcap \{B \subseteq U \mid B \text{ is closed under } \fC\}$.
        \item $A \subseteq U$ is \alert{freely generated by $\fC$} if 
            \begin{enumerate}
                \item each constructor is injective on $A^n$ and
                \item the images of different constructors are disjoint.
            \end{enumerate}
    \end{enumerate}
\end{defi}
\end{boxdef}

\begin{rem}
$\varnothing$ is closed under $\fC$ iff $\fC$ has no nullary constructors.
\end{rem}

\begin{exercise}
    $\bigcap \{B \subseteq U \mid B \text{\textup{ is closed under }} \fC\}$ is closed under $\fC$.
\end{exercise}

\begin{example}
    \hfill
    \begin{enumerate}
        \item The free group.
        \item The $\sigma$-algebra generated by a collection of subsets. (This is not freely generated.)
        \item The topology generated by a collection of subsets. (This is not freely generated.)
    \end{enumerate}
\end{example}

\begin{boxthe}
\begin{thm}[Structural Induction]
    If $A \subseteq U$ is generated by $\fC$ and $P \colon A \to \{ \top, \bot\}$ is a predicate on $A$, to prove $\forall a \in A, P(a)$ it suffices to show: for any $n$-ary $c \in \fC$ and any $x_1, \dots, x_n \in A$ if $P(x_1), \dots, P(x_n)$ then $P(c(x_1, \dots, x_n))$.
\end{thm}
\end{boxthe}
\begin{proof}
    Exercise.
\end{proof}

\begin{rem}
    The base case of the induction is given by nullary constructors.
\end{rem}

\begin{boxthe}
\begin{thm}[Structural Recursion]
    If $A \subseteq U$ is freely generated by $\fC$, $B$ is a set and for any $n$-ary $c \in \fC$ we have a $g_c \colon B^n \to B$ then there is a unique function $f \colon A \to B$ such that $f(c(a_1, \dots, a_n)) = g_c(f(a_1), \dots, f(a_n))$ for every $c \in \fC$ and $a_1, \dots a_n \in A$. 
\end{thm}
\end{boxthe}
\begin{proof}
Exercise.
\end{proof}

\begin{example}
    For $A = \bN$ this reduces to $f(0) \coloneq g_0$ and $f(S(n)) \coloneq g_s(f(n))$.
\end{example}

\clearpage
\section{First-Order Logic}

\begin{boxdef}
\begin{defi}
A (first-order) \alert{language $\fL$} is a triple $(\fF, \fR, a)$ where $\fF$ is a set of function symbols, $\fR$ is a set of relation symbols, $\fF$ and $\fR$ are disjoint and $a \colon \fF \cup \fR \to \bN$ is the arity function.
\end{defi}
\end{boxdef}

\begin{example}
A language for groups $\fL_{\text{Group}}$ has $\fF \coloneq \{\cdot, ^{-1}, 1 \}$, $\fR \coloneq \varnothing$, $a(\cdot) = 2$, $a(^{-1}) = 1$ and $a(1) = 0$.
\end{example}

\begin{boxdef}
\begin{defi}
We fix an infinite set of \alert{variables} $\alert{\fV} \coloneq \{ x_0, x_1, \dots \}$.
\end{defi}
\end{boxdef}

\begin{rem}
We use $x$ for variables, $f$ and $g$ for functions and $R$ and $S$ for relations.
\end{rem}

\begin{boxdef}
\begin{defi}
We can define the \alert{terms $T_{\fL}$} in the language $\fL$ using the \alert{Backus–Naur form (BNF)}: 
\begin{equation*}
    s, t \Coloneqq x \mid f(t_1, \dots, t_n)
\end{equation*}
where $f$ is an $n$-ary function symbol.
\end{defi}
\end{boxdef}

\begin{boxdef}
\begin{defi}
Formally, we define the \alert{terms $T_{\fL}$} in the language $\fL$ in the following way. We define the set of \alert{symbols} $S \coloneq \fF \cupdot \fV \cupdot \{\menquote{(}, \menquote{)}, \menquote{,} \}$ and the set of finite sequences of symbols $S^*$. 
Let $\fC$ be defined as: 
\begin{enumerate}
    \item for each variable $x \in \fV$ there is a nullary constructor $c_x \coloneq x$
    \item for each $n$-ary function symbol $f$ there is an $n$-ary constructor $c_f \colon (S^*)^n \to S^*$, $c_f(t_1, \dots, t_n) \coloneq f\menquote{(}t_1\menquote{,} \dots \menquote{,} t_n \menquote{)}$
\end{enumerate}
Then $T_{\fL} \subseteq S^*$ is the set generated by $\fC$.
\end{defi}
\end{boxdef}

\begin{example}
    \hfill
    \begin{enumerate}
        \item $\menquote{(}\menquote{)}\menquote{,}f$ is in $S^*$ but not in $T_{\fL}$. 
        \item If $f$ is binary then $f\menquote{(}x_0 \menquote{,} x_1 \menquote{)}$ is in $T_{\fL}$.
    \end{enumerate}
\end{example}

\begin{rem}
Technically, the brackets and commas are not necessary. 
They are however necessary when you use infix notation. 
(For example the meaning of $a \cdot b + c$ is unclear.)
\end{rem}

\begin{boxdef}
\begin{defi}
First-order \alert{formulas $\Phi_{\fL}$} are specified by
\begin{equation*}
    \varphi, \psi \Coloneqq \bot \mid s = t \mid R(t_1, \dots ,t_n) \mid (\varphi \wedge \psi) \mid (\varphi \lor \psi) \mid (\varphi \to \psi) \mid (\forall x. \varphi) \mid (\exists x. \varphi)
\end{equation*}
where $\fR$ is an $n$-ary relation symbol and $t_1, \dots, t_n \in T_{\fL}$.
\end{defi}
\end{boxdef}

\begin{rem}
In classical logic one could omit the rules $(\varphi \wedge \psi)$ and $(\varphi \lor \psi)$ (as they can be defined using the other rules). 
They are however necessary for constructive logic.
\end{rem}

\begin{rem}
We can define other connectives: 
\begin{enumerate}
    \item $\neg \varphi \coloneq (\varphi \to \bot)$
    \item $\varphi \leftrightarrow \psi \coloneq ((\varphi \to \psi) \wedge (\psi \to \varphi))$
\end{enumerate}
\end{rem}

\begin{rem}
When writing formulas we omit some parentheses: 
\begin{enumerate}
    \item $\varphi \to \psi \to \theta$ means $\varphi \to (\psi \to \theta)$
    \item $\forall x. \varphi \to \psi$ means $\forall x. (\varphi \to \psi)$
\end{enumerate}
\end{rem}

\begin{rem}
We want $\forall x. x = x$ and $\forall y. y = y$ to mean the same thing. 
Options to achieve this are: 
\begin{enumerate}
    \item Define $(\forall x. x = x) \aeq (\forall y. y = y)$ to be \alert{$\alpha$-equivalent}. And then define the set of formulas to be $\Phi_{\fL} / \aeq$.
    \item We could not use variable names for bound variables and use \alert{de Bruijn indices} instead.
\end{enumerate}
\end{rem}

\begin{rem}
    $\forall x. x = y$ has \alert{bound variables} $\{x\}$ and \alert{free variables} $\{y\}$. For a formula $\varphi$ or a term $t$ we also write \alert{$\fv{\varphi}$} and \alert{$\fv{t}$} for the set of free variables in $\varphi$ and $t$.
\end{rem}

\begin{boxdef}
\begin{defi}
    A \alert{sentence} is a formula without free variables.
\end{defi}
\end{boxdef}

\begin{boxdef}
\begin{defi}
    \alert{Substitution $s[t/x]$} of $x$ by $t$ in a term $s$ is defined recursively by 
    \begin{enumerate}
        \item {$ y[t/x] \coloneq 
                \begin{cases}
                    t & \text{if } y = x \\
                    y & \text{otherwise}
                \end{cases}$}
        \item $f(s_1, \dots, s_n)[t/x] \coloneq f(s_1[t/x], \dots, s_n[t/x])$
    \end{enumerate}
\end{defi}
\end{boxdef}

\begin{example}
    Defining substitution in formulas is a little bit harder as we need to avoid \alert{variable capture}:
    $(\exists x. x \leq z)[(x + 1)/z]$ should not be $\exists x. x \leq x + 1$ but $\exists y. y \leq x + 1$.
\end{example}

\begin{boxdef}
\begin{defi}
    For a formula $\varphi$ \alert{substitution $\varphi[t/x]$} is defined as: 
    \begin{enumerate}
        \item $(s = s')[t/x] \coloneq (s[t/x] = s'[t/x])$
        \item $R(t_1, \dots, t_n)[t/x] \coloneq R(t_1[t/x], \dots, t_n[t(x)])$
        \item $(\varphi \lor \psi)[t/x] \coloneq (\varphi[t/x] \lor \psi[t/x])$
        \item $(\varphi \wedge \psi)[t/x] \coloneq (\varphi[t/x] \wedge \psi[t/x])$
        \item $(\varphi \to \psi)[t/x] \coloneq (\varphi[t/x] \to \psi[t/x])$
        \item {$ (\forall y. \varphi)[t/x] \coloneq 
            \begin{cases}
                \forall y. \varphi & \text{if } y = x \\
                \forall z. \varphi[z/y][t/x] & \text{otherwise}
            \end{cases}$ where $z$ does not occur in $t$.}
        \item {$ (\exists y. \varphi)[t/x] \coloneq 
        \begin{cases}
            \exists y. \varphi & \text{if } y = x \\
            \exists z. \varphi[z/y][t/x] & \text{otherwise}
        \end{cases}$ where $z$ does not occur in $t$.}
    \end{enumerate}
\end{defi}
\end{boxdef}

\begin{boxdef}
\begin{defi}
    \alert{$\alpha$-equivalence} is the \alert{congruence closure} of 
    \begin{enumerate}
        \item $(\forall x. \varphi) \aeq (\forall y. \varphi[y/x])$
        \item $(\exists x. \varphi) \aeq (\exists y. \varphi[y/x])$
    \end{enumerate}
    i.e. it is the smallest equivalence relation containing these two rules and respecting the connectives: 
    \begin{enumerate}
        \item $(\varphi_1 \wedge \varphi_2) \aeq (\psi_1 \wedge \psi_2)$ for $\varphi_1 \aeq \psi_1$ and $\varphi_2 \aeq \psi_2$
        \item $(\varphi_1 \lor \varphi_2) \aeq (\psi_1 \lor \psi_2)$ for $\varphi_1 \aeq \psi_1$ and $\varphi_2 \aeq \psi_2$
        \item $(\varphi_1 \to \varphi_2) \aeq (\psi_1 \to \psi_2)$ for $\varphi_1 \aeq \psi_1$ and $\varphi_2 \aeq \psi_2$
        \item $(\forall x. \varphi) \aeq (\forall x. \psi)$ if $\varphi \aeq \psi$
        \item $(\exists x. \varphi) \aeq (\exists x. \psi)$ if $\varphi \aeq \psi$
    \end{enumerate}
\end{defi}
\end{boxdef}

\begin{rem}
    We will treat $\alpha$-equivalence as an equivalence relation. 
    You could also define the formulas as $\Phi_{\fL} / \aeq$ and thus treat $\alpha$-equivalence as equality.
\end{rem}

\subsection{Provability}

\begin{boxdef}
\begin{defi}
    Let $\Gamma$ be a set of formulas and $\varphi$ a formula.
    Then $\alert{\Gamma \vdash \varphi}$ (read: ``$\Gamma$ proves $\varphi$'') is defined inductively by 
    \begin{enumerate}
        \item {
            \AxiomC{}
            \UnaryInfC{$\Gamma, \varphi \vdash \varphi$}
            \DisplayProof
            (assumption rule)}
        \item {
            \AxiomC{$\Gamma \vdash \varphi$}
            \AxiomC{$\Gamma \vdash \psi$}
            \BinaryInfC{$\Gamma \vdash \varphi \wedge \psi$}
            \DisplayProof
            ($\wedge$-introduction)}
        \item {
            \AxiomC{$\Gamma \vdash \varphi_1 \wedge \varphi_2$}
            \UnaryInfC{$\Gamma \vdash \varphi_i$}
            \DisplayProof
            for $i = 1, 2$ 
            ($\wedge$-elimination)}
        \item {
            \AxiomC{$\Gamma \vdash \varphi_i$}
            \UnaryInfC{$\Gamma \vdash \varphi_1 \lor \varphi_2$}
            \DisplayProof
            for $i = 1, 2$
            ($\lor$-introduction)}
        \item{
            \AxiomC{$\Gamma \vdash \varphi \lor \psi$}
            \AxiomC{$\Gamma, \varphi \vdash \theta$}
            \AxiomC{$\Gamma, \psi \vdash \theta$}
            \TrinaryInfC{$\Gamma \vdash \theta$}
            \DisplayProof
            ($\lor$-elimination)}
        \item{
            \AxiomC{$\Gamma, \varphi \vdash \psi$}
            \UnaryInfC{$\Gamma \vdash \varphi \to \psi$}
            \DisplayProof
            ($\to$-introduction)}
        \item {
            \AxiomC{$\Gamma \vdash \varphi \to \psi$}
            \AxiomC{$\Gamma \vdash \varphi$}
            \BinaryInfC{$\Gamma \vdash \psi$}
            \DisplayProof
            ($\to$-elimination)}
        \item {
            \AxiomC{$\Gamma, \neg \varphi \vdash \bot$}
            \UnaryInfC{$\Gamma \vdash \varphi$}
            \DisplayProof
            (proof by contradiction)}
        \item {
            \AxiomC{$\Gamma \vdash \varphi$}
            \UnaryInfC{$\Gamma \vdash \forall x. \varphi$}
            \DisplayProof
            for $x \notin \fv{\Gamma}$
            ($\forall$-introduction)}
        \item {
            \AxiomC{$\Gamma \vdash \forall x. \varphi$}
            \UnaryInfC{$\Gamma \vdash \varphi[t/x]$}
            \DisplayProof
            ($\forall$-elimination)}
        \item{
            \AxiomC{$\Gamma \vdash \varphi[t/x]$}
            \UnaryInfC{$\Gamma \vdash \exists x. \varphi$}
            \DisplayProof
            ($\exists$-introduction)}
        \item {
            \AxiomC{$\Gamma \vdash \exists x. \varphi$}
            \AxiomC{$\Gamma, \varphi \vdash \psi$}
            \BinaryInfC{$\Gamma \vdash \psi$}
            \DisplayProof
            for $x \notin \fv{\Gamma, \psi}$
            ($\exists$-elimination)}
        \item {
            \AxiomC{}
            \UnaryInfC{$\Gamma \vdash t = t$}
            \DisplayProof
            ($=$-introduction)}
        \item {
            \AxiomC{$\Gamma \vdash s = t$}
            \AxiomC{$\Gamma \vdash \varphi[t/x]$}
            \BinaryInfC{$\Gamma \vdash \varphi[s/x]$}
            \DisplayProof
            ($=$-elimination)}
        \item {
            \AxiomC{$\Gamma \vdash \varphi$}
            \UnaryInfC{$\Gamma \vdash \psi$}
            \DisplayProof
            for $\varphi \aeq \psi$
            ($\alpha$-equivalence)}
    \end{enumerate}
\end{defi}
\end{boxdef}

\begin{rem}
    Read ``\AxiomC{A} \UnaryInfC{B} \DisplayProof'' as: ``Under the assumptions A we can prove B''.
    With ``$\Gamma, \varphi$'' we really mean $\Gamma \cup \{\varphi\}$.
\end{rem}

\begin{example}
    If $\varphi$ and $\psi$ are formulas then we can show $\vdash (\varphi \wedge \psi) \to (\psi \wedge \varphi)$ using the following \alert{proof tree} : 
    \begin{prooftree}
        \AxiomC{}
        \RightLabel{Assump.}
        \UnaryInfC{$\varphi \wedge \psi \vdash \varphi \wedge \psi$}
        \RightLabel{$\wedge$-elim.}
        \UnaryInfC{$\varphi \wedge \psi \vdash \psi$}
        \AxiomC{}
        \RightLabel{Assump.}
        \UnaryInfC{$\varphi \wedge \psi \vdash \varphi \wedge \psi$}
        \RightLabel{$\wedge$-elim.}
        \UnaryInfC{$\varphi \wedge \psi \vdash \varphi$}
        \RightLabel{$\wedge$-intro.}
        \BinaryInfC{$\varphi \wedge \psi \vdash \psi \wedge \varphi$}
        \RightLabel{$\to$-intro.}
        \UnaryInfC{$\vdash (\varphi \wedge \psi) \to (\psi \wedge \varphi)$}
    \end{prooftree}
\end{example}

\subsection{Semantics}

\begin{boxdef}
\begin{defi}
    An \alert{$\fL$-structure} $\fM$ consists of 
    \begin{enumerate}
        \item a non-empty set $\abs{\fM}$
        \item for any $n$-ary function symbol $f$ a function $f_{\fM} \colon \abs{\fM}^n \to \abs{\fM}$
        \item for any $n$-ary relation symbol $R$ a set $R_{\fM} \subseteq \abs{\fM}^n$
    \end{enumerate}
\end{defi}
\end{boxdef}

\begin{boxdef}
\begin{defi}
    If $t$ is an $\fL$-term and $\sigma \colon \fV \to \abs{\fM}$ we define \alert{$\interpret{t}_{\fM, \sigma}$} as: 
    \begin{enumerate}
        \item $\interpret{x}_{\fM, \sigma} \coloneq \sigma(x)$
        \item $\interpret{f(t_1, \dots, t_n)}_{\fM, \sigma} \coloneq f_{\fM}(\interpret{t_1}_{\fM, \sigma}, \dots, \interpret{t_n}_{\fM, \sigma})$
    \end{enumerate}
    For formulas we define recursively that \alert{$\fM \smodels \varphi$} holds if
    \begin{enumerate}
        \item $\fM \smodels R(t_1, \dots, t_n)$ iff $R_{\fM}(\interpret{t_1}_{\fM, \sigma}, \dots, \interpret{t_n}_{\fM, \sigma})$
        \item $\fM \smodels \bot$ never holds
        \item $\fM \smodels s = t$ iff $\interpret{s}_{\fM, \sigma} = \interpret{t}_{\fM, \sigma}$
        \item $\fM \smodels \varphi \wedge \psi$ iff $\fM \smodels \varphi$ and $\fM \smodels \psi$
        \item $\fM \smodels \varphi \lor \psi$ iff $\fM \smodels \varphi$ or $\fM \smodels \psi$
        \item $\fM \smodels \varphi \to \psi$ iff $\fM \smodels \varphi$ implies $\fM \smodels \psi$
        \item $\fM \smodels \forall x. \varphi$ iff for all $a \in \abs{\fM}$ we know that $\fM \operatorname{\models_{\sigma, x \mapsto a}} \varphi$ where 
        
        $(\sigma, x \mapsto a)(y) \coloneq
        \begin{cases}
            a & y = x \\
            \sigma(y) & \text{otherwise}
        \end{cases}$
        \item $\fM \smodels \exists x. \varphi$ iff there is $a \in \abs{\fM}$ such that $\fM \operatorname{\models_{\sigma, x \mapsto a}} \varphi$
    \end{enumerate}
\end{defi}
\end{boxdef}

\begin{rem}
    We write \alert{$\varphi(\vec{x})$} to mean that $\fv{\varphi} \subseteq \vec{x}$ and \alert{$\varphi(\vec{t})$} for $\varphi[\vec{t}/\vec{x}]$.
\end{rem}

\begin{rem}
    $\interpret{t}_{\fM, \sigma}$ and $\fM \smodels \varphi$ only depend on the values $\sigma(x)$ where $x \in \fv{t}$ and $x \in \fv{\varphi}$ respectively.
    If $\varphi$ is a sentence then $\fM \smodels \varphi$ does not depend on $\sigma$ and is denoted \alert{$\fM \models \varphi$} (read: ``$\fM$ realizes $\varphi$'').
\end{rem}

\begin{boxdef}
\begin{defi}
    If $\Gamma$ is a set of formulas and $\varphi$ is a formula, then \alert{$\Gamma \models \varphi$} means that for any $\fL$-structure $\fM$ and assignment $\sigma \colon \fV \to \abs{\fM}$ such that $\fM \smodels \psi$ for all $\psi \in \Gamma$ we have $\fM \smodels \varphi$.
\end{defi}
\end{boxdef}

\begin{boxthe}
\begin{thm}[Soundness theorem]
    If $\Gamma \vdash \varphi$ then $\Gamma \models \varphi$.
\end{thm}
\end{boxthe}

\begin{boxthe}
\begin{thm}[Completeness theorem]
    If $\Gamma \models \varphi$ then $\Gamma \vdash \varphi$.
\end{thm}
\end{boxthe}

\begin{boxthe}
\begin{thm}[Compactness theorem]
    If $\Gamma \models \varphi$ then for some finite $\Gamma' \subseteq \Gamma$ we have $\Gamma' \models \varphi$.
\end{thm}
\end{boxthe}

\subsection{Definite descriptions}

\begin{boxdef}
\begin{defi}
    $\alert{\exists!}x. \varphi(x, \vec{z}) \coloneq \exists x. (\varphi(x, \vec{z}) \wedge \forall y. \varphi(y, \vec{z}) \to y = x)$.
\end{defi}
\end{boxdef}

\begin{boxdef}
\begin{defi}
    Suppose $\Gamma$ is a set of $\fL$-sentences, $\Gamma'$ a set of $\fL'$-sentences and $\fL \subseteq \fL'$.
    Then \alert{$\Gamma'$ is conservative over $\Gamma$} if $\Gamma \subseteq \Gamma'$ and for all $\fL$-formulas $\psi$ such that $\Gamma' \vdash \psi$ we have $\Gamma \vdash \psi$. 
\end{defi}
\end{boxdef}

\begin{boxthm}
    Suppose that $\Gamma \vdash \forall \vec{x}. \exists! y. \varphi(\vec{x}, y)$ and that $f$ is a fresh function symbol (i.e.\ not among the function symbols of $\fL$) then $\Gamma \cup \{\forall \vec{x}. \varphi(\vec{x}, f(\vec{x}))\}$ is conservative over $\Gamma$.
\end{boxthm}

\begin{boxdef}
\begin{defi}[Axioms of ZFC]
    \alert{$\fL_{ZFC}$} has no function symbol and one binary relation ``$\in$''. 
    The axioms of ZFC are
    \begin{enumerate}
        \item Extensionality : $\forall x \forall y. (\forall z. z \in x \leftrightarrow z \in y) \to x = y$
        \item Pairing: $\forall x \forall y \exists z \forall w. w \in z \leftrightarrow w = x \lor w = y$ (``$z = \{x,y\}$'')
        
        This allows us to define $\{x\} \coloneq \{x, x\}$.
        \item Union: $\forall x \exists y \forall z. z \in y \leftrightarrow \exists w. (w \in x \wedge z \in w)$ (``$y = \bigcup x$'')
        
        This allows us to define $x \cup y \coloneq \bigcup \{x, y\}$.
        \item {Power set: $\forall x \exists y \forall z. z \in y \leftrightarrow z \subseteq x$ (``$y = \pow{x}$'')
            
        where $z \subseteq x$ means $\forall w. w \in z \to w \in x$}
        \item Separation (axiom schema): for any formula $\varphi(\vec{x}, y)$ we have $\forall \vec{x} \forall y \exists z \forall w. w \in z \leftrightarrow (w \in y \wedge \varphi(\vec{x}, w))$ (``$z = \{ w \in y \mid \varphi(\vec{x}, w)\}$'')
        \item Infinity: $\exists x. \varnothing \in x \wedge \forall y. y \in x \to y \cup \{y\} \in x$ where $\varnothing \coloneq \{w \in y \mid \bot\}$
        \item Foundation: $\forall x. (\exists y. y \in x) \to \exists y. y \in x \wedge \forall z. z \in x \to z \notin y$ (``Every set $x$ contains an element $y$ disjoint from $x$'')
        \item Replacement (axiom schema): For every formula $\varphi(z, w, \vec{y})$ we have $\forall x \forall \vec{y} (\forall z. z \in x \to \exists! w \varphi(z, w, \vec{y})) \to \exists u \forall w. w \in u \leftrightarrow \exists z. z \in x \wedge \varphi(z, w, \vec{y})$ (``If $\varphi$ is a function with domain $x$ then the image of $\varphi$ is a set.'')
        \item Choice: $\forall x. \varnothing \notin x \to \exists f. f \in (x \to \bigcup x) \wedge \forall y. y \in x  \to f(y) \in y$
        
        where we define $(x , y) \coloneq \{\{x\}, \{x, y\}\}$, 
        
        $A \times B \coloneq \{ z \in \pow{\pow{A \cup B}} \mid \exists x \in A \exists y \in B. z = (x, y)\}$, 
        
        $(A \to B) \coloneq \{f \in \pow{A \times B} \mid \forall x \in A \exists! y. (x, y) \in f \}$ and
        
        $f(x) \coloneq 
        \begin{cases}
            y & \text{if } (x, y) \in f \\
            \varnothing & \text{if no such $y$ exists} \\
        \end{cases}$
    \end{enumerate}
\end{defi}
\end{boxdef}

\begin{rem}
    The existence of at least one set is provable and therefore the empty set also exists. 
    Nonetheless, the existence of the empty set is often added as an axiom.
\end{rem}

\begin{rem}
    In principle, you can do almost all math in the combination of FOL and ZFC.
    In practice, you want to do meta-logical operations (e.g. quantifying over formulas). 
    For example: 
    \begin{enumerate}
        \item You want to be able to define new definitions with definite descriptions.
        \item You want to be able to talk about theorem schemes.
    \end{enumerate}
    Mizar and Metamath are two proof systems implementing FOL + ZFC in a metalogic.
\end{rem}

\clearpage
\section{\texorpdfstring{$\lambda$}{lambda}-calculus}

\begin{boxdefi}
    Let $\alert{\fV} = \{ x_0, x_1, \dots \} $ a countably infinite set of variables and $\alert{C} = \{ c_0, c_1, \dots \}$ be any set of constants. 
    The terms of the \alert{$\lambda$-calculus} are given by the following BNF: 
    \begin{equation*}
        s,t \Coloneqq x \mid c \mid (s t) \mid (\lambda x. t)
    \end{equation*}
    where $x$ is a variable and $c$ a constant.
\end{boxdefi}

\begin{rem}
    \hfill
    \begin{enumerate}
        \item Think of $(\lambda x. t)$ as the function $x \mapsto t(x)$ and $(s t)$ as function application. 
        \item Anything can apply to any term, e.g. $(x x)$ is a term.
        \item Abbreviate $((rs)t)u$ to $rstu$ and $\lambda x. \lambda y. \lambda z. t$ to $\lambda x y z. t$. For example, $\lambda xy. xy$ means $(\lambda x. (\lambda y. (x y)))$.
        \item $\lambda x.t$ binds the variable $x$. Like in first-order logic we can define $\alpha$-equivalence ($\aeq$) and substitution ($t[s/x]$). Here we identify $\alpha$-equivalent terms, i.e. $\lambda x.x = \lambda y.y$.
    \end{enumerate}
\end{rem}

\begin{example}
    $(x(\lambda x.x))[s/x] = s(\lambda x.x)$
\end{example}

\begin{rem}
    Consider $(\lambda x. t) s$. 
    We want this to correspond to $t[s/x]$. 
\end{rem}

\begin{boxdefi}
    \hfill
    \begin{enumerate}
        \item \alert{$\beta$-contraction ($\becont$)} is defined as $(\lambda x. t) s \becont t [s/x]$.
        \item {\alert{One-step-$\beta$-reduction ($\beored$)} is defined as the compatible closure if $\becont$, i.e.
            \begin{enumerate}
                \item If $s \becont t$ then $s \beored t$.
                \item If $s \beored t$ then $su \beored tu$, $us \beored ut$ and $\lambda x. s \beored \lambda x. t$.
            \end{enumerate}}
        \item \alert{$\beta$-reduction ($\bered$)} is the reflexive transitive closure of $\beored$, i.e. it is the smallest relation that is reflexive, transitive and contains $\beored$.
        \item \alert{$\beta$-equivalence ($\beq$)} is the smallest equivalence relation containing $\bered$.
    \end{enumerate}
\end{boxdefi}

\begin{example}\label{ex:betared}
    \hfill
    \begin{enumerate}
        \item $(\lambda x. xxy)(yz) \becont yz(yz)y$
        \item \adjustbox{valign=t}{
            \begin{tikzcd}
                (\lambda x. xx)y((\lambda z. yz)(ww))  \ar[r, "{\beta , 1}"] \ar[d, "{\beta , 1}"] & (\lambda x.xx)y(y(ww)) \ar[d, "{\beta , 1}"] \\
                yy ((\lambda z. yz)(ww)) \ar[r, "{\beta , 1}"] & yy (y(ww)) \\
            \end{tikzcd}}
            \vspace{-2em}
        \item $(\lambda x.xx) (\lambda x. xx) \beored (\lambda x.xx) (\lambda x. xx)$
    \end{enumerate}
\end{example}

\begin{boxdefi}
    We define the following combinators: 
    \begin{enumerate}
        \item $\alert{I} = \lambda x.x$
        \item $\alert{K} = \lambda xy.x$
        \item $\alert{K_*} = \lambda xy.y$
        \item $\alert{S} = \lambda xyz.xz(yz)$
    \end{enumerate}
\end{boxdefi}

\begin{rem}
    Every term can be defined using $K$ and $S$ up to $\beta$-equivalence.
\end{rem}

\begin{example}
    $SKK \bered \lambda z. Kz(Kz) \bered \lambda z.z = I$.
\end{example}

\begin{boxprop} \label{prop:fixpoi}
    There exists a \alert{fixed-point combinator} $Y$ such that $Y t \bered t(Yt)$.
\end{boxprop}
\begin{proof}
    Let $A \coloneq \lambda fx. x(ffx)$ and let $Y \coloneq AA$ be the \alert{Turing operator}.
    Then $Yt = AAt \bered t (AAt) = t(Yt)$.
\end{proof}

\begin{boxdefi}
    We can define the \alert{pairing} $\alert{P} \coloneq \lambda stx.xst$ and denote $\alert{(s, t)} \coloneq Pst \bered \lambda x. xst$.
\end{boxdefi}

\begin{rem}
    Naming this a pairing makes sense because we have $(s, t) K \bered Kst \bered s$ and $(s, t) K_* \bered K*(s, t) \bered t$.
\end{rem}

\begin{boxdefi}
    We can define \alert{Church numerals}. 
    If $n$ is a natural number we encode it as $\alert{[n]} \coloneq \lambda fx. f^n x$ where $f^0 x \coloneq x$ and $f^{n+1} x = f(f^nx) = f^n(fx)$.
\end{boxdefi}

\begin{boxdefi}
    Let $A$ be a set and $\ored$ a binary relation on $A$ with reflective transitive closure $\to$.
    \begin{enumerate}
        \item $t \in A$ is \alert{in normal form} if there is no $s$ such that $t \ored s$.
        \item $t \in A$ \alert{has normal form $s$} if $s$ is in normal form and $t \to s$.
        \item $t \in A$ is \alert{strongly normalizing} if there exists no infinite sequence $t \ored t_1 \ored t_2 \ored t_3 \ored \cdots$.
        \item $\ored$ is \alert{(weakly) normalizing} if every $t \in A$ has a normal form.
        \item $\ored$ is \alert{strongly normalizing} if every $t \in A$ is strongly normalizing.
        \item {$\ored$ is \alert{confluent} (has the \alert{Church-Rosser property}) if whenever $u \leftarrow t \to v$ there is an $s \in A$ with $u \to s \leftarrow v$.
            \begin{equation*}
            \begin{tikzcd}[ampersand replacement=\&]
                t \ar[r] \ar[d] \& v \ar[d, dashed] \\
                u \ar[r, dashed] \& s 
            \end{tikzcd}
        \end{equation*}}
    \end{enumerate}
\end{boxdefi}

\begin{rem}
    $\beta$-reduction is neither weakly nor strongly normalizing.
\end{rem}
\begin{proof}
    See the counterexamples presented in Example \ref{ex:betared} (iii) and Proposition \ref{prop:fixpoi}.
\end{proof}

\begin{boxthm}[Church-Rosser] \label{thm:ChurchRosser}
    $\beored$ is confluent.
\end{boxthm}

\begin{rem}
    It does not suffice to prove 
    \begin{equation*}
        \begin{tikzcd}
            t \ar[r, "{\beta, 1}"] \ar[d, "{\beta, 1}"] & v \ar[d, dashed, "\beta"] \\
            u \ar[r, dashed, "\beta"] & s 
        \end{tikzcd}
    \end{equation*}
    i.e. for a general binary relation Theorem \ref{thm:ChurchRosser} does not follow from this.
\end{rem}

\begin{boxdefi} \label{def:parared}
    \alert{Parallel reduction ($\Rightarrow$)} is defined inductively as
    \begin{enumerate}
        \item $x \Rightarrow x$ and $c \Rightarrow c$ where $x$ is a variable and $c$ is a constant.
        \item If $t \Rightarrow t'$ then $\lambda x.t \Rightarrow \lambda x. t'$.
    \end{enumerate}
    If $t \Rightarrow t'$ and $u \Rightarrow u'$ then 
    \begin{enumerate}[resume]
        \item $tu \Rightarrow t'u'$.
        \item $(\lambda x.t)u \Rightarrow t'[u'/x]$.
    \end{enumerate}
\end{boxdefi}

\begin{boxlem}
    Parallel induction is reflexive, i.e. $t \Rightarrow t$.
\end{boxlem}
\begin{proof}
    We show this by induction on $t$. 
    Consider $t = \lambda x.s$. 
    Then by induction hypothesis $s \Rightarrow s$, so by Definition \ref{def:parared} (ii) $\lambda x.s \Rightarrow \lambda x.s$.
    The rest of the cases are similar.
\end{proof}

\begin{boxlem} \label{lem:beoredtoarrow}
    If $t \beored s$ then $t \Rightarrow s$.
\end{boxlem}
\begin{proof}
    We show this by induction on $t \beored s$.
    If $t \becont s$, say $(\lambda x.u)v \becont u [v/x]$. 
    Then $(\lambda x. u)v \Rightarrow u [v/x]$ by Definition \ref{def:parared} (iv).
    The other cases are also easy to show.
\end{proof}

\begin{boxlem} \label{lem:arrowbered}
    If $t \Rightarrow t'$ then $t \bered t'$.
\end{boxlem}
\begin{proof}
    We show this by induction on $t \Rightarrow t'$.
    Suppose the last rule was Definition \ref{def:parared} (iv), i.e. concluding $(\lambda x.t)u \Rightarrow t'[u'/x]$ from $t \Rightarrow t'$ and $u \Rightarrow u'$.
    By induction hypothesis we have $t \bered t'$ and $u \bered u'$.
    Then $(\lambda x.t)u \bered (\lambda x.t') u \bered (\lambda x. t') u' \beored t'[u'/x]$.
    The other cases are similar.
\end{proof}

\begin{boxlem} \label{lem:arrowsubst}
    If $t \Rightarrow t'$ and $w \Rightarrow w'$ then $t[w/y] \Rightarrow t'[w'/y]$.
\end{boxlem}
\begin{proof}
    We show this by induction on $t \Rightarrow t'$.
    Suppose the last step was Definition \ref{def:parared} (iv), i.e. concluding $(\lambda x.t)u \Rightarrow t'[u'/x]$ from $t \Rightarrow t'$ and $u \Rightarrow u'$.
    By induction hypothesis we know that $t[w/y] \Rightarrow t'[w/y]$ and $u[w /y] \Rightarrow u'[w/y]$.
    We have to show $(\lambda x.t [w/y])u[w/y] \Rightarrow t'[u'/x][w'/y]$.
    $x$ is bound so we may assume that $x$ is not free in $w'$. 
    Then one can prove that $t'[u'/x][w'/y] = t'[w'/y][(u'[w'/y])/x]$.
    Now the claim follows from Definition \ref{def:parared} (iv).
    The rest of the cases are similar.
\end{proof}

\begin{boxdefi} \label{def:star}
    If $t$ is a term then \alert{$t^*$} is recursively defined as 
    \begin{enumerate}
        \item $x^* = x$ and $c^* = c$ for variables $x$ and constants $c$.
        \item $(\lambda x.t)^* = \lambda x. t^*$.
        \item $(ts)^* = t^*s^*$ if $t$ is not a $\lambda$.
        \item $((\lambda x.t) s)^* = t^*[s^*/x]$.
    \end{enumerate}
\end{boxdefi}

\begin{boxlem} \label{lem:arrowstar}
    If $s \Rightarrow t$ then $t \Rightarrow s^*$.
\end{boxlem}
\begin{proof}
    We show this by induction of the length of the derivation for $s \Rightarrow t$.

    If the last step was Definition \ref{def:parared} (iii), i.e. concluding $tu \Rightarrow t'u'$ from $t \Rightarrow t'$ and $u \Rightarrow u'$, then by induction hypothesis $t' \Rightarrow t^*$ and $u' \Rightarrow u^*$.
    We need to show that $t'u' \Rightarrow (t u)^*$.
    If $t$ is not a $\lambda$ then $(tu)^* = t^*u^*$, so we are done by Definition \ref{def:parared} (iii).
    If $t = \lambda x. s$ then $(tu)^* = s^* [u*/x]$.
    We know $\lambda x.s \Rightarrow t'$ which can only be derived using Definition \ref{def:parared} (ii). 
    So $t' = \lambda x. s'$ with $s \Rightarrow s'$.
    By induction hypothesis $s' \Rightarrow s^*$. 
    Then $(\lambda x. s') u' \Rightarrow s^* [u*/x]$ follows from Definition \ref{def:star} (iv).

    If the last step was Definition \ref{def:parared} (iv), i.e. concluding $(\lambda x.t) u \Rightarrow t'[u'/x]$ from $t \Rightarrow t'$ and $u \Rightarrow u'$, then by induction hypothesis we know that $t' \Rightarrow t^*$ and $u' \Rightarrow u^*$.
    Then $t'[u'/x] \Rightarrow ((\lambda x.t)u)^* = t^*[u^* /x]$ by Lemma \ref{lem:arrowsubst}.

    The rest of the cases are easy.
\end{proof}

\begin{boxlem}\label{lem:almostconfluence}
    If $t \beored u$ and $t \bered v$ then there is an $s$ with $u \bered s \prescript{}{\beta}{\leftarrow} v$.
    \begin{equation*}
        \begin{tikzcd}[ampersand replacement=\&]
            t \ar[r, "\beta"] \ar[d, "{\beta, 1}"] \& v \ar[d, dashed, "\beta"] \\
            u \ar[r, dashed, "\beta"] \& s 
        \end{tikzcd}
    \end{equation*}
\end{boxlem}
\begin{proof}
    Decomposing $t \bered v$ into individual steps gives us
    \begin{equation*}
        \begin{tikzcd}[/tikz/cells={/tikz/nodes={shape=asymmetrical
            rectangle,text width=0.5cm,text height=2ex,text depth=0.3ex,align=center}}]
            & t \ar[dl, "{\beta, 1}"] \ar[dr, "{\beta, 1}"] & \\
            u && v_1 \ar[dr, "{\beta, 1}"] \\
            &&& v_2 \ar[dr, "{\beta, 1}"] \\
            &&&& \ddots \ar[dr, "{\beta, 1}"] \\
            &&&&& \mathllap{v_n} = \mathrlap{v}
        \end{tikzcd}
    \end{equation*}
    which with Lemma \ref{lem:beoredtoarrow} becomes
    \begin{equation*}
        \begin{tikzcd}[/tikz/cells={/tikz/nodes={shape=asymmetrical
            rectangle,text width=0.5cm,text height=2ex,text depth=0.3ex,align=center}}]
            & t \ar[dl, Rightarrow] \ar[dr, Rightarrow] & \\
            u && v_1 \ar[dr, Rightarrow] \\
            &&& v_2 \ar[dr, Rightarrow] \\
            &&&& \ddots \ar[dr, Rightarrow] \\
            &&&&& \mathllap{v_n} = \mathrlap{v}
        \end{tikzcd}
    \end{equation*}
    which with Lemma \ref{lem:arrowstar} yields
    \begin{equation*}
        \begin{tikzcd}[/tikz/cells={/tikz/nodes={shape=asymmetrical
            rectangle,text width=0.5cm,text height=2ex,text depth=0.3ex,align=center}}]
            & t \ar[dl, Rightarrow] \ar[dr, Rightarrow] & \\
            u \ar[dr, Rightarrow] && v_1 \ar[dr, Rightarrow] \ar[dl, Rightarrow] \\
            & t^* \ar[dr, Rightarrow] && v_2 \ar[dr, Rightarrow] \ar[dl, Rightarrow] \\
            && v_1^* \ar[dr, Rightarrow] && \ddots \ar[dr, Rightarrow] \\
            &&& \ddots \ar[dr, Rightarrow] && \mathllap{v_n} = \mathrlap{v} \ar[dl, Rightarrow]\\
            &&&&\mathllap{v}_{n\mathrlap{-1}}^*
        \end{tikzcd}
    \end{equation*}
    which implies our desired statement since by Lemma \ref{lem:arrowbered} parallel induction implies $\beta$-induction and $\beta$-induction is transitive. 
\end{proof}

\begin{exercise}
    Derive Theorem \ref{thm:ChurchRosser} (Church-Rosser) from Lemma \ref{lem:almostconfluence}.
\end{exercise}

\clearpage
\section{Simple Type Theory}

Simple type theory was first presented by Church in 1940. 

\begin{boxdefi}
    We add to $\lamchu$ the type constants 
    \begin{enumerate}
        \item \alert{$I$} (``\alert{individuals}'')
        \item \alert{$\proptype$}
    \end{enumerate}
    and the term constants 
    \begin{enumerate}
        \item $\alert{\forallterm{\tau}} : (\tau \to \proptype) \to \proptype$
        \item $\alert{{\Rightarrow}} : \proptype \to \proptype \to \proptype$
        \item $\alert{=_\tau} : \tau \to \tau \to \proptype$
        \item $\alert{\varepsilon_\tau} : (\tau \to \proptype) \to \tau$
    \end{enumerate}
    We view $A : \proptype$ as a formula. 
    We write: 
    \begin{enumerate}
        \item $\forall x : \tau. A$ for $\forallterm{\tau}(\lambda x : \tau.A)$
        \item $A \Rightarrow B : \proptype$ for ${\Rightarrow} (A)(B)$ where $A, B : \proptype$
        \item $s =_\tau t : \proptype$ for ${=_\tau}(s)(t)$ where $s, t : \tau$
    \end{enumerate}
    Now we can define: 
    \begin{enumerate}
        \item $\alert{\bot} \coloneq \forall P : \proptype. P$
        \item $\alert{\neg A} \coloneq A \Rightarrow \bot$
        \item $\alert{A \vee B} \coloneq \neg A \Rightarrow B$
        \item $\alert{A \wedge B} \coloneq \neg (\neg A \vee \neg B)$
        \item $\alert{A \Leftrightarrow B} \coloneq (A \Rightarrow B) \wedge (B \Rightarrow A)$
        \item $\alert{\exists x : \tau. A} \coloneq \neg (\forall x : \tau. \neg A)$
    \end{enumerate}
\end{boxdefi}

\begin{rem}
    See sheet 6 exercise 6 for a different (equivalent) way to define $A \wedge B$, $A \vee B$ and $\exists x : \tau. A$.
\end{rem}

\begin{boxdefi}
    In addition to the typing judgements $\Gamma \vdash t : \tau$ we can now define \alert{provability judgements} $\alert{\Delta \mathrel{\vdash_\Gamma} A}$ (or $\Delta \vdash A$ for short) where $\Delta$ is a set of terms $B$ such that $\Gamma \vdash B : \proptype$ and $\Gamma \vdash A : \proptype$.
    This gives a proof system with the following rules where the typing constraints are marked in gray: 
    \begin{enumerate}
        \item {(Ass)
            \AxiomC{}
            \UnaryInfC{$\Delta, A \vdash A$}
            \DisplayProof} 
        \item {($\Rightarrow$ I)
            \AxiomC{$\Delta, A \vdash B$}
            \UnaryInfC{$\Delta \vdash A \Rightarrow B$}
            \DisplayProof}
        \item {($\Rightarrow$ E)
            \AxiomC{$\Delta \vdash A \Rightarrow B$}
            \AxiomC{$\Delta \vdash A$}
            \BinaryInfC{$\Delta \vdash B$}
            \DisplayProof}
        \item {($\forall$ I) 
            \AxiomC{$\Delta \vdash A$}
            \AxiomC{$x \notin \fv{\Delta}$}
            \AxiomC{$\color{gray}\Gamma \vdash x : \tau$}
            \TrinaryInfC{$\Delta \vdash \forall x : \tau. A$}
            \DisplayProof}
        \item {($\forall$ E)
            \AxiomC{$\Delta \vdash \forall x : \tau. A$}
            \AxiomC{$\color{gray}\Gamma \vdash t : \tau$}
            \BinaryInfC{$\Delta \vdash A [t/x]$}
            \DisplayProof}
        \item {($=$ Refl)
            \AxiomC{$\color{gray} \Gamma \vdash t : \tau$}
            \UnaryInfC{$\Delta \vdash t \mathrel{=_\tau} t$}
            \DisplayProof}
        \item {($=$ Symm)
            \AxiomC{$\Delta \vdash t \mathrel{=_\tau} s$}
            \UnaryInfC{$\Delta \vdash s \mathrel{=_\tau} t$}
            \DisplayProof}
        \item {($=$ Trans)
            \AxiomC{$\Delta \vdash t \mathrel{=_\tau} s$}
            \AxiomC{$\Delta \vdash s \mathrel{=_\tau} r$}
            \BinaryInfC{$\Delta \vdash t \mathrel{=_\tau} r$}
            \DisplayProof}
        \item {(${\to}$ Congr)
            \AxiomC{$\Delta \vdash t \mathrel{=_{\sigma \to \tau}} t'$}
            \AxiomC{$\Delta \vdash s \mathrel{=_{\sigma}} s'$}
            \BinaryInfC{$\Delta \vdash t s \mathrel{=_\tau} t' s'$}
            \DisplayProof}
        \item {($\beta$ Equiv)
            \AxiomC{$\color{gray} \Gamma, x : \sigma \vdash t : \tau$}
            \AxiomC{$\color{gray} \Gamma \vdash s : \sigma$}
            \BinaryInfC{$\Delta \vdash (\lambda x : \sigma. t) s \mathrel{=_\tau} t [s/x]$}
            \DisplayProof}
        \item To be continued\dots
    \end{enumerate}
\end{boxdefi}

\clearpage
\section{Dependent Type Theory}


The rules of simple type theory (when ignoring the terms) are precisely the rules of implicational logic. 
So, instead of having a type specifically for propositions, we could view types as propositions and terms as proofs. 
This is called \alert{Curry-Howard correspondence} or the \alert{propositions-as-types interpretation}.
We could add other type formers for connectives, e.g.\ $\sigma \times \tau$ corresponding to conjunction and $\sigma + \tau$ corresponding to disjunction.

How do we represent quantifiers?
We will add the the \alert{dependent product type}, or \alert{pi type}, $\prod x : \sigma. \tau$ or $\prod_{x : \sigma} \tau$, corresponding to $\forall x : \sigma. \tau$ and where $\tau$ can depend on $x$.
A term $t : \prod x : \sigma. \tau$ is a dependent function. 
For $s : \sigma$, we get $ts : \tau[s/x]$.

\begin{example}
    We could then define $\tau \coloneq \bR^n$ where $n : \bN$. 
    Then $f$, defined to send $n : \bN$ to $\underbrace{(0, \dots, 0)}_{\text{length }n}$ would have type $\prod_{n : \bN} \bR^n$.
\end{example}

\subsection{Pure Type Systems (PTSs)}

\begin{boxdefi}\label{def:pts}
    A \alert{pure type system (PTS)} is determined by $(\fS, \fA, \fR)$, where 
    \begin{enumerate}
        \item $\fS$ is a set of \alert{sorts} (or \alert{universes})
        \item $\fA \subseteq \fS \times \fS$ is a set of \alert{axioms}
        \item $\fR \subseteq \fS \times \fS \times \fS$ is a set of \alert{relations}
    \end{enumerate}
    We additionally have an infinite set of variables. 
    We can define a lot of things we have seen before again:
    \begin{enumerate}[resume]
        \item{ A PTS has \alert{preterms} 
            \begin{equation*}
                A, B, M, N \Coloneqq x  \mid s \mid (MN) \mid (\lambda x : A. M) \mid \prod x : A. B
            \end{equation*}
            where $x$ is a variable and $s$ a sort.
            If $B$ does not depend on $x$, we write $\prod x : A. B$ as $\alert{A \to B}$.}
        \item {\alert{Contexts} are lists of the form $\Gamma \coloneq x_1 : A_1, x_2 : A_2, \dots, x_n : A_n$.
            Here, the order of the context matters. 
            We set $\alert{\domain{\Gamma}} \coloneq \{x_1, \dots, x_n\}$. }
        \item {We identify terms up to \alert{$\alpha$-equivalence}, so e.g.\ $\lambda x : \tau. x \aeq \lambda y : \tau. y$ and $\prod x : \tau. B(x) \aeq \prod y : \tau. B(y)$.}
        \item {We define \alert{$\beta$-reduction} as before.}
        \item {The \alert{types} are precisely the terms that have a sort as its type.}
    \end{enumerate} 
    We write $s$ for sorts, $A$, $B$ for types and $M$, $N$ for terms (that could be types).
    The typing rules are: 
    \begin{enumerate}[resume]
        \item {(ax) \AxiomC{} \UnaryInfC{$\vdash s_1 : s_2$} \DisplayProof for $(s_1, s_2) \in \fA$}
        \item {(var) \AxiomC{$\Gamma \vdash A : s$} \UnaryInfC{$\Gamma, x : A \vdash x : A$} \DisplayProof where $x \notin \domain{\Gamma}$}
        \item {(weak) \AxiomC{$\Gamma \vdash M : A$} \AxiomC{$\Gamma \vdash B : s$} \BinaryInfC{$\Gamma, x : B \vdash M : A$} \DisplayProof where $x \notin \domain{\Gamma}$}
        \item {(prod) \AxiomC{$\Gamma \vdash A : s_1$} \AxiomC{$\Gamma, x : A \vdash B : s_2$} \BinaryInfC{$\Gamma \vdash \prod x : A. B : s_3$} \DisplayProof for $(s_1, s_2, s_3) \in \fR$}
        \item {(abs) \AxiomC{$\Gamma, x : A \vdash M : B$} \AxiomC{$\Gamma \vdash \prod x : A. B : s$} \BinaryInfC{$\Gamma \vdash \lambda x : A. M : \prod x : A.B$} \DisplayProof}
        \item {(app) \AxiomC{$\Gamma \vdash M : \prod x : A.B$} \AxiomC{$\Gamma \vdash N : A$} \BinaryInfC{$\Gamma \vdash MN : B[N/x]$} \DisplayProof}
        \item {(conv) \AxiomC{$\Gamma \vdash M : A$} \AxiomC{$\Gamma \vdash A' : s$} \BinaryInfC{$\Gamma \vdash M : A'$} \DisplayProof for $A \beq A'$}
    \end{enumerate}
\end{boxdefi}

\begin{example}
    If $\tau : s$, then $(\lambda x : s. x) \tau \beq \tau$. 
    It depends on the PTS whether $(\lambda x : s. x)\tau$ is a well-formed type.
\end{example}

\begin{rem}
    A rule $(s_1, s_2, s_2)$ is abbreviated as $(s_1, s_2)$ and axioms $(s_1, s_2)$ are denoted $s_1 : s_2$.
    $\forall x : A. B$ is another way to write $\prod x : A. B$.
\end{rem}

\begin{boxdefi}
    A pure type system is called \alert{strongly} resp.\ \alert{weakly normalizing} if $\beta$-reduction on well-typed terms (including sorts) is strongly resp.\ weakly normalizing. 
\end{boxdefi}

\begin{example}\label{ex:PTS}
    \hfill
    \begin{enumerate}
        \item {
            Let $\fS = \{ *,  \square\}$, $\fA = \{* : \square\}$ and $\fR = \{(*, *)\}$.
            The only term of type $\square$ is $*$.
            In this system, types cannot depend on terms. 
            This system corresponds to simply-typed $\lambda$-calculus. 
            An example of a derivation in this system is: 
            \begin{prooftree}
                \AxiomC{}
                \LeftLabel{ax}
                \UnaryInfC{$* : \square$}
                \LeftLabel{var}
                \UnaryInfC{$\sigma : * \vdash \sigma : *$}
                \LeftLabel{weak}
                \UnaryInfC{$\sigma : *, \tau : * \vdash \sigma : *$}
                \AxiomC{}
                \LeftLabel{ax}
                \UnaryInfC{$\vdash * : \square$}
                \LeftLabel{weak}
                \UnaryInfC{$\sigma : * \vdash * : \square$}
                \LeftLabel{var}
                \UnaryInfC{$\sigma : *, \tau : * \vdash \tau : *$}
                \LeftLabel{weak}
                \UnaryInfC{$\sigma : *, \tau : *, x : \sigma \vdash \tau : *$}
                \LeftLabel{prod}
                \BinaryInfC{$\sigma : *, \tau : * \vdash \sigma \to \tau : *$}
            \end{prooftree} 
            We can also proof things like
            \begin{equation*}
                \sigma : *, \tau : * \vdash (\sigma \to \sigma) \to \tau \to \sigma : *
            \end{equation*}
            and 
            \begin{equation*}
                \tau : *, \sigma : *, f : \sigma \to \tau, g : \tau \to \tau, x : \tau \vdash g(g(f(x))) : \tau.
            \end{equation*}
            }
        \item {
            Consider $\fS = \{ *,  \square\}$, $\fA = \{* : \square\}$ and $\fR = \{(*, *), (\square, *)\}$. 
            This is called \alert{system F}.
            We can now derive $\vdash \prod \alpha : *. \alpha \to \alpha : *$. 
            The last step of that derivation is 
            \begin{prooftree}
                \AxiomC{$\vdots$}
                \UnaryInfC{$\vdash * : \square$}
                \AxiomC{$\vdots$}
                \UnaryInfC{$\alpha : * \vdash \alpha \to \alpha : *$}
                \LeftLabel{prod}
                \BinaryInfC{$\vdash \prod \alpha : *, \alpha \to \alpha : *$}
            \end{prooftree}
            This is a \alert{polymorphic type}. 
            An inhabitant of this type is $\lambda \alpha : *. \lambda x : \alpha. x$. 
            This is called the \alert{polymorphic identity function}. 
            The word ``\alert{polymorphic}'' means ``for all types at the same time''.
            
            We can give impredicative encodings of connectives. 
            For example, $\bot \coloneq \prod_{\alpha : *} \alpha$ and $A \wedge B \coloneq \prod_{\alpha : *}(A \to B \to \alpha) \to \alpha$.
        }
        \item {
            Let $\fS = \{ *,  \square\}$, $\fA = \{* : \square\}$ and $\fR = \{(*, *), (*, \square)\}$.
            This is called \alert{$\lambda$P}.
            In this system $\alpha : * \vdash \alpha \to * : \square$. 
            This system is where real dependent types start to show up. 
            See this partial derivation: 
            \begin{prooftree}
                \AxiomC{$\vdots$}
                \UnaryInfC{$\alpha : * \vdash \alpha : *$}
                \AxiomC{$\vdots$}
                \UnaryInfC{$\alpha : * \vdash * : \square$}
                \LeftLabel{prod}
                \BinaryInfC{$\alpha : * \vdash \alpha \to * : \square$}
                \LeftLabel{var}
                \UnaryInfC{$\alpha : *, \beta : \alpha \to * \vdash \beta : \alpha \to *$}
                \UnaryInfC{$\vdots$}
                \UnaryInfC{$\alpha : *, \beta : \alpha \to * \vdash \prod x : \alpha. \beta x : *$}
            \end{prooftree}
            We can also show 
            \begin{equation*}
                \alpha : *, \beta : \alpha \to *, \gamma : \alpha \to *, f : \prod_{x : \alpha} \beta x, g : \prod_{x : \alpha} \beta x \to \gamma x \vdash \lambda x : \alpha. gx(fx): \prod_{x : \alpha} \gamma x.
            \end{equation*}
        }
        \item {
            Consider $\fS = \{ *,  \square\}$, $\fA = \{* : \square\}$ and $\fR = \{(*, *), (\square, \square)\}$. 
            This is called \alert{$\lambda \underline{\omega}$}.
            In this system we can derive $\vdash * \to * : \square$ and $\vdash \lambda \alpha : *. \alpha \to \alpha : *$.
            If we add $(\square, *)$ to $\fR$, we can derive
            \begin{equation*}
                \vdash \lambda A : *. \lambda B : *. A \wedge B : * \to * \to *
            \end{equation*}
        }
        \item{
            Let $\fS = \{ *,  \square\}$ and $\fA = \{* : \square\}$.
            \alert{$\lambda$C} or the \alert{calculus of constructions} has all 4 rules from the previous examples, i.e.\ $\fR = \{(*, *), (*, \square), (\square, *), (\square, \square)\}$.

        }
        \item{
            Lean's type theory has the following rules for sorts and function types: 
            \begin{align*}
                \fS &= \{\proptype\} \cup \{ \typeu{u} \mid u \in \bN\} \\
                \fA &= \{\proptype : \typeu{0}\} \cup \{\typeu{u} : \typeu{(u + 1)} \mid u \in \bN\} \\
                \fR &= \{ (\typeu{u}, \typeu{v}, \typeu{(\max u\ v)}) \mid u,v \in \bN\} \cup \{(s, \proptype, \proptype) \mid s \in \fS\}
            \end{align*}
            If $A : \typeu{u}$ and $B : \typeu{v}$, then $A \to B : \typeu{(\max u\ v)}$.
            If $P : A \to \proptype$, then $\forall x : A. Px : \proptype$. 
            If $A : \typeu{u}$ and $C : A \to \typeu{v}$, then $\prod_{x : A} Cx : \typeu{(\max u \ v)}$.
            $\proptype$ is called an \alert{impredicative universe}.
        }
        \item {
            Consider $\fS = \{*\}$, $\fA = \{* :*\}$, $\fR = \{(*, *)\}$. 
            We can construct a term of type $\prod_{\alpha : *} \alpha$. 
            This system is \alert{inconsistent}, i.e.\ every type is inhabited. 
            In this system, weak normalization fails. 
        }
    \end{enumerate}
\end{example}

\begin{boxthm}
    In Example \ref{ex:PTS}, the PTSs (i)-(vi) are strongly normalizing.
\end{boxthm}

\begin{conj}
    If a PTS is weakly normalizing, then it is strongly normalizing. 
\end{conj}

\begin{boxprop}\label{prop:PTS}
    In any PTS 
    \begin{enumerate}
        \item If $\Gamma \vdash M : A$, then $\Gamma \vdash A : s$ or $A = s$.
        \item (Substitution) If $\Gamma, x : A, \Delta \vdash M : B$ and $\Gamma \vdash N : A$, then $\Gamma, \Delta[N/x] \vdash M[N/x] : B[N/x]$.
        \item (Subject reduction) If $\Gamma \vdash M : A$ and $M \bered M'$, then $\Gamma \vdash M' : A$.
    \end{enumerate}
\end{boxprop}

\begin{rem}
    $\vdash * : \square$ shows that we need to consider $A = s$ in Proposition \ref{prop:PTS} (i).
\end{rem}

\begin{boxdefi}
    A PTS is \alert{functional} if 
    \begin{enumerate}
        \item If $(s_1, s_2), (s_1, s_2') \in \fA$, then $s_2 =s_2'$.
        \item If $(r_1, r_2, r_3), (r_1, r_2, r_3') \in \fR$, then $r_3 = r_3'$.
    \end{enumerate}
\end{boxdefi}

\begin{boxthm}[Unique typing]
    In a functional PTS, if $\Gamma \vdash M : A$ and $\Gamma \vdash M : A'$ then $A \beq A'$.
\end{boxthm}

\begin{rem}
    In Lean, $\bN, \bR, \bR^\bR : \typeu{0}$. 
    The category of groups in $\typeu{u}$ has type $\typeu{(u + 1)}$.
\end{rem}

\subsection{Inductive types}

For concreteness we will stick to Lean's type theory. 
Instead of providing the very general and complicated definition of an inductive type, we will be explaining them mostly by example. 

\begin{rem}
    To make talking about $\proptype$ and $\type$ easier, we define $\alert{\sort}$ as $\sortu{0} \coloneq \proptype$ and $\sortu{(u + 1)} \coloneq \typeu{u}$.
\end{rem}

\begin{boxdefi}
    New type formers can be specified using constants and several rules: 
    \begin{enumerate}
        \item \alert{formation rule}: specifying when a type is well-formed
        \item \alert{introduction rules}: specifying how to form an element of the type (using constructors)
        \item \alert{elimination rules}: specifying how you can use elements of the type (using eliminators)
        \item \alert{computation rules}: specifying how an eliminator applied to a constructor simplifies
        \item \alert{uniqueness principle}: specifying a reduction rule involving an arbitrary element of the type
    \end{enumerate}
\end{boxdefi}

\begin{example}
    Not only inductive types follow this schema.
    For example, the dependent product type can be defined in this way: (prod) is the formation rule, (abs) the introduction rule, (app) an elimination rule, $\beta$-reduction is a computation rule, and $\eta$-reduction $\lambda x : A.fx \etred f$ is a uniqueness principle. 
\end{example}

\begin{boxdefi}
    \alert{Cartesian products} can now be defined inductively: 
    \begin{enumerate}
        \item formation rule: \AxiomC{$\Gamma \vdash A : \typeu{u}$} \AxiomC{$\Gamma \vdash B : \typeu{v}$} \BinaryInfC{$\Gamma \vdash A \times B : \typeu{(\maxx{u}{v})}$} \DisplayProof
    \item introduction rule: \AxiomC{$\Gamma \vdash a : A$} \AxiomC{$\Gamma \vdash b : B$} \BinaryInfC{$\Gamma \vdash (a, b) : A \times B$} \DisplayProof
    \item elimination rules: \AxiomC{$\Gamma \vdash v : A \times B$} \UnaryInfC{$\Gamma \vdash \piproj{1}{v} : A$} \DisplayProof and\AxiomC{$\Gamma \vdash A \times B$}\UnaryInfC{$\Gamma \vdash \piproj{2}{v} : B$} \DisplayProof
    \item computation rules: $\piproj{1}{a, b} \bered a$ and $\piproj{2}{a, b} \bered b$
    \item uniqueness principle: $(\piproj{1}{v}, \piproj{2}{v}) \etred v$
    \end{enumerate}
\end{boxdefi}

\begin{boxdefi}
    We can define \alert{$\Sigma$-types} which are also called \alert{dependent sum types} using:
    \begin{enumerate}
        \item formation rule: \AxiomC{$\Gamma \vdash A : \typeu{u}$} \AxiomC{$\Gamma \vdash B : A \to \typeu{v}$} \BinaryInfC{$\Gamma \vdash \sum_{x : A} B(x) : \typeu{(\maxx{u}{v})}$} \DisplayProof
        \item introduction rule: \AxiomC{$\Gamma \vdash a : A$} \AxiomC{$\Gamma \vdash b : B(a)$} \BinaryInfC{$\Gamma \vdash (a, b) : \sum_{x : A} B(x)$} \DisplayProof
        \item elimination rules: \AxiomC{$\Gamma \vdash v : \sum_{x : A} B(x)$} \UnaryInfC{$\Gamma \vdash \piproj{1}{v} : A$} \DisplayProof and\AxiomC{$\Gamma \vdash v : \sum_{x : A} B(x)$}\UnaryInfC{$\Gamma \vdash \piproj{2}{v} : B (\piproj{1}{v})$} \DisplayProof
        \item computation rules: $\piproj{1}{a, b} \bered a$ and $\piproj{2}{a, b} \bered b$
        \item uniqueness rule: $(\piproj{1}{v}, \piproj{2}{v}) \etred v$
    \end{enumerate}
\end{boxdefi}

\begin{rem}
    The formation rule in the above definition could equivalently be stated as 
    \begin{prooftree}
        \AxiomC{$\Gamma \vdash A : \typeu{u}$}
        \AxiomC{$\Gamma, x : A \vdash B : \typeu{v}$}
        \BinaryInfC{$\Gamma \vdash \sum_{x : A} B : \typeu{(\maxx{u}{v})}$}
    \end{prooftree}
\end{rem}

\begin{rem}
    Notice that for $(a, b) : \sum_{a : A} B(a)$, we have $\piproj{2}{a, b} : B(\piproj{1}{a, b})$ and $b : B(a)$. 
    But since $B(\piproj{1}{a, b})$ and $B(a)$ are $\beta$-equivalent, $\piproj{2}{a, b}$ and $b$ have the same type by Definition \ref{def:pts} (conv).
\end{rem}

\begin{example}
    A \alert{magma} is a set (or type) with a binary operation and no axioms.
    We can write this using $\Sigma$-types. 
    The type of magmas is $\sum_{A : \typeu{u}} (A \to A \to A) : \typeu{(u + 1)}$. 
    The type of pointed magmas is $\sum_{A : \typeu{u}} (A \to A \to A) \times A : \typeu{(u + 1)}$.
\end{example}

\begin{rem}
    How can we define a (dependent) function out of $A \times B$?
    Precisely we want to know:
    \begin{enumerate}
        \item Given $C : \sortu{w}$, when is $A \times B \to C$ inhabited?
        \item Given $C : A \times B \to \sortu{w}$, when is $\prod_{v : A \times B} C(v)$ inhabited?
    \end{enumerate} 
    For (i), we need $f : A \to B \to C$, which then gives us $g : A \times B \to C, v \mapsto f(\piproj{1}{v})(\piproj{2}{v})$. 
    For (ii), we need $f : \prod_{a : A}\prod_{b : B}C(a, b)$ to give us $g : \prod_{v : A \times B} C(v), v \mapsto f(\piproj{1}{v})(\piproj{2}{v})$.
    Thus, there is a derivable recursion principle 
    \begin{equation*}
        \rec{A \times B} : (A \to B \to C) \to A \times B \to C
    \end{equation*}
    and a derivable induction principle
    \begin{equation*}
        \ind{A \times B} : \left(\prod_{a : A}\prod_{b : B}C(a, b)\right) \to \prod_{v : A \times B} C(v).
    \end{equation*}
\end{rem}

\begin{rem}
    When we define a function writing $v \mapsto g v$ we actually mean $\lambda v. gv$.
\end{rem}

\begin{boxdefi}
    We can define \alert{coproducts} as follows
    \begin{enumerate}
        \item formation rule: \AxiomC{$\Gamma \vdash A : \typeu{u}$} \AxiomC{$\Gamma \vdash B : \typeu{v}$} \BinaryInfC{$\Gamma \vdash A + B : \typeu{(\maxx{u}{v})}$} \DisplayProof
        \item introduction rules: \AxiomC{$\Gamma \vdash a : A$} \UnaryInfC{$\Gamma \vdash \inl(a) : A + B$} \DisplayProof and\AxiomC{$\Gamma \vdash b : B$}\UnaryInfC{$\inr(b) : A + B$} \DisplayProof
        \item {elimination rule: 

            \def\defaultHypSeparation{\hskip .1in}
            \AxiomC{$\Gamma \vdash C : A + B \to \sortu{w}$} \AxiomC{$\Gamma \vdash f : \prod_{a : A}C(\inl(a))$} \AxiomC{$\Gamma \vdash g : \prod_{b : B} C(\inr(b))$} \TrinaryInfC{$\Gamma \vdash \ind{A + B}(C, f, g) : \prod_{v : A + B}C(v)$} \DisplayProof}
            \def\defaultHypSeparation{\hskip.2in}
        \item computation rules: 

        $\ind{A + B}(C, f, g)(\inl a) \iored f(a)$ and $\ind{A + B}(C, f, g)(\inr b) \iored g(b)$
        \item there is no uniqueness rule
    \end{enumerate}
\end{boxdefi}

\begin{boxdefi}
    We can also define the \alert{natural numbers}: 
    \begin{enumerate}
        \item formation rule: \AxiomC{}\UnaryInfC{$\Gamma \vdash \bN : \typeu{0}$}\DisplayProof
        \item introduction rules: \AxiomC{}\UnaryInfC{$\Gamma \vdash 0 : \bN$}\DisplayProof and \AxiomC{$\Gamma \vdash n : \bN$}\UnaryInfC{$\Gamma \vdash \ssuc{n} : \bN$}\DisplayProof
        \item {elimination rule: 
        
        \AxiomC{$\Gamma \vdash C : \bN \to \sortu{w}$}\AxiomC{$\Gamma \vdash f_0 : C(0)$}\AxiomC{$\Gamma \vdash f_{\ssucOP} : \prod_{n : \bN} C(n) \to C(\ssuc{n})$}\TrinaryInfC{$\Gamma \vdash \ind{\bN}(C, f_0, f_{\ssucOP}) : \prod_{n : \bN} C(n)$}\DisplayProof}
        \item computation rules: $\ind{\bN}(C, f_0, f_{\ssucOP})(0) \iored f_0$ and $\ind{\bN}(C, f_0, f_{\ssucOP})(\ssuc{n}) \iored f_{\ssucOP}(n)(\ind{\bN}(c, f_0, f_{\ssucOP})(n))$
    \end{enumerate}
\end{boxdefi}

\begin{rem}
    We could also write the second introduction rule in the above definition as \AxiomC{}\UnaryInfC{$\Gamma \vdash \ssucOP : \bN \to \bN$}\DisplayProof.
\end{rem}

\begin{rem}
    To define $g : \prod_{n : \bN} C(n)$ we need $g(0) \colonequiv_{\iota} f_0 : C(0)$ and $g(\ssuc{n}) \colonequiv_\iota f_{\ssucOP}(n, g(n)) : C (\ssuc{n})$. 
    This way of defining terms is called \alert{pattern matching}.
\end{rem}

\begin{boxdefi}
    We can define \alert{addition} on $\bN$ using pattern matching: 
    to define $n + {-} : \bN \to \bN$ we set $n + 0 \coloneq n$ and $n + \ssuc{m} = \ssuc{n + m}$.
\end{boxdefi}

\begin{boxdefi}
    We define the \alert{equality type} (also called \alert{identity type}) as the smallest reflexive relation: 
    \begin{enumerate}
        \item formation rule: \AxiomC{$\Gamma \vdash A : \sortu{u}$}\AxiomC{$\Gamma \vdash x : A$}\AxiomC{$\Gamma \vdash y : A$}\TrinaryInfC{$\Gamma \vdash x \mathrel{=_A} y : \proptype$}\DisplayProof
        \item introduction rule: \AxiomC{$\Gamma \vdash x : A$}\UnaryInfC{$\Gamma \vdash \refl{x} : x \mathrel{=_A} x$}\DisplayProof
        \item elimination rule: 
        
        \AxiomC{$\Gamma \vdash C : A \to \sortu{w}$}\AxiomC{$\Gamma \vdash v : C(x)$}\AxiomC{$\Gamma \vdash h : x \mathrel{=_A} y$}\TrinaryInfC{$\Gamma \vdash \rec{=_A}(C, v, h) : C(y)$}\DisplayProof
        \item computation rule: $\rec{=}(C, v, \refl{x}) \iored v$
    \end{enumerate}
\end{boxdefi}

\begin{rem}
    We could have also used \AxiomC{}\UnaryInfC{$\Gamma \vdash \prod_{A : \typeu{u}}\prod_{x : A}x \mathrel{=_A} x$}\DisplayProof as the introduction rule in the above definition. 
\end{rem}

\begin{rem}
    Inductive types are the least/initial types generated by their introduction rules. 
\end{rem}

\begin{rem}
    Lean additionally has: 
    \begin{enumerate}
        \item universes
        \item a general scheme for inductive types
        \item definitional proof irrelevance: if $P : \proptype$ and $h_1, h_2 : P$, then $h_1 \equiv h_2$.
        \item a generalized conversion rule: the conversion rule of Definition \ref{def:pts} is extended to proof-irrelevance, ($\delta$-, $\zeta$-), $\eta$- and $\iota$-reduction.
        \item the ability to make new definitions
        \item let-expressions
        \item analogous inductive propositions ${\wedge}, {\vee}, \dots$
        \item $\mathrm{propext}: \prod_{P, Q : \proptype} (P \leftrightarrow Q) \to P \mathrel{=_\proptype} Q$
        \item $\mathrm{choice} : \prod_{A : \typeu{u}} \mathrm{nonempty}(A) \to A$ where $\mathrm{nonempty}(A)$ us the inductive proposition stating that $A$ is non-empty.
        \item quotient types
    \end{enumerate}
\end{rem}

\begin{example}
    $\delta$- and $\zeta$-reduction relate to unfolding definitions.
    For the definition 
    \begin{center}
        {\ttfamily def two : $\bN$ := s(s(0))}
    \end{center} 
    we get {\ttfamily two $\mathrel{\to_\delta}$ s(s(0))}. 
    Within a definition you might write 
    \begin{center}
        {\ttfamily let two := s(s(0))}, 
    \end{center}
    then {\ttfamily two $\to_{\zeta}$ s(s(0))}.
\end{example}

\begin{boxdefi}
    While \alert{quotient types} are not inductive types, they can be defined using the same schema: 
    \begin{enumerate}
        \item formation rule: \AxiomC{$\Gamma \vdash A : \typeu{u}$} \AxiomC{$\Gamma \vdash R : A \to A \to \proptype$} \BinaryInfC{$\Gamma \vdash A/R : \typeu{u}$} \DisplayProof
        \item introduction rule: \AxiomC{$\Gamma \vdash a : A$} \UnaryInfC{$\Gamma \vdash \llbracket a \rrbracket : A / R$} \DisplayProof
        \item equality: \AxiomC{$\Gamma \vdash h : R a b$} \UnaryInfC{$\Gamma \vdash \operatorname{sound}(h) : \llbracket a \rrbracket \mathrel{=_{A/R} \llbracket b \rrbracket}$} \DisplayProof
        \item{ elimination rules:

                (rec) 
                \def\defaultHypSeparation{\hskip 1mm}
                \AxiomC{$\Gamma \vdash C : \typeu{w}$} 
                \AxiomC{$\Gamma \vdash f : A \to C$} 
                \AxiomC{$\Gamma \vdash h : \prod_{a, b : A} Rab \to f(a) \mathrel{=_C} f(b)$} 
                \TrinaryInfC{$\Gamma \vdash \recOp(C, f, f) : A/R \to C$}
                \DisplayProof
                \def\defaultHypSeparation{\hskip.2in}

                (ind)
                \AxiomC{$\Gamma \vdash P : A/R \to \proptype$}
                \AxiomC{$\Gamma \vdash h : \prod_{a : A}P(\llbracket a \rrbracket)$}
                \BinaryInfC{$\Gamma \vdash \indOp(P, h) : \prod_{v : A/R} P(v)$}
                \DisplayProof}
        \item computation rule: $\recOp(C,f,h) \llbracket a \rrbracket \iored f(a)$
    \end{enumerate}
\end{boxdefi}

\subsection{Homotopy type theory}

Instead of thinking of types as sets we now want to think of them as topological spaces. 
To do that we use Lean's type theory but remove $\proptype$ and $\operatorname{choice}$ and all axioms relating to them. 

\begin{rem}
    Homotopy type theory is a way to do \alert{synthetic homotopy theory}. 
    You might know synthetic geometry, where instead of considering euclidean geometry in the complex plane, the theory is built up from axioms.
\end{rem}

\begin{boxdefi}
    We define a new notion of \alert{equality}:
    \begin{enumerate}
        \item formation rule: \AxiomC{$\Gamma \vdash A : \typeu{u}$} \AxiomC{$\Gamma \vdash x, y : A$} \BinaryInfC{$\Gamma \vdash x \mathrel{=_A} y : \typeu{u}$} \DisplayProof
        \item introduction rule: \AxiomC{$\Gamma \vdash x : A$} \UnaryInfC{$\Gamma \vdash \refl{x} : x \mathrel{=_A} x$} \DisplayProof
        \item elimination rule (\alert{path induction}): 

            \def\defaultHypSeparation{\hskip -1mm}
            \AxiomC{$\Gamma \vdash C : \prod_{x, y: A} (x \mathrel{=_A} y \to \typeu{w})$} \AxiomC{$\Gamma \vdash r : \prod_{x : A} C(x, x, \refl{x})$} \AxiomC{$\Gamma \vdash p : x \mathrel{=_A} y$} \TrinaryInfC{$\Gamma \vdash \ind{=}(C, r, p) : C(x, y, p)$} \DisplayProof
            \def\defaultHypSeparation{\hskip.2in}
            where $x$ and $y$ are distinct variables
        \item computation rule: $\ind{=}(C, r, \refl{x}) \iored r(x)$
    \end{enumerate}
\end{boxdefi}

\begin{rem}
    We interpret this type theory as follows: 
    \begin{enumerate}
        \item $A, B : \typeu{u}$ as topological spaces
        \item $x : A$ as a point in $A$
        \item $p, q : x \mathrel{=_A} y$ as paths from $x$ to $y$
        \item $h : p \mathrel{=_{x \mathrel{=_A} y}} q$ as a homotopy from $p$ to $q$
        \item $f : A \to B$ as a continuous function
    \end{enumerate}
\end{rem}

\begin{boxlem}
    There is a function $\alert{({-})^{-1}} : x \mathrel{=_A} y \to y \mathrel{=_A} x$, $p \mapsto p^{-1}$ such that $\refl{x}^{-1} \equiv \refl{x}$.
\end{boxlem}
\begin{proof}
    We define $({-})^{-1}$ using path induction. 
    Let 
    \begin{equation*}
        C \coloneq (\lambda x,y :A, p : x \mathrel{=_A} y. y \mathrel{=_A} x) : \prod_{x, y : A} x \mathrel{=_A} y \to \typeu{u}
    \end{equation*}
    and 
    \begin{equation*}
        r \coloneq (\lambda x : A. \refl{x}) : \prod_{x : A} C(x, x, \refl{x})
    \end{equation*}
    where the typing is correct because of the conversion rule. 
    Define $p^{-1} \coloneq \ind{=}(C, r, p) : y \mathrel{=_A} x$.
    Then $\refl{x}^{-1} \equiv \ind{=}(C, r, \refl{x}) \equiv r(x) \equiv \refl{x}$.
\end{proof}

\begin{proof}[Informal version of the proof]
    Given $p : x \mathrel{=_A} y$. 
    By path induction, we may assume that $p$ is $\refl{x}$ and $y$ is $x$. 
    Here we are using that the endpoints of $p$ are distinct variables. 
    Now we have to construct an element of type $x \mathrel{=_A} x$. 
    Define $p^{-1} \equiv \refl{x}^{-1} \colonequiv \refl{x}$. 
\end{proof}

\begin{rem}
    Path induction corresponds to the topological fact that path spaces where one or both endpoints are free are contractible, i.e.\ in our interpretation with don't require homotopies to fix endpoints that are free variables.
\end{rem}

\begin{boxlem}
    For $p : x \mathrel{=_A} y$ and $q : y \mathrel{=_A} z$ we construct $\alert{p.q} : x \mathrel{=_A} z$ such that $p.\refl{y} \equiv p$.
\end{boxlem}

\begin{proof}
    By path induction on $q$, we may assume that $z$ is $y$ and $q$ is $\refl{y}$. 
    Now we have to construct $p.\refl{y} : x \mathrel{=_A} y$ which we define as $p$. 
\end{proof}

\begin{boxlem}\label{lem:paths}
    Let $p : x \mathrel{=_A} y$, $q : y \mathrel{=_A} z$ and $r : z \mathrel{=_A} w$. 
    Then
    \begin{enumerate}
        \item $p.p^{-1} \mathrel{=_{x \mathrel{=_A} x}} \refl{x}$.
        \item $p^{-1}.p \mathrel{=_{y \mathrel{=_A} y}} \refl{y}$.
        \item $(p^{-1})^{-1} \mathrel{=_{x \mathrel{=_A} y}} p$.
        \item $(p.q).r \mathrel{=_{x \mathrel{=_A} w}} p.(q.r)$.
    \end{enumerate}
\end{boxlem}

\begin{proof}
    Let us proof (iii). 
    The rest of the lemma is an exercise. 
    By path induction we may assume that $y$ is $x$ and $p$ is $\refl{x}$, so we have to prove $(\refl{x}^{-1})^{-1} = \refl{x}$. 
    We saw that $(\refl{x}^{-1})^{-1} \equiv \refl{x}^{-1} \equiv \refl{x}$.
    So $\refl{\refl{x}} : (\refl{x}^{-1})^{-1} = \refl{x}$.
\end{proof}

\begin{rem}
    This same proof doesn't work to prove $p^{-1} = p$ for $p : x \mathrel{=_A} y$ because already for typing reasons that isn't a valid statement. 
    For $p : x \mathrel{=_A} x$ this still does not work because the endpoints of the path agree and we therefore cannot use path induction.
\end{rem}

\begin{rem}
    Lemma \ref{lem:paths} gives each type the structure of an $\infty$-groupoid. 
\end{rem}

\begin{boxlem}
    If $f : A \to B$ then there is a map $\alert{\mathop{\operatorname{ap}_f}} : x \mathrel{=_A} y \to f(x) \mathrel{=_B} f(y)$ with $\mathop{\operatorname{ap}_f}(\refl{x}) \equiv \refl{f(x)}$.
\end{boxlem}

\begin{boxlem}
    If $P : A \to \typeu{v}$ and $p : x \mathrel{=_A} y$, then $\alert{p_*} : P(x) \to P(y)$ with $(\refl{x})_* \equiv \refl{P(x)}$.
\end{boxlem}

\begin{rem}
    In the lemma above, $P$ gives us a space $\sum_{x : A} P(x)$. 
    We get a fibration $\mathop{\pi_1} : \sum_{x : A} P(x) \twoheadrightarrow A$. 
    Every $f : \sum_{x : A} P(x)$ is a section of that fibration. 
\end{rem}

\begin{boxdefi}
    If $f, g : \prod_{x : A} P(x)$, then a \alert{homotopy} from $f$ to $g$ is an element of $\alert{f \sim g} \coloneq \prod_{x : A} f(x) \mathrel{=_{P(x)}} g(x)$.
\end{boxdefi}

\begin{rem}
    Remember that we interpret all functions as continuous maps, so an element of $f \sim g$ assigns paths continuously. 
\end{rem}

\begin{boxdefi}
    Given $f : A \to B$, $f$ is a \alert{(homotopy) equivalence} if $f$ has a left and right inverse, i.e.\
    \begin{equation*}
        \alert{\operatorname{isequiv}}(f) \coloneq \left( \sum_{g : B \to A} g \circ f \sim \id_A \right) \times \left( \sum_{h : B \to A} f \circ h \sim \id_B \right).
    \end{equation*}
    We set $(\alert{A \simeq B}) \coloneq \sum_{f : A \to B} \operatorname{isequiv}(f)$.
\end{boxdefi}

\begin{boxlem}
    $((x,y) \mathrel{=_{A \times B} (x', y')}) \simeq ((x \mathrel{=_A} x') \times (y \mathrel{=_B} y'))$ is inhabited.
\end{boxlem}

\begin{boxdefi}
    We add the following axioms: 
    \begin{enumerate}
        \item \alert{function extentionality}: $(f \mathrel{=_{\Pi_{x : A}P(x)}} g) \simeq (f \sim g)$ is inhabited.
        \item \alert{univalence}: $(A \mathrel{=_{\typeu{u}}} B) \simeq (A \simeq B)$ is inhabited.
    \end{enumerate}
\end{boxdefi}

\begin{boxthm}
    Univalence implies function extensionality.
\end{boxthm}

\begin{rem}
    With univalence we can prove that there are path spaces with more than one element: 
    consider $2 : \typeu{0}$, the type with two elements. 
    Note that $2 \simeq 2$ simply consists of the bijections from $2$ to itself and thus has two elements. 
    By univalence, $2 \mathrel{=_{\typeu{0}}} 2$ has two distinct elements.
\end{rem}


\begin{rem}
    Using higher inductive types we can construct spheres, suspensions, pullbacks and pushouts. 
\end{rem}

\begin{rem}
    The are categorical models of homotopy type theory such as the simplicial set model. 
\end{rem}



\end{document}