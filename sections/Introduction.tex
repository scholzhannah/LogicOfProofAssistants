\section{Introduction}

The topics of this class are: 
\begin{enumerate}
    \item First-order Logic/ Set Theory
    \item Lambda Calculus
    \item Simple Type Theory (Higher-Order Logic)
    \item Dependent Type Theory/ Homotopy Type Theory
\end{enumerate}

\begin{example}
    Here are examples of proof assistants for these different types of logics: 
    \begin{enumerate}
        \item First-order Logic/ Set Theory: Mizar, Metamath
        \item Simple Type Theory: Isabelle/HoL, HoL Light
        \item Dependent Type Theory: Lean, Rocq (formally Coq), Agda
        \item Homotopy Type Theory: cubicaltt, rezk
    \end{enumerate}
\end{example}

\begin{rem}
    You might want to have the following criteria for a logic: 
    \begin{enumerate}
        \item Appropriate (You can encode mathematical arguments.)
        \item Simple (It is relatively easy to understand.)
        \item Expressive (Mathematical arguments are convenient to express.)
    \end{enumerate}
\end{rem}

\begin{thm}
Let $\pi$ be the prime counting function, i.e. $\pi \colon \bR \to \bN$, $x \mapsto \lvert \{ p \le x \mid p \text{\textup{ prime}} \} \rvert$.
    Then $\lim_{x \to \infty} \frac{\pi (x)}{x / \log(x)} = 1$.
\end{thm}

\begin{rem}
When formalizing/stating this theorem in a formal logic there are a few things that you need to think about: 
\begin{enumerate}
    \item What do you do about division by zero?
    \item What does division even mean? (Do you define division for $\bR$ explicitly? 
        Do you define it generally for a field? 
        Or even for a group? 
        How do you ensure that the ``correct'' field structure on $\bR$ gets used?)
    \item How do you define a limit? (Do you define a limit for $\bR$ explicitly? 
        Or for every topological space?
        How do you ensure the ``correct'' topology on $\bR$ gets used? 
        How do you deal with potentially non-unique limits (for example in non-Hausdorff spaces)?)
\end{enumerate}
\end{rem}

\begin{rem}
You can make the following design choices for ``a logic'': 
\begin{enumerate}
    \item Is the logic typed or untyped?
    \item Is the logic constructive or classical?
    \item Does the logic support computation?
\end{enumerate}
\end{rem}

\begin{rem}
In logic there is the \alert{object language} and we reason about it in a \alert{meta-language} ("ordinary mathematical reasoning").
\end{rem}

\subsection{Inductive Definitions}

\begin{example}
    The natural numbers are inductively defined by $0 \in \bN$ and $S \colon \bN \to \bN$, $n \mapsto n + 1$. 
\end{example}

\begin{defi}
    Let $U$ be a set and $\fC \subseteq \bigcup_{n \in \bN} (U^n \to U)$ a set of \alert{constructors}.
    $c \colon U^n \to U$ is called an \alert{$n$-ary function}.
    \begin{enumerate}
        \item $A \subseteq U$ is \alert{closed under $\fC$} if for any $n$-ary $c \in \fC$ and for all $x_1, \dots , x_n \in A$ we have that $c(x_1, \dots, x_n) \in A$.
        \item $A \subseteq U$ is  \alert{generated by $\fC$} or \alert{inductively defined by $\fC$} if $A$ is the smallest set that is closed under $\fC$, i.e. $A = \bigcap \{B \subseteq U \mid B \text{ is closed under } \fC\}$.
        \item $A \subset U$ is \alert{freely generated by $\fC$} if 
            \begin{enumerate}
                \item each constructor is injective on $A$ and
                \item the images of different constructors are disjoint.
            \end{enumerate}
    \end{enumerate}
\end{defi}

\begin{rem}
$\varnothing$ is closed under $\fC$ iff $\fC$ has no nullary constructors.
\end{rem}

\begin{exercise}
    $\bigcap \{B \subseteq U \mid B \text{\textup{ is closed under }} \fC\}$ is closed under $\fC$.
\end{exercise}

\begin{example}
    \hfill
    \begin{enumerate}
        \item The free group.
        \item The $\sigma$-algebra generated by a collection of subsets. (This is not freely generated.)
        \item The topology generated by a collection of subsets. (This is not freely generated.)
    \end{enumerate}
\end{example}

\begin{thm}[Structural Induction]
    If $A \subset U$ is generated by $\fC$ and $P$ is a predicate on $A$, to prove $\forall a \in A, P(a)$ it suffices to show: for any $n$-ary $c \in \fC$ and any $x_1, \dots, x_n \in A$ if $P(x_1), \dots, P(x_n)$ then $P(c(x_1, \dots, x_n))$.
\end{thm}
\begin{proof}
    Exercise.
\end{proof}

\begin{rem}
    The base case of the induction is given by nullary constructors.
\end{rem}

\begin{thm}[Structural Recursion]
    If $A \subset U$ is freely generated by $\fC$, $B$ is a set and for any $n$-ary $c \in \fC$ we have a $g_c \colon B^n \to B$ then there is a unique function $f \colon A \to B$ such that $f(c(a_1, \dots, a_n)) = g_c(f(a_1), \dots, f(a_n))$ for every $c \in \fC$ and $a_1, \dots a_n \in A$. 
\end{thm}
\begin{proof}
Exercise.
\end{proof}

\begin{example}
    For $A = \bN$ this reduces to $f(0) \coloneq g_0$ and $f(S(n)) \coloneq g_s(f(n))$.
\end{example}