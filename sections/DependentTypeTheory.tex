\section{Dependent Type Theory}


The rules of simple type theory (when ignoring the terms) are precisely the rules of implicational logic. 
So, instead of having a type specifically for propositions, we could view types as propositions and terms as proofs. 
This is called \alert{Curry-Howard correspondence} or the \alert{propositions-as-types interpretation}.
We could add other type formers for connectives, e.g.\ $\sigma \times \tau$ corresponding to conjunction and $\sigma + \tau$ corresponding to disjunction.

How do we represent quantifiers?
We will add the the \alert{dependent product type}, or \alert{pi type}, $\prod x : \sigma. \tau$ or $\prod_{x : \sigma} \tau$, corresponding to $\forall x : \sigma. \tau$ and where $\tau$ can depend on $x$.
A term $t : \prod x : \sigma. \tau$ is a dependent function. 
For $s : \sigma$, we get $ts : \tau[s/x]$.

\begin{example}
    We could then define $\tau \coloneq \bR^n$ where $n : \bN$. 
    Then $f$, defined to send $n : \bN$ to $\underbrace{(0, \dots, 0)}_{\text{length }n}$ would have type $\prod_{n : \bN} \bR^n$.
\end{example}

\subsection{Pure Type Systems (PTSs)}

\begin{boxdefi}\label{def:pts}
    A \alert{pure type system (PTS)} is determined by $(\fS, \fA, \fR)$, where 
    \begin{enumerate}
        \item $\fS$ is a set of \alert{sorts} (or \alert{universes})
        \item $\fA \subseteq \fS \times \fS$ is a set of \alert{axioms}
        \item $\fR \subseteq \fS \times \fS \times \fS$ is a set of \alert{relations}
    \end{enumerate}
    We additionally have an infinite set of variables. 
    We can define a lot of things we have seen before again:
    \begin{enumerate}[resume]
        \item{ A PTS has \alert{preterms} 
            \begin{equation*}
                A, B, M, N \Coloneqq x  \mid s \mid (MN) \mid (\lambda x : A. M) \mid \prod x : A. B
            \end{equation*}
            where $x$ is a variable and $s$ a sort.
            If $B$ does not depend on $x$, we write $\prod x : A. B$ as $\alert{A \to B}$.}
        \item {\alert{Contexts} are lists of the form $\Gamma \coloneq x_1 : A_1, x_2 : A_2, \dots, x_n : A_n$.
            Here, the order of the context matters. 
            We set $\alert{\domain{\Gamma}} \coloneq \{x_1, \dots, x_n\}$. }
        \item {We identify terms up to \alert{$\alpha$-equivalence}, so e.g.\ $\lambda x : \tau. x \aeq \lambda y : \tau. y$ and $\prod x : \tau. B(x) \aeq \prod y : \tau. B(y)$.}
        \item {We define \alert{$\beta$-reduction} as before.}
        \item {The \alert{types} are precisely the terms that have a sort as its type.}
    \end{enumerate} 
    We write $s$ for sorts, $A$, $B$ for types and $M$, $N$ for terms (that could be types).
    The typing rules are: 
    \begin{enumerate}[resume]
        \item {(ax) \AxiomC{} \UnaryInfC{$\vdash s_1 : s_2$} \DisplayProof for $(s_1, s_2) \in \fA$}
        \item {(var) \AxiomC{$\Gamma \vdash A : s$} \UnaryInfC{$\Gamma, x : A \vdash x : A$} \DisplayProof where $x \notin \domain{\Gamma}$}
        \item {(weak) \AxiomC{$\Gamma \vdash M : A$} \AxiomC{$\Gamma \vdash B : s$} \BinaryInfC{$\Gamma, x : B \vdash M : A$} \DisplayProof where $x \notin \domain{\Gamma}$}
        \item {(prod) \AxiomC{$\Gamma \vdash A : s_1$} \AxiomC{$\Gamma, x : A \vdash B : s_2$} \BinaryInfC{$\Gamma \vdash \prod x : A. B : s_3$} \DisplayProof for $(s_1, s_2, s_3) \in \fR$}
        \item {(abs) \AxiomC{$\Gamma, x : A \vdash M : B$} \AxiomC{$\Gamma \vdash \prod x : A. B : s$} \BinaryInfC{$\Gamma \vdash \lambda x : A. M : \prod x : A.B$} \DisplayProof}
        \item {(app) \AxiomC{$\Gamma \vdash M : \prod x : A.B$} \AxiomC{$\Gamma \vdash N : A$} \BinaryInfC{$\Gamma \vdash MN : B[N/x]$} \DisplayProof}
        \item {(conv) \AxiomC{$\Gamma \vdash M : A$} \AxiomC{$\Gamma \vdash A' : s$} \BinaryInfC{$\Gamma \vdash M : A'$} \DisplayProof for $A \beq A'$}
    \end{enumerate}
\end{boxdefi}

\begin{example}
    If $\tau : s$, then $(\lambda x : s. x) \tau \beq \tau$. 
    It depends on the PTS whether $(\lambda x : s. x)\tau$ is a well-formed type.
\end{example}

\begin{rem}
    A rule $(s_1, s_2, s_2)$ is abbreviated as $(s_1, s_2)$ and axioms $(s_1, s_2)$ are denoted $s_1 : s_2$.
    $\forall x : A. B$ is another way to write $\prod x : A. B$.
\end{rem}

\begin{boxdefi}
    A pure type system is called \alert{strongly} resp.\ \alert{weakly normalizing} if $\beta$-reduction on well-typed terms (including sorts) is strongly resp.\ weakly normalizing. 
\end{boxdefi}

\begin{example}\label{ex:PTS}
    \hfill
    \begin{enumerate}
        \item {
            Let $\fS = \{ *,  \square\}$, $\fA = \{* : \square\}$ and $\fR = \{(*, *)\}$.
            The only term of type $\square$ is $*$.
            In this system, types cannot depend on terms. 
            This system corresponds to simply-typed $\lambda$-calculus. 
            An example of a derivation in this system is: 
            \begin{prooftree}
                \AxiomC{}
                \LeftLabel{ax}
                \UnaryInfC{$* : \square$}
                \LeftLabel{var}
                \UnaryInfC{$\sigma : * \vdash \sigma : *$}
                \LeftLabel{weak}
                \UnaryInfC{$\sigma : *, \tau : * \vdash \sigma : *$}
                \AxiomC{}
                \LeftLabel{ax}
                \UnaryInfC{$\vdash * : \square$}
                \LeftLabel{weak}
                \UnaryInfC{$\sigma : * \vdash * : \square$}
                \LeftLabel{var}
                \UnaryInfC{$\sigma : *, \tau : * \vdash \tau : *$}
                \LeftLabel{weak}
                \UnaryInfC{$\sigma : *, \tau : *, x : \sigma \vdash \tau : *$}
                \LeftLabel{prod}
                \BinaryInfC{$\sigma : *, \tau : * \vdash \sigma \to \tau : *$}
            \end{prooftree} 
            We can also proof things like
            \begin{equation*}
                \sigma : *, \tau : * \vdash (\sigma \to \sigma) \to \tau \to \sigma : *
            \end{equation*}
            and 
            \begin{equation*}
                \tau : *, \sigma : *, f : \sigma \to \tau, g : \tau \to \tau, x : \sigma \vdash g(g(f(x))) : \tau.
            \end{equation*}
            }
        \item {
            Consider $\fS = \{ *,  \square\}$, $\fA = \{* : \square\}$ and $\fR = \{(*, *), (\square, *)\}$. 
            This is called \alert{system F}.
            We can now derive $\vdash \prod \alpha : *. \alpha \to \alpha : *$. 
            The last step of that derivation is 
            \begin{prooftree}
                \AxiomC{$\vdots$}
                \UnaryInfC{$\vdash * : \square$}
                \AxiomC{$\vdots$}
                \UnaryInfC{$\alpha : * \vdash \alpha \to \alpha : *$}
                \LeftLabel{prod}
                \BinaryInfC{$\vdash \prod \alpha : *, \alpha \to \alpha : *$}
            \end{prooftree}
            This is a \alert{polymorphic type}. 
            An inhabitant of this type is $\lambda \alpha : *. \lambda x : \alpha. x$. 
            This is called the \alert{polymorphic identity function}. 
            The word ``\alert{polymorphic}'' means ``for all types at the same time''.
            
            We can give impredicative encodings of connectives. 
            For example, $\bot \coloneq \prod_{\alpha : *} \alpha$ and $A \wedge B \coloneq \prod_{\alpha : *}(A \to B \to \alpha) \to \alpha$.
        }
        \item {
            Let $\fS = \{ *,  \square\}$, $\fA = \{* : \square\}$ and $\fR = \{(*, *), (*, \square)\}$.
            This is called \alert{$\lambda$P}.
            In this system $\alpha : * \vdash \alpha \to * : \square$. 
            This system is where real dependent types start to show up. 
            See this partial derivation: 
            \begin{prooftree}
                \AxiomC{$\vdots$}
                \UnaryInfC{$\alpha : * \vdash \alpha : *$}
                \AxiomC{$\vdots$}
                \UnaryInfC{$\alpha : * \vdash * : \square$}
                \LeftLabel{prod}
                \BinaryInfC{$\alpha : * \vdash \alpha \to * : \square$}
                \LeftLabel{var}
                \UnaryInfC{$\alpha : *, \beta : \alpha \to * \vdash \beta : \alpha \to *$}
                \UnaryInfC{$\vdots$}
                \UnaryInfC{$\alpha : *, \beta : \alpha \to * \vdash \prod x : \alpha. \beta x : *$}
            \end{prooftree}
            We can also show 
            \begin{equation*}
                \alpha : *, \beta : \alpha \to *, \gamma : \alpha \to *, f : \prod_{x : \alpha} \beta x, g : \prod_{x : \alpha} \beta x \to \gamma x \vdash \lambda x : \alpha. gx(fx): \prod_{x : \alpha} \gamma x.
            \end{equation*}
        }
        \item {
            Consider $\fS = \{ *,  \square\}$, $\fA = \{* : \square\}$ and $\fR = \{(*, *), (\square, \square)\}$. 
            This is called \alert{$\lambda \underline{\omega}$}.
            In this system we can derive $\vdash * \to * : \square$ and $\vdash \lambda \alpha : *. \alpha \to \alpha : *$.
            If we add $(\square, *)$ to $\fR$, we can derive
            \begin{equation*}
                \vdash \lambda A : *. \lambda B : *. A \wedge B : * \to * \to *
            \end{equation*}
        }
        \item{
            Let $\fS = \{ *,  \square\}$ and $\fA = \{* : \square\}$.
            \alert{$\lambda$C} or the \alert{calculus of constructions} has all 4 rules from the previous examples, i.e.\ $\fR = \{(*, *), (*, \square), (\square, *), (\square, \square)\}$.

        }
        \item{
            Lean's type theory has the following rules for sorts and function types: 
            \begin{align*}
                \fS &= \{\proptype\} \cup \{ \typeu{u} \mid u \in \bN\} \\
                \fA &= \{\proptype : \typeu{0}\} \cup \{\typeu{u} : \typeu{(u + 1)} \mid u \in \bN\} \\
                \fR &= \{ (\typeu{u}, \typeu{v}, \typeu{(\max u\ v)}) \mid u,v \in \bN\} \cup \{(s, \proptype, \proptype) \mid s \in \fS\}
            \end{align*}
            If $A : \typeu{u}$ and $B : \typeu{v}$, then $A \to B : \typeu{(\max u\ v)}$.
            If $P : A \to \proptype$, then $\forall x : A. Px : \proptype$. 
            If $A : \typeu{u}$ and $C : A \to \typeu{v}$, then $\prod_{x : A} Cx : \typeu{(\max u \ v)}$.
            $\proptype$ is called an \alert{impredicative universe}.
        }
        \item {
            Consider $\fS = \{*\}$, $\fA = \{* :*\}$, $\fR = \{(*, *)\}$. 
            We can construct a term of type $\prod_{\alpha : *} \alpha$. 
            This system is \alert{inconsistent}, i.e.\ every type is inhabited. 
            In this system, weak normalization fails. 
        }
    \end{enumerate}
\end{example}

\begin{boxlem}
    In Example \ref{ex:PTS}, the PTSs (i)-(vi) are strongly normalizing.
\end{boxlem}

\begin{boxprop}[Confluence]
    In pure type systems, $\beta$-reduction is confluent.
\end{boxprop}

\begin{conj}
    If a PTS is weakly normalizing, then it is strongly normalizing. 
\end{conj}

\begin{boxprop}\label{prop:PTS}
    In any PTS 
    \begin{enumerate}
        \item If $\Gamma \vdash M : A$, then $\Gamma \vdash A : s$ or $A = s$.
        \item (Substitution) If $\Gamma, x : A, \Delta \vdash M : B$ and $\Gamma \vdash N : A$, then $\Gamma, \Delta[N/x] \vdash M[N/x] : B[N/x]$.
        \item (Subject reduction) If $\Gamma \vdash M : A$ and $M \bered M'$, then $\Gamma \vdash M' : A$.
    \end{enumerate}
\end{boxprop}

\begin{rem}
    $\vdash * : \square$ shows that we need to consider $A = s$ in Proposition \ref{prop:PTS} (i).
\end{rem}

\begin{boxdefi}
    A PTS is \alert{functional} if 
    \begin{enumerate}
        \item If $(s_1, s_2), (s_1, s_2') \in \fA$, then $s_2 =s_2'$.
        \item If $(r_1, r_2, r_3), (r_1, r_2, r_3') \in \fR$, then $r_3 = r_3'$.
    \end{enumerate}
\end{boxdefi}

\begin{boxthm}[Unique typing]
    In a functional PTS, if $\Gamma \vdash M : A$ and $\Gamma \vdash M : A'$ then $A \beq A'$.
\end{boxthm}

\begin{rem}
    In Lean, $\bN, \bR, \bR^\bR : \typeu{0}$. 
    The category of groups in $\typeu{u}$ has type $\typeu{(u + 1)}$.
\end{rem}

\subsection{Inductive types}

For concreteness we will stick to Lean's type theory. 
Instead of providing the very general and complicated definition of an inductive type, we will be explaining them mostly by example. 

\begin{rem}
    To make talking about $\proptype$ and $\type$ easier, we define $\alert{\sort}$ as $\sortu{0} \coloneq \proptype$ and $\sortu{(u + 1)} \coloneq \typeu{u}$.
\end{rem}

\begin{boxdefi}
    New type formers can be specified using constants and several rules: 
    \begin{enumerate}
        \item \alert{formation rule}: specifying when a type is well-formed
        \item \alert{introduction rules}: specifying how to form an element of the type (using constructors)
        \item \alert{elimination rules}: specifying how you can use elements of the type (using eliminators)
        \item \alert{computation rules}: specifying how an eliminator applied to a constructor simplifies
        \item \alert{uniqueness principle}: specifying a reduction rule involving an arbitrary element of the type
    \end{enumerate}
\end{boxdefi}

\begin{example}
    Not only inductive types follow this schema.
    For example, the dependent product type can be defined in this way: (prod) is the formation rule, (abs) the introduction rule, (app) an elimination rule, $\beta$-reduction is a computation rule, and $\eta$-reduction $\lambda x : A.fx \etred f$ is a uniqueness principle. 
\end{example}

\begin{boxdefi}
    \alert{Cartesian products} can now be defined inductively: 
    \begin{enumerate}
        \item formation rule: \AxiomC{$\Gamma \vdash A : \typeu{u}$} \AxiomC{$\Gamma \vdash B : \typeu{v}$} \BinaryInfC{$\Gamma \vdash A \times B : \typeu{(\maxx{u}{v})}$} \DisplayProof
    \item introduction rule: \AxiomC{$\Gamma \vdash a : A$} \AxiomC{$\Gamma \vdash b : B$} \BinaryInfC{$\Gamma \vdash (a, b) : A \times B$} \DisplayProof
    \item elimination rules: \AxiomC{$\Gamma \vdash v : A \times B$} \UnaryInfC{$\Gamma \vdash \piproj{1}{v} : A$} \DisplayProof and\AxiomC{$\Gamma \vdash A \times B$}\UnaryInfC{$\Gamma \vdash \piproj{2}{v} : B$} \DisplayProof
    \item computation rules: $\piproj{1}{a, b} \bered a$ and $\piproj{2}{a, b} \bered b$
    \item uniqueness principle: $(\piproj{1}{v}, \piproj{2}{v}) \etred v$
    \end{enumerate}
\end{boxdefi}

\begin{boxdefi}
    We can define \alert{$\Sigma$-types} which are also called \alert{dependent sum types} using:
    \begin{enumerate}
        \item formation rule: \AxiomC{$\Gamma \vdash A : \typeu{u}$} \AxiomC{$\Gamma \vdash B : A \to \typeu{v}$} \BinaryInfC{$\Gamma \vdash \sum_{x : A} B(x) : \typeu{(\maxx{u}{v})}$} \DisplayProof
        \item introduction rule: \AxiomC{$\Gamma \vdash a : A$} \AxiomC{$\Gamma \vdash b : B(a)$} \BinaryInfC{$\Gamma \vdash (a, b) : \sum_{x : A} B(x)$} \DisplayProof
        \item elimination rules: \AxiomC{$\Gamma \vdash v : \sum_{x : A} B(x)$} \UnaryInfC{$\Gamma \vdash \piproj{1}{v} : A$} \DisplayProof and\AxiomC{$\Gamma \vdash v : \sum_{x : A} B(x)$}\UnaryInfC{$\Gamma \vdash \piproj{2}{v} : B (\piproj{1}{v})$} \DisplayProof
        \item computation rules: $\piproj{1}{a, b} \bered a$ and $\piproj{2}{a, b} \bered b$
        \item uniqueness rule: $(\piproj{1}{v}, \piproj{2}{v}) \etred v$
    \end{enumerate}
\end{boxdefi}

\begin{rem}
    The formation rule in the above definition could equivalently be stated as 
    \begin{prooftree}
        \AxiomC{$\Gamma \vdash A : \typeu{u}$}
        \AxiomC{$\Gamma, x : A \vdash B : \typeu{v}$}
        \BinaryInfC{$\Gamma \vdash \sum_{x : A} B : \typeu{(\maxx{u}{v})}$}
    \end{prooftree}
\end{rem}

\begin{rem}
    Notice that for $(a, b) : \sum_{a : A} B(a)$, we have $\piproj{2}{a, b} : B(\piproj{1}{a, b})$ and $b : B(a)$. 
    But since $B(\piproj{1}{a, b})$ and $B(a)$ are $\beta$-equivalent, $\piproj{2}{a, b}$ and $b$ have the same type by Definition \ref{def:pts} (conv).
\end{rem}

\begin{example}
    A \alert{magma} is a set (or type) with a binary operation and no axioms.
    We can write this using $\Sigma$-types. 
    The type of magmas is $\sum_{A : \typeu{u}} (A \to A \to A) : \typeu{(u + 1)}$. 
    The type of pointed magmas is $\sum_{A : \typeu{u}} (A \to A \to A) \times A : \typeu{(u + 1)}$.
\end{example}

\begin{rem}
    How can we define a (dependent) function out of $A \times B$?
    Precisely we want to know:
    \begin{enumerate}
        \item Given $C : \sortu{w}$, when is $A \times B \to C$ inhabited?
        \item Given $C : A \times B \to \sortu{w}$, when is $\prod_{v : A \times B} C(v)$ inhabited?
    \end{enumerate} 
    For (i), we need $f : A \to B \to C$, which then gives us $g : A \times B \to C, v \mapsto f(\piproj{1}{v})(\piproj{2}{v})$. 
    For (ii), we need $f : \prod_{a : A}\prod_{b : B}C(a, b)$ to give us $g : \prod_{v : A \times B} C(v), v \mapsto f(\piproj{1}{v})(\piproj{2}{v})$.
    Thus, there is a derivable recursion principle 
    \begin{equation*}
        \rec{A \times B} : (A \to B \to C) \to A \times B \to C
    \end{equation*}
    and a derivable induction principle
    \begin{equation*}
        \ind{A \times B} : \left(\prod_{a : A}\prod_{b : B}C(a, b)\right) \to \prod_{v : A \times B} C(v).
    \end{equation*}
\end{rem}

\begin{rem}
    When we define a function writing $v \mapsto g v$ we actually mean $\lambda v. gv$.
\end{rem}

\begin{boxdefi}
    We can define \alert{coproducts} as follows
    \begin{enumerate}
        \item formation rule: \AxiomC{$\Gamma \vdash A : \typeu{u}$} \AxiomC{$\Gamma \vdash B : \typeu{v}$} \BinaryInfC{$\Gamma \vdash A + B : \typeu{(\maxx{u}{v})}$} \DisplayProof
        \item introduction rules: \AxiomC{$\Gamma \vdash a : A$} \UnaryInfC{$\Gamma \vdash \inl(a) : A + B$} \DisplayProof and\AxiomC{$\Gamma \vdash b : B$}\UnaryInfC{$\inr(b) : A + B$} \DisplayProof
        \item {elimination rule: 

            \def\defaultHypSeparation{\hskip .1in}
            \AxiomC{$\Gamma \vdash C : A + B \to \sortu{w}$} \AxiomC{$\Gamma \vdash f : \prod_{a : A}C(\inl(a))$} \AxiomC{$\Gamma \vdash g : \prod_{b : B} C(\inr(b))$} \TrinaryInfC{$\Gamma \vdash \ind{A + B}(C, f, g) : \prod_{v : A + B}C(v)$} \DisplayProof}
            \def\defaultHypSeparation{\hskip.2in}
        \item computation rules: 

        $\ind{A + B}(C, f, g)(\inl a) \iored f(a)$ and $\ind{A + B}(C, f, g)(\inr b) \iored g(b)$
        \item there is no uniqueness rule
    \end{enumerate}
\end{boxdefi}

\begin{boxdefi}
    We can also define the \alert{natural numbers}: 
    \begin{enumerate}
        \item formation rule: \AxiomC{}\UnaryInfC{$\Gamma \vdash \bN : \typeu{0}$}\DisplayProof
        \item introduction rules: \AxiomC{}\UnaryInfC{$\Gamma \vdash 0 : \bN$}\DisplayProof and \AxiomC{$\Gamma \vdash n : \bN$}\UnaryInfC{$\Gamma \vdash \ssuc{n} : \bN$}\DisplayProof
        \item {elimination rule: 
        
        \AxiomC{$\Gamma \vdash C : \bN \to \sortu{w}$}\AxiomC{$\Gamma \vdash f_0 : C(0)$}\AxiomC{$\Gamma \vdash f_{\ssucOP} : \prod_{n : \bN} C(n) \to C(\ssuc{n})$}\TrinaryInfC{$\Gamma \vdash \ind{\bN}(C, f_0, f_{\ssucOP}) : \prod_{n : \bN} C(n)$}\DisplayProof}
        \item computation rules: $\ind{\bN}(C, f_0, f_{\ssucOP})(0) \iored f_0$ and $\ind{\bN}(C, f_0, f_{\ssucOP})(\ssuc{n}) \iored f_{\ssucOP}(n)(\ind{\bN}(c, f_0, f_{\ssucOP})(n))$
    \end{enumerate}
\end{boxdefi}

\begin{rem}
    We could also write the second introduction rule in the above definition as \AxiomC{}\UnaryInfC{$\Gamma \vdash \ssucOP : \bN \to \bN$}\DisplayProof.
\end{rem}

\begin{rem}
    To define $g : \prod_{n : \bN} C(n)$ we need $g(0) \colonequiv_{\iota} f_0 : C(0)$ and $g(\ssuc{n}) \colonequiv_\iota f_{\ssucOP}(n, g(n)) : C (\ssuc{n})$. 
    This way of defining terms is called \alert{pattern matching}.
\end{rem}

\begin{boxdefi}
    We can define \alert{addition} on $\bN$ using pattern matching: 
    to define $n + {-} : \bN \to \bN$ we set $n + 0 \coloneq n$ and $n + \ssuc{m} = \ssuc{n + m}$.
\end{boxdefi}

\begin{boxdefi}
    We define the \alert{equality type} (also called \alert{identity type}) as the smallest reflexive relation: 
    \begin{enumerate}
        \item formation rule: \AxiomC{$\Gamma \vdash A : \sortu{u}$}\AxiomC{$\Gamma \vdash x : A$}\AxiomC{$\Gamma \vdash y : A$}\TrinaryInfC{$\Gamma \vdash x \mathrel{=_A} y : \proptype$}\DisplayProof
        \item introduction rule: \AxiomC{$\Gamma \vdash x : A$}\UnaryInfC{$\Gamma \vdash \refl{x} : x \mathrel{=_A} x$}\DisplayProof
        \item elimination rule: 
        
        \AxiomC{$\Gamma \vdash C : A \to \sortu{w}$}\AxiomC{$\Gamma \vdash v : C(x)$}\AxiomC{$\Gamma \vdash h : x \mathrel{=_A} y$}\TrinaryInfC{$\Gamma \vdash \rec{=_A}(C, v, h) : C(y)$}\DisplayProof
        \item computation rule: $\rec{=}(C, v, \refl{x}) \iored v$
    \end{enumerate}
\end{boxdefi}

\begin{rem}
    We could have also used \AxiomC{}\UnaryInfC{$\Gamma \vdash \prod_{A : \typeu{u}}\prod_{x : A}x \mathrel{=_A} x$}\DisplayProof as the introduction rule in the above definition. 
\end{rem}

\begin{rem}
    Inductive types are the least/initial types generated by their introduction rules. 
\end{rem}

\begin{rem}
    Lean additionally has: 
    \begin{enumerate}
        \item universes
        \item a general scheme for inductive types
        \item definitional proof irrelevance: if $P : \proptype$ and $h_1, h_2 : P$, then $h_1 \equiv h_2$.
        \item a generalized conversion rule: the conversion rule of Definition \ref{def:pts} is extended to proof-irrelevance, ($\delta$-, $\zeta$-), $\eta$- and $\iota$-reduction.
        \item the ability to make new definitions
        \item let-expressions
        \item analogous inductive propositions ${\wedge}, {\vee}, \dots$
        \item $\mathrm{propext}: \prod_{P, Q : \proptype} (P \leftrightarrow Q) \to P \mathrel{=_\proptype} Q$
        \item $\mathrm{choice} : \prod_{A : \typeu{u}} \mathrm{nonempty}(A) \to A$ where $\mathrm{nonempty}(A)$ us the inductive proposition stating that $A$ is non-empty.
        \item quotient types
    \end{enumerate}
\end{rem}

\begin{example}
    $\delta$- and $\zeta$-reduction relate to unfolding definitions.
    For the definition 
    \begin{center}
        {\ttfamily def two : $\bN$ := s(s(0))}
    \end{center} 
    we get {\ttfamily two $\mathrel{\to_\delta}$ s(s(0))}. 
    Within a definition you might write 
    \begin{center}
        {\ttfamily let two := s(s(0))}, 
    \end{center}
    then {\ttfamily two $\to_{\zeta}$ s(s(0))}.
\end{example}

\begin{boxdefi}
    While \alert{quotient types} are not inductive types, they can be defined using the same schema: 
    \begin{enumerate}
        \item formation rule: \AxiomC{$\Gamma \vdash A : \typeu{u}$} \AxiomC{$\Gamma \vdash R : A \to A \to \proptype$} \BinaryInfC{$\Gamma \vdash A/R : \typeu{u}$} \DisplayProof
        \item introduction rule: \AxiomC{$\Gamma \vdash a : A$} \UnaryInfC{$\Gamma \vdash \llbracket a \rrbracket : A / R$} \DisplayProof
        \item equality introduction rule: \AxiomC{$\Gamma \vdash h : R a b$} \UnaryInfC{$\Gamma \vdash \operatorname{sound}(h) : \llbracket a \rrbracket \mathrel{=_{A/R} \llbracket b \rrbracket}$} \DisplayProof
        \item{ elimination rules:

                (rec) 
                \def\defaultHypSeparation{\hskip 1mm}
                \AxiomC{$\Gamma \vdash C : \typeu{w}$} 
                \AxiomC{$\Gamma \vdash f : A \to C$} 
                \AxiomC{$\Gamma \vdash h : \prod_{a, b : A} Rab \to f(a) \mathrel{=_C} f(b)$} 
                \TrinaryInfC{$\Gamma \vdash \recOp(C, f, f) : A/R \to C$}
                \DisplayProof
                \def\defaultHypSeparation{\hskip.2in}

                (ind)
                \AxiomC{$\Gamma \vdash P : A/R \to \proptype$}
                \AxiomC{$\Gamma \vdash h : \prod_{a : A}P(\llbracket a \rrbracket)$}
                \BinaryInfC{$\Gamma \vdash \indOp(P, h) : \prod_{v : A/R} P(v)$}
                \DisplayProof}
        \item computation rule: $\recOp(C,f,h) \llbracket a \rrbracket \iored f(a)$
    \end{enumerate}
\end{boxdefi}

\subsection{Homotopy type theory}

Instead of thinking of types as sets we now want to think of them as topological spaces. 
To do that we use Lean's type theory but remove $\proptype$ and $\operatorname{choice}$ and all axioms relating to them. 

\begin{rem}
    Homotopy type theory is a way to do \alert{synthetic homotopy theory}. 
    You might know synthetic geometry, where instead of considering euclidean geometry in the complex plane, the theory is built up from axioms.
\end{rem}

\begin{boxdefi}
    We define a new notion of \alert{equality}:
    \begin{enumerate}
        \item formation rule: \AxiomC{$\Gamma \vdash A : \typeu{u}$} \AxiomC{$\Gamma \vdash x, y : A$} \BinaryInfC{$\Gamma \vdash x \mathrel{=_A} y : \typeu{u}$} \DisplayProof
        \item introduction rule: \AxiomC{$\Gamma \vdash x : A$} \UnaryInfC{$\Gamma \vdash \refl{x} : x \mathrel{=_A} x$} \DisplayProof
        \item elimination rule (\alert{path induction}): 

            \def\defaultHypSeparation{\hskip -1mm}
            \AxiomC{$\Gamma \vdash C : \prod_{x, y: A} (x \mathrel{=_A} y \to \typeu{w})$} \AxiomC{$\Gamma \vdash r : \prod_{x : A} C(x, x, \refl{x})$} \AxiomC{$\Gamma \vdash p : x \mathrel{=_A} y$} \TrinaryInfC{$\Gamma \vdash \ind{=}(C, r, p) : C(x, y, p)$} \DisplayProof
            \def\defaultHypSeparation{\hskip.2in}
            where $x$ and $y$ are distinct variables
        \item computation rule: $\ind{=}(C, r, \refl{x}) \iored r(x)$
    \end{enumerate}
\end{boxdefi}

\begin{rem}
    We interpret this type theory as follows: 
    \begin{enumerate}
        \item $A, B : \typeu{u}$ as topological spaces
        \item $x : A$ as a point in $A$
        \item $p, q : x \mathrel{=_A} y$ as paths from $x$ to $y$
        \item $h : p \mathrel{=_{x \mathrel{=_A} y}} q$ as a homotopy from $p$ to $q$
        \item $f : A \to B$ as a continuous function
    \end{enumerate}
\end{rem}

\begin{boxlem}
    There is a function $\alert{({-})^{-1}} : x \mathrel{=_A} y \to y \mathrel{=_A} x$, $p \mapsto p^{-1}$ such that $\refl{x}^{-1} \equiv \refl{x}$.
\end{boxlem}
\begin{proof}
    We define $({-})^{-1}$ using path induction. 
    Let 
    \begin{equation*}
        C \coloneq (\lambda x,y :A, p : x \mathrel{=_A} y. y \mathrel{=_A} x) : \prod_{x, y : A} x \mathrel{=_A} y \to \typeu{u}
    \end{equation*}
    and 
    \begin{equation*}
        r \coloneq (\lambda x : A. \refl{x}) : \prod_{x : A} C(x, x, \refl{x})
    \end{equation*}
    where the typing is correct because of the conversion rule. 
    Define $p^{-1} \coloneq \ind{=}(C, r, p) : y \mathrel{=_A} x$.
    Then $\refl{x}^{-1} \equiv \ind{=}(C, r, \refl{x}) \equiv r(x) \equiv \refl{x}$.
\end{proof}

\begin{proof}[Informal version of the proof]
    Given $p : x \mathrel{=_A} y$. 
    By path induction, we may assume that $p$ is $\refl{x}$ and $y$ is $x$. 
    Here we are using that the endpoints of $p$ are distinct variables. 
    Now we have to construct an element of type $x \mathrel{=_A} x$. 
    Define $p^{-1} \equiv \refl{x}^{-1} \colonequiv \refl{x}$. 
\end{proof}

\begin{rem}
    Path induction corresponds to the topological fact that path spaces where one or both endpoints are free are contractible, i.e.\ in our interpretation with don't require homotopies to fix endpoints that are free variables.
\end{rem}

\begin{boxlem}
    For $p : x \mathrel{=_A} y$ and $q : y \mathrel{=_A} z$ we construct $\alert{p.q} : x \mathrel{=_A} z$ such that $p.\refl{y} \equiv p$.
\end{boxlem}

\begin{proof}
    By path induction on $q$, we may assume that $z$ is $y$ and $q$ is $\refl{y}$. 
    Now we have to construct $p.\refl{y} : x \mathrel{=_A} y$ which we define as $p$. 
\end{proof}

\begin{boxlem}\label{lem:paths}
    Let $p : x \mathrel{=_A} y$, $q : y \mathrel{=_A} z$ and $r : z \mathrel{=_A} w$. 
    Then
    \begin{enumerate}
        \item $p.p^{-1} \mathrel{=_{x \mathrel{=_A} x}} \refl{x}$.
        \item $p^{-1}.p \mathrel{=_{y \mathrel{=_A} y}} \refl{y}$.
        \item $(p^{-1})^{-1} \mathrel{=_{x \mathrel{=_A} y}} p$.
        \item $(p.q).r \mathrel{=_{x \mathrel{=_A} w}} p.(q.r)$.
    \end{enumerate}
\end{boxlem}

\begin{proof}
    Let us proof (iii). 
    The rest of the lemma is an exercise. 
    By path induction we may assume that $y$ is $x$ and $p$ is $\refl{x}$, so we have to prove $(\refl{x}^{-1})^{-1} = \refl{x}$. 
    We saw that $(\refl{x}^{-1})^{-1} \equiv \refl{x}^{-1} \equiv \refl{x}$.
    So $\refl{\refl{x}} : (\refl{x}^{-1})^{-1} = \refl{x}$.
\end{proof}

\begin{rem}
    This same proof doesn't work to prove $p^{-1} = p$ for $p : x \mathrel{=_A} y$ because already for typing reasons that isn't a valid statement. 
    For $p : x \mathrel{=_A} x$ this still does not work because the endpoints of the path agree and we therefore cannot use path induction.
\end{rem}

\begin{rem}
    Lemma \ref{lem:paths} gives each type the structure of an $\infty$-groupoid. 
\end{rem}

\begin{boxlem}
    If $f : A \to B$ then there is a map $\alert{\mathop{\operatorname{ap}_f}} : x \mathrel{=_A} y \to f(x) \mathrel{=_B} f(y)$ with $\mathop{\operatorname{ap}_f}(\refl{x}) \equiv \refl{f(x)}$.
\end{boxlem}

\begin{boxlem}
    If $P : A \to \typeu{v}$ and $p : x \mathrel{=_A} y$, then $\alert{p_*} : P(x) \to P(y)$ with $(\refl{x})_* \equiv \refl{P(x)}$.
\end{boxlem}

\begin{rem}
    In the lemma above, $P$ gives us a space $\sum_{x : A} P(x)$. 
    We get a fibration $\mathop{\pi_1} : \sum_{x : A} P(x) \twoheadrightarrow A$. 
    Every $f : \sum_{x : A} P(x)$ is a section of that fibration. 
\end{rem}

\begin{boxdefi}
    If $f, g : \prod_{x : A} P(x)$, then a \alert{homotopy} from $f$ to $g$ is an element of $\alert{f \sim g} \coloneq \prod_{x : A} f(x) \mathrel{=_{P(x)}} g(x)$.
\end{boxdefi}

\begin{rem}
    Remember that we interpret all functions as continuous maps, so an element of $f \sim g$ assigns paths continuously. 
\end{rem}

\begin{boxdefi}
    Given $f : A \to B$, $f$ is a \alert{(homotopy) equivalence} if $f$ has a left and right inverse, i.e.\
    \begin{equation*}
        \alert{\operatorname{isequiv}}(f) \coloneq \left( \sum_{g : B \to A} g \circ f \sim \id_A \right) \times \left( \sum_{h : B \to A} f \circ h \sim \id_B \right).
    \end{equation*}
    We set $(\alert{A \simeq B}) \coloneq \sum_{f : A \to B} \operatorname{isequiv}(f)$.
\end{boxdefi}

\begin{boxlem}
    $((x,y) \mathrel{=_{A \times B} (x', y')}) \simeq ((x \mathrel{=_A} x') \times (y \mathrel{=_B} y'))$ is inhabited.
\end{boxlem}

\begin{boxdefi}
    We add the following axioms: 
    \begin{enumerate}
        \item \alert{function extentionality}: $(f \mathrel{=_{\Pi_{x : A}P(x)}} g) \simeq (f \sim g)$ is inhabited.
        \item \alert{univalence}: $(A \mathrel{=_{\typeu{u}}} B) \simeq (A \simeq B)$ is inhabited.
    \end{enumerate}
\end{boxdefi}

\begin{boxthm}
    Univalence implies function extensionality.
\end{boxthm}

\begin{rem}
    With univalence we can prove that there are path spaces with more than one element: 
    consider $2 : \typeu{0}$, the type with two elements. 
    Note that $2 \simeq 2$ simply consists of the bijections from $2$ to itself and thus has two elements. 
    By univalence, $2 \mathrel{=_{\typeu{0}}} 2$ has two distinct elements.
\end{rem}


\begin{rem}
    Using higher inductive types we can construct spheres, suspensions, pullbacks and pushouts. 
\end{rem}

\begin{rem}
    The are categorical models of homotopy type theory such as the simplicial set model. 
\end{rem}

