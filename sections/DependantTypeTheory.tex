\section{Dependant Type Theory}


The rules of simple type theory (when ignoring the terms) are precisely the rules of implicational logic. 
So, instead of having a type specifically for propositions, we could view types as propositions and terms as proofs. 
This is called \alert{Curry-Howard correspondence} or the \alert{propositions-as-types interpretation}.
We could add other type formers for connectives, e.g.\ $\sigma \times \tau$ corresponding to conjunction and $\sigma + \tau$ corresponding to disjunction.

How do we represent quantifiers?
We will add the the \alert{dependant product type}, or \alert{pi type}, $\prod x : \sigma. \tau$ or $\prod_{x : \sigma} \tau$, corresponding to $\forall x : \sigma. \tau$ and where $\tau$ can depend on $x$.
A term $t : \prod x : \sigma. \tau$ is a dependant function. 
For $s : \sigma$, we get $ts : \tau[s/x]$.

\begin{example}
    We could then define $\tau \coloneq \bR^n$ where $n : \bN$. 
    Then $f$, defined to send $n : \bN$ to $\underbrace{(0, \dots, 0)}_{\text{length }n}$ would have type $\prod_{n : \bN} \bR^n$.
\end{example}

\subsection{Pure Type Systems (PTSs)}

\begin{boxdefi}
    A \alert{pure type system (PTS)} is determined by $(\fS, \fA, \fR)$, where 
    \begin{enumerate}
        \item $\fS$ is a set of \alert{sorts} (or \alert{universes})
        \item $\fA \subseteq \fS \times \fS$ is a set of \alert{axioms}
        \item $\fR \subseteq \fS \times \fS \times \fS$ is a set of \alert{relations}
    \end{enumerate}
    We additionally have an infinite set of variables. 
    We can define a lot of things we have seen before again:
    \begin{enumerate}[resume]
        \item{ A PTS has \alert{preterms} 
            \begin{equation*}
                A, B, M, N \Coloneqq x  \mid s \mid (MN) \mid (\lambda x : A. M) \mid \prod x : A, B
            \end{equation*}
            where $x$ is a variable and $s$ a sort.
            If $B$ does not depend on $x$, we write $\prod x : A. B$ as $\alert{A \to B}$.}
        \item {\alert{Contexts} are lists of the form $\Gamma \coloneq x_1 : A_1, x_2 : A_2, \dots, x_n : A_n$.
            Here, the order of the context matters. 
            We set $\alert{\domain{\Gamma}} \coloneq \{x_1, \dots, x_n\}$. }
        \item {We identify terms up to \alert{$\alpha$-equivalence}, so e.g.\ $\lambda x : \tau. x \aeq \lambda y : \tau. y$ and $\prod x : \tau. B(x) \aeq \prod y : \tau. B(y)$.}
        \item {We define \alert{$\beta$-reduction} as before.}
        \item {The \alert{types} are precisely the terms that have a sort as its type.}
    \end{enumerate} 
    We write $s$ for sorts, $A$, $B$ for types and $M$, $N$ for terms (that could be types).
    The typing rules are: 
    \begin{enumerate}[resume]
        \item {(ax) \AxiomC{} \UnaryInfC{$\vdash s_1 : s_2$} \DisplayProof for $(s_1, s_2) \in \fA$}
        \item {(var) \AxiomC{$\Gamma \vdash A : s$} \UnaryInfC{$\Gamma, x : A \vdash x : A$} \DisplayProof where $x \notin \domain{\Gamma}$}
        \item {(weak) \AxiomC{$\Gamma \vdash M : A$} \AxiomC{$\Gamma \vdash B : s$} \BinaryInfC{$\Gamma, x : B \vdash M : A$} \DisplayProof where $x \notin \domain{\Gamma}$}
        \item {(prod) \AxiomC{$\Gamma \vdash A : s_1$} \AxiomC{$\Gamma, x : A \vdash B : s_2$} \BinaryInfC{$\Gamma \vdash \prod x : A. B : s_3$} \DisplayProof for $(s_1, s_2, s_3) \in \fR$}
        \item {(abs) \AxiomC{$\Gamma, x : A \vdash M : B$} \AxiomC{$\Gamma \vdash \prod x : A. B : s$} \BinaryInfC{$\Gamma \vdash \lambda x : A. M : \prod x : A.B$} \DisplayProof}
        \item {(app) \AxiomC{$\Gamma \vdash M : \prod x : A.B$} \AxiomC{$\Gamma \vdash N : A$} \BinaryInfC{$\Gamma \vdash MN : B[N/x]$} \DisplayProof}
        \item {(conv) \AxiomC{$\Gamma \vdash M : A$} \AxiomC{$\Gamma \vdash A' : s$} \BinaryInfC{$\Gamma \vdash M : A'$} \DisplayProof for $A \beq A'$}
    \end{enumerate}
\end{boxdefi}

\begin{example}
    If $\tau : s$, then $(\lambda x : s. x) \tau \beq \tau$. 
    It depends on the PTS whether $(\lambda x : s. x)\tau$ is a well-formed type.
\end{example}

\begin{rem}
    A rule $(s_1, s_2, s_2)$ is abbreviated as $(s_1, s_2)$ and axioms $(s_1, s_2)$ are denoted $s_1 : s_2$.
\end{rem}

\begin{example}
    \begin{enumerate}
        \item {
            Let $\fS = \{ *,  \square\}$, $\fA = \{* : \square\}$ and $\fR = \{(*, *)\}$.
            The only term of type $\square$ is $*$.
            In this system, types cannot depend on terms. 
            This system corresponds to simply-typed $\lambda$-calculus. 
            An example of a derivation in this system is: 
            \begin{prooftree}
                \AxiomC{}
                \LeftLabel{ax}
                \UnaryInfC{$* : \square$}
                \LeftLabel{var}
                \UnaryInfC{$\sigma : * \vdash \sigma : *$}
                \LeftLabel{weak}
                \UnaryInfC{$\sigma : *, \tau : * \vdash \sigma : *$}
                \AxiomC{}
                \LeftLabel{ax}
                \UnaryInfC{$\vdash * : \square$}
                \LeftLabel{weak}
                \UnaryInfC{$\sigma : * \vdash * : \square$}
                \LeftLabel{var}
                \UnaryInfC{$\sigma : *, \tau : * \vdash \tau : *$}
                \LeftLabel{weak}
                \UnaryInfC{$\sigma : *, \tau : *, x : \sigma \vdash \tau : *$}
                \LeftLabel{prod}
                \BinaryInfC{$\sigma : *, \tau : * \vdash \sigma \to \tau : *$}
            \end{prooftree} 
            We can also proof things like
            \begin{equation*}
                \sigma : *, \tau : * \vdash (\sigma \to \sigma) \to \tau \to \sigma : *
            \end{equation*}
            and 
            \begin{equation*}
                \tau : *, \sigma : *, f : \sigma \to \tau, g : \tau \to \tau, x : \tau \vdash g(g(f(x))) : \tau.
            \end{equation*}
            }
        \item {
            Consider $\fS = \{ *,  \square\}$, $\fA = \{* : \square\}$ and $\fR = \{(*, *), (\square, *)\}$. 
            This is called \alert{system F}.
            We can now derive $\vdash \prod \alpha : *. \alpha \to \alpha : *$. 
            The last step of that derivation is 
            \begin{prooftree}
                \AxiomC{$\vdots$}
                \UnaryInfC{$\vdash * : \square$}
                \AxiomC{$\vdots$}
                \UnaryInfC{$\alpha : * \vdash \alpha \to \alpha : *$}
                \LeftLabel{prod}
                \BinaryInfC{$\vdash \prod \alpha : *, \alpha \to \alpha : *$}
            \end{prooftree}
            This is a \alert{polymorphic type}. 
            An inhabitant of this type is $\lambda \alpha : *. \lambda x : \alpha. x$. 
            This is called the \alert{polymorphic identity function}. 
            The word ``\alert{polymorphic}'' means ``for all types at the same time''.
            
            We can give impredicative encodings of connectives. 
            For example, $\bot \coloneq \prod_{\alpha : *} \alpha$ and $A \wedge B \coloneq \prod_{\alpha : *}(A \to B \to \alpha) \to \alpha$.
        }
        \item {
            Let $\fS = \{ *,  \square\}$, $\fA = \{* : \square\}$ and $\fR = \{(*, *), (*, \square)\}$.
            This is called \alert{$\lambda$P}.
            In this system $\alpha : * \vdash \alpha \to * : \square$. 
            This system is where real dependant types start to show up. 
            See this partial derivation: 
            \begin{prooftree}
                \AxiomC{$\vdots$}
                \UnaryInfC{$\alpha : * \vdash \alpha : *$}
                \AxiomC{$\vdots$}
                \UnaryInfC{$\alpha : * \vdash * : \square$}
                \LeftLabel{prod}
                \BinaryInfC{$\alpha : * \vdash \alpha \to * : \square$}
                \LeftLabel{var}
                \UnaryInfC{$\alpha : *, \beta : \alpha \to * \vdash \beta : \alpha \to *$}
                \UnaryInfC{$\vdots$}
                \UnaryInfC{$\alpha : *, \beta : \alpha \to * \vdash \prod x : \alpha. \beta x : *$}
            \end{prooftree}
            We can also show 
            \begin{equation*}
                \alpha : *, \beta : \alpha \to *, \gamma : \alpha \to *, f : \prod_{x : \alpha} \beta x, g : \prod_{x : \alpha} \beta x \to \gamma x \vdash \lambda x : \alpha. gx(fx): \prod_{x : \alpha} \gamma x.
            \end{equation*}
        }
        \item {
            Consider $\fS = \{ *,  \square\}$, $\fA = \{* : \square\}$ and $\fR = \{(*, *), (\square, \square)\}$. 
            This is called \alert{$\lambda \underline{\omega}$}.
            In this system we can derive $\vdash * \to * : \square$ and $\vdash \lambda \alpha : *. \alpha \to \alpha : *$.
            If we add $(\square, *)$ to $\fR$, we can derive
            \begin{equation*}
                \vdash \lambda A : *. \lambda B : *. A \wedge B : * \to * \to *
            \end{equation*}
        }
        \item{
            Let $\fS = \{ *,  \square\}$ and $\fA = \{* : \square\}$.
            \alert{$\lambda$C} or the \alert{calculus of constructions} has all 4 rules from the previous examples, i.e.\ $\fR = \{(*, *), (*, \square), (\square, *), (\square, \square)\}$.

        }
        \item{
            Lean's type theory has the following rules for sorts and function types: 
            To be continued\dots
        }
    \end{enumerate}
\end{example}