\section{Simple Type Theory}

Simple type theory was first presented by Church in 1940.

\begin{boxdefi}
    We add to $\lamchu$ the type constants
    \begin{enumerate}
        \item \alert{$I$} (``\alert{individuals}'')
        \item \alert{$\proptype$}
    \end{enumerate}
    and the term constants
    \begin{enumerate}
        \item $\alert{\forallterm{\tau}} : (\tau \to \proptype) \to \proptype$
        \item $\alert{{\Rightarrow}} : \proptype \to \proptype \to \proptype$
        \item $\alert{=_\tau} : \tau \to \tau \to \proptype$
        \item $\alert{\varepsilon_\tau} : (\tau \to \proptype) \to \tau$
    \end{enumerate}
    We view $A : \proptype$ as a formula.
    We write:
    \begin{enumerate}
        \item $\forall x : \tau. A$ for $\forallterm{\tau}(\lambda x : \tau.A)$
        \item $A \Rightarrow B : \proptype$ for ${\Rightarrow} (A)(B)$ where $A, B : \proptype$
        \item $s =_\tau t : \proptype$ for ${=_\tau}(s)(t)$ where $s, t : \tau$
    \end{enumerate}
    Now we can define:
    \begin{enumerate}
        \item $\alert{\bot} \coloneq \forall P : \proptype. P$
        \item $\alert{\top} \coloneq \forall P : \proptype. P \Rightarrow P$
        \item $\alert{\neg A} \coloneq A \Rightarrow \bot$
        \item $\alert{A \vee B} \coloneq \forall P:\proptype.\,(A\Rightarrow P)\Rightarrow(B\Rightarrow P)\Rightarrow P$
        \item $\alert{A \wedge B} \coloneq \forall P:\proptype.\,(A\Rightarrow B\Rightarrow P)\Rightarrow P$
        \item $\alert{A \Leftrightarrow B} \coloneq (A \Rightarrow B) \wedge (B \Rightarrow A)$
        \item $\alert{\exists x : \tau. A} \coloneq \forall P:\proptype.\,(\forall x:\tau.\, A\Rightarrow P)\Rightarrow P$
    \end{enumerate}
\end{boxdefi}

\begin{rem}
    When using classical logic, we could alternatively define
    $A\vee B$ and $A\wedge B$ and $\exists x : \tau. A$ as
    $\neg A \Rightarrow B$ and $\neg (\neg A \vee \neg B)$ and $\neg (\forall x : \tau. \neg A)$, respectively.
\end{rem}

\begin{boxdefi}\label{def:provejudge}
    In addition to the typing judgements $\Gamma \vdash t : \tau$ we can now define \alert{provability judgements} $\alert{\Delta \mathrel{\vdash_\Gamma} A}$ (or $\Delta \vdash A$ for short) where $\Delta$ is a set of terms $B$ such that $\Gamma \vdash B : \proptype$ and $\Gamma \vdash A : \proptype$.
    This gives a proof system with the following rules where the typing constraints are marked in gray:
    \begin{enumerate}
        \item {(Ass)
            \AxiomC{}
            \UnaryInfC{$\Delta, A \vdash A$}
            \DisplayProof}
        \item {($\Rightarrow$ I)
            \AxiomC{$\Delta, A \vdash B$}
            \UnaryInfC{$\Delta \vdash A \Rightarrow B$}
            \DisplayProof}
        \item {($\Rightarrow$ E)
            \AxiomC{$\Delta \vdash A \Rightarrow B$}
            \AxiomC{$\Delta \vdash A$}
            \BinaryInfC{$\Delta \vdash B$}
            \DisplayProof}
        \item {($\forall$ I)
            \AxiomC{$\Delta \vdash A$}
            \AxiomC{$x \notin \fv{\Delta}$}
            \AxiomC{$\color{gray}\Gamma \vdash x : \tau$}
            \TrinaryInfC{$\Delta \vdash \forall x : \tau. A$}
            \DisplayProof}
        \item {($\forall$ E)
            \AxiomC{$\Delta \vdash \forall x : \tau. A$}
            \AxiomC{$\color{gray}\Gamma \vdash t : \tau$}
            \BinaryInfC{$\Delta \vdash A [t/x]$}
            \DisplayProof}
        \item {($=$ Refl)
            \AxiomC{$\color{gray} \Gamma \vdash t : \tau$}
            \UnaryInfC{$\Delta \vdash t \mathrel{=_\tau} t$}
            \DisplayProof}
        \item {($=$ Symm)
            \AxiomC{$\Delta \vdash t \mathrel{=_\tau} s$}
            \UnaryInfC{$\Delta \vdash s \mathrel{=_\tau} t$}
            \DisplayProof}
        \item {($=$ Trans)
            \AxiomC{$\Delta \vdash t \mathrel{=_\tau} s$}
            \AxiomC{$\Delta \vdash s \mathrel{=_\tau} r$}
            \BinaryInfC{$\Delta \vdash t \mathrel{=_\tau} r$}
            \DisplayProof}
        \item {(App Congr)
            \AxiomC{$\Delta \vdash t \mathrel{=_{\sigma \to \tau}} t'$}
            \AxiomC{$\Delta \vdash s \mathrel{=_{\sigma}} s'$}
            \BinaryInfC{$\Delta \vdash t s \mathrel{=_\tau} t' s'$}
            \DisplayProof}
        \item {($\beta$ Reduce)
            \AxiomC{$\color{gray} \Gamma, x : \sigma \vdash t : \tau$}
            \AxiomC{$\color{gray} \Gamma \vdash s : \sigma$}
            \BinaryInfC{$\Delta \vdash (\lambda x : \sigma. t) s \mathrel{=_\tau} t [s/x]$}
            \DisplayProof}
        \item {(Conv)
            \AxiomC{$\Delta \vdash A$}
            \AxiomC{$\Delta \vdash A \mathrel{=_\proptype} B$}
            \BinaryInfC{$\Delta \vdash B$}
            \DisplayProof}
        \item {(Propositional extensionality)

            \AxiomC{}
            \UnaryInfC{$\Delta \vdash \forall P, Q : \proptype. (P \Leftrightarrow Q) \Rightarrow P \mathrel{=_\proptype} Q$}
            \DisplayProof}
        \item {(Functional extensionality)

            \AxiomC{}
            \UnaryInfC{$\Delta \vdash \forall f, g : \sigma \to \tau. (\forall x : \sigma. f x \mathrel{=_\tau} g x) \Rightarrow f \mathrel{=_{\sigma \to \tau}} g$}
            \DisplayProof}
        \item {(Infinity)

            \AxiomC{}
            \UnaryInfC{$\Delta \vdash \exists f : I \to I. (\forall x, x' : I. f x \mathrel{=_I} f x' \Rightarrow x \mathrel{=_I} x') \wedge \exists y : I. \forall x : I. \neg (f x \mathrel{=_I} y)$}
            \DisplayProof}
        \item {(Hilbert $\varepsilon_\tau$ function / global choice operator)

            \AxiomC{}
            \UnaryInfC{$\Delta \vdash \forall P : \tau \to \proptype. \forall x : \tau. (P x \Rightarrow P(\varepsilon_\tau P))$}
            \DisplayProof}
    \end{enumerate}
\end{boxdefi}

\begin{rem}
    Interpret $\varepsilon$ as a choice operator.
    $\varepsilon_\tau P$ for $P : \tau \to \proptype$ is an arbitrary choice of a $t : \tau$ such that $P t$ holds if such a $t$ exists.
    One could also define the rule to be:
    \begin{prooftree}
        \AxiomC{}
        \UnaryInfC{$\Delta \vdash \forall P : \proptype. (\exists x : \tau. P x) \Rightarrow P (\varepsilon_\tau P)$}
    \end{prooftree}
\end{rem}

\begin{boxlem}
    We can derive
    \begin{prooftree}
        \AxiomC{$\Delta \vdash P s$}
        \AxiomC{$\Delta \vdash s \mathrel{=_\tau} t$}
        \AxiomC{$\color{gray} \Gamma \vdash P : \tau \to \proptype$}
        \TrinaryInfC{$\Delta \vdash P t$}
    \end{prooftree}
\end{boxlem}

\begin{rem}
    \hfill
    \begin{enumerate}
        \item {All types are non-empty.
            Consider $\varepsilon_\tau(\lambda x : \tau. \bot)$ which has type $\tau$ in any context.}
        \item {Given a type $\tau$, we can define the collection of subsets of $\tau$ as $\alert{(\set \tau)} \coloneq (\tau \to \proptype)$.
        We view $P : \tau \to \proptype$ as the set $\{ x : \tau \mid P x\}$.}
        \item {Proof assistants that implement simple type theory/higher order logic: HOL4, HOL Light, Isabelle/HOL.}
        \item {These proof assistants give another way to define new types:
            Given a type $\tau$ and $P : \tau \to \proptype$, if $P$ is not always false, then you can define the subtype of $\tau$ where $P$ holds.}
    \end{enumerate}
\end{rem}

\begin{example}
    We can define $\alert{\alpha \times \beta}$ as the subtype of $\tau = (\alpha \to \beta \to \proptype)$ where $P : \tau \to \proptype$ holds where $P$ is
    \begin{equation*}
        \lambda f : \tau. \exists x : \alpha \exists y : \beta. f x y \wedge \forall x' : \alpha \forall y' : \beta. f x' y' \Rightarrow x \mathrel{=_\alpha} x' \wedge y \mathrel{=_\beta} y'
    \end{equation*}
\end{example}

\begin{rem}
    If you remove $\varepsilon$ from Definition \ref{def:provejudge}, you get a constructive version of simple type theory.
\end{rem}

\begin{exercise}\label{exe:topnebot}
    Prove that $\neg(\top = \bot)$ holds.
\end{exercise}

\begin{boxprop}[Diaconescu]
    $\forall P : \proptype. P \vee \neg P$.
\end{boxprop}
\begin{proof}
    We give an informal argument.
    Let $P : \proptype$.
    Define
    \begin{equation*}
        U \coloneq (\lambda A : \proptype. (A \mathrel{=_\proptype} \top) \vee P) : \proptype \to \proptype
    \end{equation*}
    and
    \begin{equation*}
        V \coloneq (\lambda A : \proptype. (A \mathrel{=_\proptype} \bot) \vee P) : \proptype \to \proptype
    \end{equation*}
    \begin{claim}
        $P \Rightarrow \varepsilon U = \varepsilon V$
        \begin{proof}
            If $P$ holds then $UA$ and $VA$ are true for any $A : \proptype$ so in particular $UA \Leftrightarrow VA$.
            Therefore by propositional extensionality $UA \mathrel{=_\proptype} VA$ so then by functional extensionality $U \mathrel{=_{\proptype \to \proptype} V}$.
            So $\varepsilon U \mathrel{=_{\proptype}} \varepsilon V$.
        \end{proof}
    \end{claim}
    Notice that $U \top$ is provable, so by the choice axiom $U(\varepsilon U)$ is provable, so $(\varepsilon U = \top) \vee P$ holds.
    Similarly, $V \bot$ holds, so $(\varepsilon V = \bot) \vee P$ does as well.

    Do case distinctions on both of these disjunctions.
    In three cases $P$ holds and we are done.
    In the last case we know $\varepsilon U = \top$ and $\varepsilon V = \bot$.
    Thus by Exercise \ref{exe:topnebot}, $\varepsilon U \neq \varepsilon V$.
    If $P$ holds, then $\varepsilon U = \varepsilon V$ which is a contradiction.
    So we conclude $\neg P$.
\end{proof}

\begin{rem}
    \hfill
    \begin{enumerate}
        \item {
            \AxiomC{$\Delta, P \vdash \bot$}
            \UnaryInfC{$\Delta \vdash \neg P$}
            \DisplayProof
            is constructively valid.}
        \item {
            \AxiomC{$\Delta, \neg P \vdash \bot$}
            \UnaryInfC{$\Delta \vdash P$}
            \DisplayProof
            is not always constructively valid.
        }
    \end{enumerate}
\end{rem}

\begin{example}
    Define $(\exists x : \tau. A) \coloneq \forall P : \proptype. (\forall x : \tau. A \Rightarrow P) \Rightarrow P$.
    We will prove
    \begin{prooftree}
        \AxiomC{$\Gamma \vdash t : \tau$}
        \AxiomC{$\Delta \vdash A [t/x]$}
        \BinaryInfC{$\exists x : \tau. A$}
    \end{prooftree}
    as an example.
    First one would need to show a weakening rule, i.e.\ that for $\Delta \subseteq \Delta'$ we know
    \begin{prooftree}
        \AxiomC{$\Delta \vdash P$}
        \UnaryInfC{$\Delta' \vdash P$}
    \end{prooftree}
    for any $P : \proptype$.
    Set $B \coloneq (\forall x : \tau. A \Rightarrow P)$.
    Then we can prove that claim by the following proof tree
    \begin{prooftree}
        \AxiomC{}
        \LeftLabel{Ass}
        \UnaryInfC{$\Delta, B \vdash B$}
        \AxiomC{$\Gamma \vdash t : \tau$}
        \LeftLabel{$\forall$ E}
        \BinaryInfC{$\Delta, B \vdash A[t/x] \Rightarrow P$}
        \AxiomC{$\Delta \vdash A[t/x]$}
        \LeftLabel{Weak}
        \UnaryInfC{$\Delta, B \vdash A[t/x]$}
        \LeftLabel{$\Rightarrow$ E}
        \BinaryInfC{$\Delta, B \vdash P$}
        \LeftLabel{$\Rightarrow$ I}
        \UnaryInfC{$\Delta \vdash (\forall x : \tau. A \Rightarrow P) \Rightarrow P$}
        \LeftLabel{$\forall$ I}
        \UnaryInfC{$\Delta \vdash \exists x : \tau. A$}
    \end{prooftree}
\end{example}

\begin{example}
    We can define $\tau$ having a group structure by considering
    \begin{equation*}
        \groupstruct \alpha \coloneq (\alpha \to \alpha \to \alpha) \times \alpha \times (\alpha \to \alpha)
    \end{equation*}
    and a predicate $\groupstruct \alpha \to \proptype$ expressing that the group axioms are satisfied where we use the first component to represent the group operation, the second for the identity and the third for the inverse.
\end{example}

\begin{rem}
    Differentiating types and terms can have annoying consequences.
    While we can define $\groupstruct \bZ$, we cannot define the group $\bZ / n \bZ$ as a type for a general $n$.
    This is because $\bZ /n\bZ$ cannot be a type as it depends on a term.
    In the same way we can define $\bR^2$ and $\bR^3$ but cannot define $\bR^n$ as a type for a general $n$.
    Due to the distinction of types and terms, we also cannot quantify over types.
\end{rem}